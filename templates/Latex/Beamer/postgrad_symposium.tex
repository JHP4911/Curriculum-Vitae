\documentclass[10pt]{beamer}    % using the Beamer class for presentations
\usetheme{progressbar}          % using the custom theme "Progressbar"
\usecolortheme{crane}           % using the yellow color theme "crane"
\usepackage{pst-solides3d}      % loads the PSTricks package for 3D graphics
\usepackage{animate}            % loads the package for animations
\usepackage[T1]{fontenc}        % using the "Palatino" font
 \usepackage{iwona}
 \usepackage{palatino}


\usepackage[percent]{overpic}   % used for insertion of graphics
\usepackage{bbding}             % used for the hand symbol (\HandRight)

\setlength{\parskip}{3mm}

\title{\bf The maximally entangled permutation-symmetric quantum state}
\author{Martin Aulbach}
\institute{Post-Graduate Research Symposium \\
  University of Leeds, UK}
\date{June 30, 2010}
% \date{\today}

\begin{document}

\frame[plain]{\titlepage}

\section{Quantum Entanglement}

\frame{
  \frametitle{Quantum Entanglement}
  
  \begin{minipage}{100mm}
    \begin{center}
      \begin{figure}[ht]
        \includegraphics[scale=0.5]{images/entanglement}
      \end{figure}
      {\scriptsize (Cover of Nature, October 2006 Vol 2 No 10) }
    \end{center}
  \end{minipage}
}

\frame{
  Quantum Mechanics allows for \textbf{non-local correlations} between
  spatially separated systems. This is known as \textbf{entanglement}.
  
  \[ \mbox{Example:} \quad \quad | \psi \rangle = \frac{1}{\sqrt{2}} \big(
  \, | \! \uparrow \rangle_{1} | \! \uparrow \rangle_{2} + | \! \downarrow
  \rangle_{1} | \! \downarrow \rangle_{2} \, \big) \]
  
  Measuring the spin of particle 1 yields either $| \! \uparrow
  \rangle_{1}$ or $| \!  \downarrow \rangle_{1}$.
  
  Composite system then collapses into $| \psi \rangle = | \! \uparrow
  \rangle_{1} | \!  \uparrow \rangle_{2}$ or $| \psi \rangle = | \!
  \downarrow \rangle_{1} | \! \downarrow \rangle_{2}$.
  
  \vspace{3mm}
  
  $\:$ \HandRight $\:$ Spin of particle 2 instantaneously affected by
  change of particle 1!
}

\frame{
  \textbf{Applications for Entanglement:}
  
  \begin{itemize}
  \item Quantum Computation (as introduced during Katherine's talk)

  \item Quantum Teleportation
    
  \item Quantum Cryptography
    
  \item \textit{etc.}
  \end{itemize}
  
  \pause
  \vspace{2mm}
  
  $\:$ \HandRight $\:$ Need to quantify the amount of entanglement of
  quantum states
  
  \vspace{2mm}
  
  For example, in \textit{measurement-based quantum computation} the
  entanglement of the initial quantum state must come ``in the right
  dose'', i.e. not too low and not too high.
}

\frame{
  \textbf{Entanglement Measures:}
  
  \begin{itemize}
  \item real-valued function $E( | \psi \rangle )$ on Hilbert-space:
    \quad $E \! : \mathcal{H} \rightarrow \mathbb{R}$
    
  \item \textbf{separable} (non-entangled) states have zero entanglement: $E(|
    \psi \rangle) = 0$

  \item which quantum states are the maximally entangled ones?
  \end{itemize}

  \pause

  $\:$ \HandRight $\:$ Maximally entangled state depends on choice of
  $E$ !

  \vspace{1mm}
  
  $\:$ \HandRight $\:$ \textbf{Here:} Geometric Measure of
  Entanglement $E_G$
  
  \begin{itemize}
  \item defined by scalar product overlap $\langle \psi | \lambda
    \rangle$ with closest separable state $| \lambda \rangle$ (``the
    further away $| \psi \rangle$ is from separable states, the more
    entangled it is'')

  \item somewhat easier to compute than other measures

  \item close links to other entanglement measures
    
  \item \alert{visual representation possible for symmetric quantum states}
  \end{itemize}
}


\section{Symmetric Quantum States}


\frame{
  \frametitle{Permutation-Symmetric Quantum States}
  
  \begin{definition}
    A quantum state $| \psi \rangle$ is \textbf{symmetric} if it
    remains invariant under the permutation of any two of its
    subsystems:
    
    \begin{center}
      $P_{ij} | \psi \rangle = | \psi \rangle$ \quad for all $i,j$
    \end{center}
  \end{definition}
  
  \begin{example}
    $| \psi \rangle = | 001 \rangle \! + \! | 010 \rangle \! + \! |
    100 \rangle$ is symmetric \vspace{2mm}

    \quad \HandRight \quad $P_{12} | \psi \rangle = | 001 \rangle \! +
    \! | 100 \rangle \! + \! | 010 \rangle = | \psi \rangle$
    \checkmark \quad (same for $P_{13}$ and $P_{23}$)
  \end{example}

  \begin{example}
    $| \psi \rangle = | 001 \rangle \! + \! | 010 \rangle \! - \! |
    100 \rangle$ is not symmetric \vspace{2mm}

    \quad \HandRight \quad $P_{12} | \psi \rangle = | 001 \rangle \! +
    \! | 100 \rangle \! - \! | 010 \rangle \neq | \psi \rangle$
  \end{example}
}

\frame{
  \begin{minipage}{65mm}
    A single qubit $| \phi \rangle$ can be represented \\
    by a point on the \textbf{Bloch sphere}:
    \vspace{5mm}

    $| \phi \rangle = \cos \tfrac{\theta}{2} \, |0\rangle +
    \text{e}^{\text{i} \varphi} \sin \tfrac{\theta}{2} |1\rangle$
  \end{minipage}
  \begin{minipage}{40mm}
    \begin{center}
      \begin{figure}[ht]
        \begin{overpic}[scale=.45]{images/bloch_sphere}
          \put(44,61){$\theta$}
          \put(33,46.5){{\footnotesize $\varphi$}}
        \end{overpic}
      \end{figure}
      \vspace{2mm}
    \end{center}
  \end{minipage}
  \vspace{1mm}
  
  \alert{What about quantum systems consisting of several qubits?}
  \vspace{1mm}

  $\:$ \HandRight $\:$ In general no such representation exists.

  $\:$ \HandRight $\:$ \textbf{But:} for symmetric states there exists
  such a representation!
}

\frame{
  \begin{block}{Definition of the \textbf{Majorana Representation}:}
    Any symmetric state of $n$ qubits $| \psi \rangle$ can
    be uniquely represented by $n$ single qubit states $| \phi_1
    \rangle , \dotsc , | \phi_n \rangle$:
    
    \[| \psi \rangle = \sum_{ \text{perm} } | \phi_{1} \rangle |
    \phi_{2} \rangle \dotsm | \phi_{n} \rangle\]
  \end{block}

  $\:$ \HandRight $\:$ Representation of $| \psi \rangle$ via $n$
  points on Bloch sphere

  \vspace{2mm}
  \begin{example}
    \begin{minipage}{52mm}
      \vspace{1mm}
     
      $| \psi \rangle = |001\rangle + |010\rangle + |100\rangle$
      \vspace{1.5mm}
      
      $| \phi_1 \rangle = | 0 \rangle$
      
      $| \phi_2 \rangle = | 0 \rangle$
      
      $| \phi_3 \rangle = | 1 \rangle$
      \vspace{1.5mm}

      the $| \sigma \rangle$ represent the closest \\
      product states
    \end{minipage}
    \begin{minipage}{50mm}
        \begin{figure}[ht]
          \begin{overpic}[scale=.4]{images/bloch_w}
            \put(29,84){{\footnotesize $| \phi_1 \rangle$}}
            \put(54,84){{\footnotesize $| \phi_2 \rangle$}}
            \put(54,10){{\footnotesize $| \phi_3 \rangle$}}
            \put(-10,71){{\footnotesize $| \sigma \rangle$}}
            \put(43,53){{\footnotesize $\theta$}}
          \end{overpic}
        \end{figure}
    \end{minipage}
  \end{example}
}


\section{Spherical Point Distributions}

\frame{
  \frametitle{Spherical Point Distributions}
  
  \begin{minipage}{100mm}
    \begin{center}
      \begin{figure}[ht]
        \includegraphics[scale=0.30]{images/pollen}
      \end{figure}
      {\scriptsize (Martin Oeggerli, Ralf Buchner and Heidemarie
        Halbritter, National Geographic) }
    \end{center}
  \end{minipage}
}

\frame{
  Does our problem have similarities with ``classical'' point
  distribution problems on the sphere?

  \begin{block}{\textbf{T\'{o}th's Problem:}}
    Distribute $n$ points on unit sphere so that the \textbf{minimum}
    distance of all pairs of points is \textbf{maximised}, i.e.
    
    \[ \max_{ \{ \vec{r_k} \} } \, \min_{i \neq j} \, | \vec{r_i} -
    \vec{r_j} | \]
  \end{block}
  
  \begin{block}{\textbf{Thomson's Problem:}}
    Distribute $n$ point charges (interacting through Coulomb's
    inverse square law) on the unit sphere so that the
    \textbf{potential energy} is \textbf{minimised}, i.e.
    
    \[ \min_{ \{ \vec{r_k} \} } \, \sum_{i \neq j} \, \frac{1}{|
      \vec{r_i} - \vec{r_j} |} \]
  \end{block}
}

\frame{
  The vertices of the five \textbf{Platonic solids} are promising
  candidates:
  \vspace{-7mm}
  
  \begin{figure}[ht]
    \hspace*{-7mm}
    \begin{minipage}{140mm}
        \begin{animateinline}[autoplay, loop, poster = first]{6}
          \multiframe{60}{rtet=10+2}{
            \psset{viewpoint=0 -100 30,Decran=58,lightsrc=-50 -100 80, lightintensity=1.2}
            \begin{pspicture}(2.2,2.5)
              \psSolid[object=tetrahedron,a=1,RotZ=\rtet,fillcolor=blue,
              action=draw*](1.98,0,1.6)
            \end{pspicture}
          }
        \end{animateinline}
        \begin{animateinline}[autoplay, loop, poster = first]{6}
          \multiframe{45}{roct=10+2}{
            \psset{viewpoint=0 -100 30,Decran=102,lightsrc=-50 -100 80, lightintensity=1.2}
            \begin{pspicture}(2.2,2.5)
              \psSolid[object=octahedron,a=1,RotZ=\roct,fillcolor=red,
              action=draw*](1.08,0,1.2)
            \end{pspicture}
          }
        \end{animateinline}
        \hspace{0.5mm}
        \begin{animateinline}[autoplay, loop, poster = first]{6}
          \multiframe{45}{rhex=15+2}{
            \psset{viewpoint=0 -100 30,Decran=150,lightsrc=-50 -100 80, lightintensity=1.5}
            \begin{pspicture}(2.2,2.5)
              \psSolid[object=cube,a=1,RotZ=\rhex,fillcolor=orange,
              action=draw*](0.75,0,0.8)
            \end{pspicture}
          }
        \end{animateinline}
        \hspace{0.5mm}
        \begin{animateinline}[autoplay, loop, poster = first]{6}
          \multiframe{36}{rico=0+2}{
            \psset{viewpoint=0 -100 30,Decran=115,lightsrc=-50 -100 80, lightintensity=1.2}
            \begin{pspicture}(2.2,2.5)
              \psSolid[object=icosahedron,a=1,RotZ=\rico,fillcolor=yellow,
              action=draw*](1.0,0,1.05)
            \end{pspicture}
          }
        \end{animateinline}
        \hspace{0.5mm}
        \begin{animateinline}[autoplay, loop, poster = first]{6}
          \multiframe{36}{rdod=0+2}{
            \psset{viewpoint=0 -100 30,Decran=110,lightsrc=-50 -100 80, lightintensity=1.2}
            \begin{pspicture}(2.2,2.5)
              \psSolid[object=dodecahedron,a=1,RotZ=\rdod,fillcolor=green,
              action=draw*](1.05,0,1.1)
            \end{pspicture}
          }
        \end{animateinline}
        \end{minipage}
  \end{figure}

  \vspace{-1mm}
  \begin{minipage}{140mm}
    \hspace*{-5mm} tetrahedron \quad octahedron \quad $\,\,$
    hexahedron \quad $\,\,$ icosahedron \quad dodecahedron
  \end{minipage}

  \vspace{-2mm}
  $n = 4$ \hspace{12mm} $n = 6$ \hspace{14mm} $n = 8$ \hspace{13mm} $n
  = 12$ \hspace{12mm} $n = 20$

  \pause

  \vspace{1mm}
  \alert{Do they solve T\'{o}th's and Thomson's problem?}
  \vspace{-2mm}
  
  \begin{minipage}{50mm}
    \begin{itemize}
    \item \textbf{Yes}: for $n = 4, 6, 12$
      
    \item \textbf{No}: for $n = 8, 20$
    \end{itemize}
  \end{minipage}
  \begin{minipage}{15mm}
    \begin{center}
      Example: n=8
    \end{center}
  \end{minipage}
  \begin{minipage}{35mm}
    \begin{center}
      \begin{animateinline}[autoplay, loop, poster = first]{5}
        \multiframe{45}{ranti=0+2}{
          \psset{viewpoint=0 -100 30,Decran=150,lightsrc=-50 -100 80, lightintensity=1.5}
          \begin{pspicture}(3.0,2.5)
            \psSolid[object=new,RotZ=\ranti,fillcolor=orange,
            sommets=
            0.3289   0.7941  0.5110
            -0.3289 -0.7941  0.5110
            0.7941  -0.3289  0.5110
            -0.7941  0.3289  0.5110
            0.3289  -0.7941 -0.5110
            -0.3289  0.7941 -0.5110
            0.7941   0.3289 -0.5110
            -0.7941 -0.3289 -0.5110,
            faces={
              [4 6 5 7]
              [2 6 0]
              [2 4 6]
              [1 4 2]
              [1 7 4]
              [3 7 1]
              [3 5 7]
              [0 5 3]
              [0 6 5]
              [0 3 1 2]},
            action=draw*](1.08,0,0.95)
          \end{pspicture}
        }
      \end{animateinline}
    \end{center}
  \end{minipage}
}


\section{Results}

\frame{
  \frametitle{Results}
  
  \begin{itemize}
  \item Relationship between coefficients of $| \psi \rangle$ and
    point distributions, \newline
    e.g. points are restricted to certain areas if coefficients are
    all positive.
    
  \item Bound on maximal symmetric $n$ qubit entanglement:
    $E_G^{\text{max}} = \mathcal{O}(\log n)$ (general non-symmetric
    case: $E_G^{\text{max}} = \mathcal{O}(n)$ )

  \item Upper bound on number of closest product states: 2n-4 \newline 
    $\:$ \HandRight $\:$ this is intriguing, because for polyhedra
    with $n$ vertices Euler's formula gives $2n-4$ as upper bound to
    number of faces. Signature of a deeper connection?
    
  \item Found highly entangled symmetric states from the known
    solutions of T\'{o}th's and Thomson's problem.
  \end{itemize}    
  \vspace{-3mm}

  \begin{minipage}{50mm}
    Example: ``Icosahedron state'' of 12 qubits, a strong candidate
    for maximal entanglement.
  \end{minipage}
  \begin{minipage}{57mm}
    \begin{center}
      \begin{figure}[ht]
        \begin{overpic}[scale=.33]{images/bloch_12b}
        \end{overpic}
        \hspace{1mm}
        \begin{overpic}[scale=0.09]{images/icosahedronplot}
        \end{overpic}
      \end{figure}
    \end{center}
  \end{minipage}
}

\frame{
  \begin{minipage}{53mm}
    Analytical \& numerical results \\
    $\Longrightarrow$ find maximally entangled \\
    symmetric state of up to 12 qubits
    \vspace{3mm}

    \textbf{\underbar{3 qubits:}}
    
    \begin{itemize}
    \item {\small $| W \rangle = |001\rangle \! + \!  |010\rangle \! +
        \! |100\rangle$} maximally entangled

    \item T\'{o}th's and Thomson's problem: equilateral triangle
      (GHZ-state)
    \end{itemize}
    
    \begin{figure}[ht]
      \begin{overpic}[scale=.3]{images/bloch_w}
        \put(27,84){{\scriptsize $| \phi_1 \rangle$}}
        \put(54,84){{\scriptsize $| \phi_2 \rangle$}}
        \put(54,10){{\scriptsize $| \phi_3 \rangle$}}
        \put(-11,71){{\scriptsize $| \sigma \rangle$}}
        \put(42,53){{\scriptsize $\theta$}}
      \end{overpic}
      \hspace{4mm}
      \begin{overpic}[scale=.3]{images/bloch_ghz}
        \put(-19,53){{\scriptsize $| \phi_1 \rangle$}}
        \put(72,30){{\scriptsize $| \phi_2 \rangle$}}
        \put(65,63){{\scriptsize $| \phi_3 \rangle$}}
        \put(28,83){{\scriptsize $| \sigma_1 \rangle$}}
        \put(28,9){{\scriptsize $| \sigma_2 \rangle$}}
      \end{overpic}
    \end{figure}
    
  \end{minipage}
  \begin{minipage}{53mm}
    \textbf{\underbar{4 qubits:}}
    
    \begin{itemize}
    \item Tetrahedron solves T\'{o}th, Thomson and Majorana
    \end{itemize}
    \vspace{-2mm}
    
    \begin{figure}[ht]
      \begin{overpic}[scale=.3]{images/bloch_4}
        \put(27,85){{\footnotesize $| \phi_1 \rangle$}}
        \put(-17,21){{\footnotesize $| \phi_2 \rangle$}}
        \put(64,15){{\footnotesize $| \phi_3 \rangle$}}
        \put(68,47){{\footnotesize $| \phi_4 \rangle$}}
      \end{overpic}
    \end{figure}
    \vspace{-1mm}

    \textbf{\underbar{5 qubits:}}

    \begin{itemize}
    \item again different solutions
    \end{itemize}
    \vspace{-2mm}

    \begin{figure}[ht]
      \begin{overpic}[scale=.3]{images/bloch_5a}
        \put(27,83){{\scriptsize $| \phi_1 \rangle$}}
        \put(7,34){{\scriptsize $| \phi_2 \rangle$}}
        \put(75,31){{\scriptsize $| \phi_3 \rangle$}}
        \put(74,56){{\scriptsize $| \phi_4 \rangle$}}
        \put(27,11){{\scriptsize $| \phi_5 \rangle$}}
      \end{overpic}
      \hspace{4mm}
      \begin{overpic}[scale=.3]{images/bloch_5b}
        \put(27,85){{\scriptsize $| \phi_1 \rangle$}}
        \put(20,15){{\scriptsize $| \phi_2 \rangle$}}
        \put(63,20){{\scriptsize $| \phi_3 \rangle$}}
        \put(74,50){{\scriptsize $| \phi_4 \rangle$}}
        \put(4,49){{\scriptsize $| \phi_5 \rangle$}}
      \end{overpic}
    \end{figure}

  \end{minipage}
}

\frame{
  \hspace{-18mm}
  \begin{minipage}{74mm}
    \begin{figure}[ht]
      \begin{minipage}{49mm}
        \begin{center}
          \begin{overpic}[scale=0.77, unit=1mm]{images/entanglement_graph}
          \end{overpic}
        \end{center}
      \end{minipage}
    \end{figure}
  \end{minipage}
  \begin{minipage}{48mm}
    \textbf{Figure:} Scaling of maximal symmetric entanglement with
    number of qubits $n$.
    \vspace{2mm}
    
    \begin{itemize}
      {\footnotesize
      \item  black: upper and lower bounds
        
      \item  blue\&red: numerical solutions \newline
        (for positive and general case)
        
      \item green: classical solutions
      }
    \end{itemize}
    
    \vspace{2mm}
    Findings:
    \vspace{1mm}
    
    \begin{itemize}
      {\footnotesize
      \item classical solutions are in general not optimal
        
      \item But: solutions of Thomson's problem are a good bound
        
      \item maximally entangled symmetric state has positive
        coefficients only for low $n$ }
    \end{itemize}
  \end{minipage}
}

\section{Conclusion}

\frame{
  \begin{block}{Conclusion}
    We found highly or maximally entangled states of symmetric $n$
    qubit systems, and visualised them as spherical point
    distributions.  This ``Majorana Representation'' is extremely
    helpful both for theoretical and practical reasons.
  \end{block}
  \vspace{4mm}
  
  \begin{center}
    {\Large \textbf{Thanks for your attention!}}
    \vspace{4mm}
    
    For details see: \textbf{arXiv:1003.5643}
    \vspace{-3mm}

    (to be published in New J. Phys.)
  \end{center}
}

\end{document}
