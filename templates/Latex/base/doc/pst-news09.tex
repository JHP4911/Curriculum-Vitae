%% $Id: pst-news09.tex 856 2013-12-09 10:34:40Z herbert $
\documentclass[11pt,english,BCOR10mm,DIV12,bibliography=totoc,parskip=false,smallheadings
    headexclude,footexclude,oneside]{pst-doc}
\listfiles
\usepackage[utf8]{inputenc}
\usepackage{pst-plot}
\let\pstplotFV\fileversion
\let\pstplotFD\filedate
\usepackage{pst-node}
\let\pstnodeFV\fileversion
\let\pstnodeFD\filedate
\usepackage{pst-tree}
\let\psttreeFV\fileversion
\let\psttreeFD\filedate
\usepackage{pst-eucl}
\renewcommand\bgImage{\psscalebox{15}{\color{blue!20}2009}}
\begin{document}

%\psset{PstDebug=1}
\title{\texttt{News -- 2009}\\ \Large new macros and bugfixes for the
basic packages \nxLFile{pstricks}, \nxLFile{pst-plot}, and \nxLFile{pst-node}}
\author{Herbert Voß}
\date{\today}

\maketitle

\clearpage
\tableofcontents

\clearpage
\part{\texttt{pstricks} -- package}

\section{General}
There exists a new document class \LClass{pst-doc} for writing PSTricks documentations,
like this news document. It depends on the KOMA-Script document class \LClass{scrartcl}.
\LClass{pst-doc} defines a lot of special macros to create a good index. Take one of
the already existing package documentation and look into the source file. Then it will be
easy to understand, how all these macros have to be used.



%--------------------------------------------------------------------------------------
\section{\texttt{pstricks.sty}}
%--------------------------------------------------------------------------------------

\subsection{New optional argument}
With the new optional argument \Lkeyval{97} \LPack{pstricks} has the old behaviour of the
version of 1997. The \Lenv{pspicture} environment then has only a value as
optional argument, which shifts the box in vertical direction. The new options
\Lkeyword{shift} and \Lkeyword{showgrid} do not work in this case.
The support of transparency colors is also disabled.


%--------------------------------------------------------------------------------------
\section{\texttt{pstricks.tex} (\pstricksFV -- \pstricksFD)}
%--------------------------------------------------------------------------------------
\subsection{Makro \nxLcs{next}}
To prevent problems with other packages which also define a \Lcs{next}
macro, it is now renamed to \Lcs{ps@next}. This should be not important
for the user, because it is used only by the base system itself.

\subsection{Makro \nxLcs{psDEBUG}}
\LPack{pstricks} defines the option \texttt{PstDebug=0|1}, which can be used for 
debugging. The new macro \Lcs{psDEBUG} makes it easier to write some \Index{debugging}
information into the package files. The macro is only valid, if \Lkeyword{PstDebug}=1
is set, otherwise the macro does nothing.

\begin{BDef}
\Lcs{psDEBUG}\OptArgs\Largb{text}
\end{BDef}

\Lcs{psDEBUG} writes the argument \Larg{text} into the log file.
Without an optional argument the word \Larg{pstricks} is used. The following output 
of the log file 

\begin{lstlisting}
...
<key:xticksize>: setting ticksize to max
LaTeX Font Info:    External font `cmex10' loaded for size
(Font)              <7> on input line 26.
LaTeX Font Info:    External font `cmex10' loaded for size
(Font)              <5> on input line 26.
<pst@@hlabels>: xticksizeC=0.0pt
...
\end{lstlisting}

was possible with \Lcs{psset}\Largb{\Lkeyword{PstDebug}}=1\footnote{Can also be used locally for a 
macro when used as optional argument
in the usual way.} 
and inside of \LPack{pstricks-add} with 
(only the first for example):
\begin{lstlisting}
...
  \psDEBUG[key:ticksize]{setting ticksize}
...
\end{lstlisting}

\subsection{Special Coordinates}
The macros \Lcs{SpecialCoor} and the corresponding \Lcs{NormalCoor} can now be used for
the \Index{radius} of \Index{arc}s and \Index{circle}s. With a preceeding \Lnotation{!} character the length can
be of any PostScript expression and will be taken with the unit pt in user coordinates: 

\begin{LTXexample}[width=6cm]
\begin{pspicture}[showgrid=true](-3,-3)(3,3)
\pscircle[linecolor=red](0,0){1cm}
\SpecialCoor
\pscircle[linecolor=blue](0,0){!3 sqrt}
\NormalCoor
\psarc[linewidth=2pt](0,0){2cm}{90}{270}
\SpecialCoor
\psarc[linecolor=green](0,0){! 60 sin dup add }{90}{270}
\end{pspicture}
\end{LTXexample}

\SpecialCoor

\subsection{\nxLcs{psarcAB} and \nxLcs{psarcnAB}}

\begin{BDef}
\Lcs{psarcAB}\OptArgs\OptArg*{\Largb{arrows}}\coord{C}\coord{A}\coord{B}\\
\Lcs{psarcnAB}\OptArgs\OptArg*{\Largb{arrows}}\coord{C}\coord{A}\coord{B}
\end{BDef}

$(x_C, y_C)$ is the center  and $(x_A, y_A)$ $(x_B, y_B)$ are two given points of the
\Index{circle}. The angles for the \Index{arc} are calculated by \TeX. The arc is drawn from the first
to the second point. The radius is the distance from the center to the second point.
The first point can have another distance, it is only used for calculating the angle.
The following examples need the package \LPack{pst-eucl}.

\begin{LTXexample}[width=6cm]
\begin{pspicture}(-3,-3)(3,3)
\pstTriangle[PosAngle={180,0,90}](-3,0){A}(3,0){B}(0,3){C}
\pstCircleOA{I}{A} 
\pstMiddleAB{A}{B}{Iab}
\pscustom[linestyle=none,fillstyle=solid,fillcolor=blue!50]{%
  \psarcAB(C)(A)(B)
  \psarcnAB(Iab)(B)(A)}
\pstMiddleAB{A}{C}{Iac}
\pscustom[linestyle=none,fillstyle=solid,fillcolor=red!50]{%
  \psarcnAB(I)(A)(C)
  \psarcAB(Iac)(C)(A)}
\pstMiddleAB{B}{C}{Ibc}
\pscustom[linestyle=none,fillstyle=solid,fillcolor=green!50]{%
  \psarcAB(I)(B)(C)
  \psarcnAB(Ibc)(C)(B)}
\end{pspicture}
\end{LTXexample}


\subsection{Makro \nxLcs{psPline}}
\begin{BDef}
\Lcs{psPline}\OptArgs\coord1\coord2\coord3
\end{BDef}

The first point P1 is a projection onto the line of $\overline{P_2P_3}$
and connected with drawing a line similiar to \Lcs{psline}. This macro
has three mandatory arguments for the three points.

\begin{LTXexample}[pos=t]
\psset{unit=3}
\begin{pspicture}[showgrid=true](3,2)
\pspolygon(0,1)(3,2)(2.5,0)
\psPline[linestyle=dashed,arrowscale=2]{->}(0,1)(3,2)(2.5,0)
\psPline[linestyle=dashed,arrowscale=2]{->}(3,2)(2.5,0)(0,1)
\psPline[linestyle=dashed,arrowscale=2]{->}(2.5,0)(0,1)(3,2)
\end{pspicture}
\end{LTXexample}



%--------------------------------------------------------------------------------------
\section{The PostScript header files}
\subsection{\nxLFile{pstricks.pro}}
Only cosemetic changes to the code to make it more readable.
%--------------------------------------------------------------------------------------
\subsection{\nxLFile{pst-node.pro}}
Only cosemetic changes to the code to make it more readable.

\part{\nxLPack{pst-node} -- package}

\section{\nxLFile{pst-node.tex} (\pstnodeFV -- \pstnodeFD)}

There was a bug in \Lcs{fnode}. It is now fixed.

foo

     z  \quad
        \pnode{testL1}
        \fnode[linecolor=lightgray,framesize=6pt 50pt](0,.5ex){eqL}
        \pnode(0,2ex){testL2}
        \quad
     =
        \quad%
        \pnode{testR1}
        \fnode[linecolor=lightgray,framesize=6pt 50pt](0,.5ex){eqR}
        \pnode(0,2ex){testR2}
        \quad
     z
  \ncline[linecolor=red]{testL1}{testR1}
  \ncline[linecolor=blue]{testL2}{testR2}

bar

\bigskip now with an aligned node
\psset{nodealign=true}

foo

     z  \quad
        \pnode{testL1}
        \fnode[linecolor=lightgray,framesize=6pt 50pt](0,.5ex){eqL}
        \pnode(0,2ex){testL2}
        \quad
     =
        \quad%
        \pnode{testR1}
        \fnode[linecolor=lightgray,framesize=6pt 50pt](0,.5ex){eqR}
        \pnode(0,2ex){testR2}
        \quad
     z
  \ncline[linecolor=red]{testL1}{testR1}
  \ncline[linecolor=blue]{testL2}{testR2}

bar


\begin{lstlisting}
foo

     z  \quad
        \pnode{testL1}
        \fnode[linecolor=lightgray,framesize=6pt 50pt](0,.5ex){eqL}
        \pnode(0,2ex){testL2}
        \quad
     =
        \quad%
        \pnode{testR1}
        \fnode[linecolor=lightgray,framesize=6pt 50pt](0,.5ex){eqR}
        \pnode(0,2ex){testR2}
        \quad
     z
  \ncline[linecolor=red]{testL1}{testR1}
  \ncline[linecolor=blue]{testL2}{testR2}

bar

\bigskip now with an aligned node
\psset{nodealign=true}

foo

     z  \quad
        \pnode{testL1}
        \fnode[linecolor=lightgray,framesize=6pt 50pt](0,.5ex){eqL}
        \pnode(0,2ex){testL2}
        \quad
     =
        \quad%
        \pnode{testR1}
        \fnode[linecolor=lightgray,framesize=6pt 50pt](0,.5ex){eqR}
        \pnode(0,2ex){testR2}
        \quad
     z
  \ncline[linecolor=red]{testL1}{testR1}
  \ncline[linecolor=blue]{testL2}{testR2}

bar
\end{lstlisting}


\part{\nxLPack{pst-plot} -- package}

\section{\nxLFile{pst-plot.tex} (\pstplotFV -- \pstplotFD)}

To be more consistent to other macro names there are now the names
\Lcs{psparametricplot}, \Lcs{psfileplot}, \Lcs{pslistplot}, and
\Lcs{psdataplot}. The old macros without the preceeding \verb+ps+
still work!

There is now a new \PS boolean variable which can control loops.
With setting \texttt{/\Lps{ps@Exit} true def} you can leave a loop
for plotting a function with \Lcs{psplot} and \Lcs{psparametricplot}.


\part{\nxLPack{pst-tree} -- package}
\section{\nxLFile{pst-tree.tex} (\psttreeFV -- \psttreeFD)}

The package \LPack{pst-tree} now uses the advanced key handling from \LPack{xkeyval}. The reason
why it moved from the base into the contrib sections, where all packages uses \LPack{xkeyval}.

\nocite{*}
\bibliographystyle{plain}
\bibliography{PSTricks}

\printindex


\end{document}
