% !TeX program = LuaLaTeX
\documentclass[11pt,draft]{article}

% Font set up
\usepackage{fontspec}
\setmainfont[Ligatures = TeX,Numbers = OldStyle]{TeX Gyre Pagella}
\setsansfont[Ligatures = TeX]{TeX Gyre Adventor}
\setmonofont[Ligatures = TeX]{Inconsolata}

\usepackage[english,UKenglish]{babel}
\usepackage[a4paper,nohead,nofoot,hmargin=1.5cm,vmargin=2cm]{geometry}
\usepackage[version=3]{mhchem}
\usepackage[final]{microtype}

\usepackage{amstext,array,comment,csquotes,etaremune,fixltx2e,pifont,ragged2e}
\usepackage{textcomp,titlesec,xcolor}

\usepackage[final]{hyperref}

\hypersetup
  {
    hidelinks = true             ,
    pdfauthor = Joseph Wright    ,
    pdftitle  = Curriculum Vitae
  }

% Create some nice looking section dividers without too much fuss
\titleformat\section{\Large\bfseries\sffamily}{}{0 em}
  {%
    % This is put in _before_ the text so that an overlap is possible
    \begingroup
      \color{gray!30}%
      \titleline{\leaders\hrule height 0.6 em\hfill\kern 0 pt\relax}%
    \endgroup
    \nobreak
    \vspace{-1.2 em}%
    \nobreak
  }
  
\titleformat\subsection{\large\itshape}{}{0 em}{}

\renewcommand*\arraystretch{1.4}
\setlength\hfuzz{1 pt}
\pagestyle{empty}
\frenchspacing

% Publication list set up
\renewcommand*\labelenumi{\textbf{\textsf{(\theenumi)}}}
\newcommand*\paper[2]{%
  % Standard style
  \item \href{http://dx.doi.org/#1}{\ignorespaces#2\unskip.}
  % Including DOI
  %\item \href{http://dx.doi.org/#1}
  %  {\ignorespaces#2\unskip.\\\textsc{doi}: \texttt{#1}}
}
\newcommand*\papertitle[1]{%
  \begingroup
    \addfontfeature{Numbers = Lining}%
    \emph{#1}%
  \endgroup
}
\hyphenation{Bosch-ert}

% I prefer 'running' appearance for these
\newcommand*\AllTeX{(La)TeX}
\renewcommand*\LaTeX{LaTeX}
\renewcommand*\TeX{TeX}
  
% Semi-automatically set up the table width
\newlength\sidewidth
\newlength\mainwidth
\AtBeginDocument{%
  \settowidth\sidewidth{\textbf{Professional bodies}\hspace{0.75 em}}%
  \setlength\mainwidth{\dimexpr\linewidth - \sidewidth\relax}%
}

\newcommand*\headline[1]{%
  \hbox{%
    \llap{\ding{72}\hspace*{0.2 em}}%
    \textbf{#1}%
  }%
}
\newenvironment{CVtable}
  {%
    \begin{tabular}
      {@{}>{\bfseries}p{\sidewidth}@{}>{\RaggedRight}p{\mainwidth}@{}}%
  }
  {\end{tabular}}

\begin{document}

% Title block
\begin{raggedleft}
  \textbf{Joseph Wright}    \\
  School of Chemistry       \\
  University of East Anglia \\
  Norwich NR4 7TJ           \\
  Tel.: 01603 592902        \\
  Mobile: 0797 414 8180     \\
  \href{mailto:joseph.wright@uea.ac.uk}
    {\texttt{joseph.wright@uea.ac.uk}} \\
\end{raggedleft}

\begin{center}
  \Huge\bfseries\sffamily
  Joseph Alexander Wright
\end{center}

\section{Employment history}

\subsection{Current position}
\nocite{*}

\begin{CVtable}
  2008-- &
    \headline{PDRA -- University of East Anglia} \par
    Supervisor Prof.~C.~J.~Pickett \par
    Studies on [Fe]- and [FeFe]-hydrogenase active sites mimics \par
    Synthesis of novel ligands and model compounds \par
    Mechanistic studies using stopped-flow UV and IR spectroscopies
  \\
\end{CVtable}

\subsection{Previous positions}

\begin{CVtable}
  2007--2008 &
    \headline{Senior Demonstrator -- University of East Anglia} \par
    Teaching degree level chemistry:
    tutorials and laboratory classes \par
    Preparation of M.~Chem.~third year practical course in
    organic chemistry
  \\
      
  2005--2008 &
    \headline{PDRA -- University of East Anglia} \par 
    Supervisor Prof.~M.~Bochmann \par 
    Use of zirconium phosphonates as heterogeneous catalyst supports \par
    Synthesis of novel ligand systems for early transition metals
  \\

  2003--2004 & 
    \headline{PDRA -- University of Southampton} \par
    Supervisor Dr A.\,A.~Danopoulos \par
    Synthesis of novel N-heterocyclic carbene complexes \par
    Catalytic testing on novel systems
  \\
\end{CVtable}

\section{Academic history}

\begin{CVtable}
  1999--2002 &
    \headline{Ph.D. -- University of Cambridge} \par
    Supervisor Dr J.\,B.~Spencer \par
    Studies on the mechanisms of late transition-metal-catalysed 
    reactions \par
    Development of a novel protective-group strategy for alcohols \par
    Thesis title:
      \emph{Control of Regioselectivity: Oxidation and Deprotection}
    % \par
    % Viva date: 31st March 2003 % When needed 
  \\

  1995--1999 &
    \headline{M.Chem. University of Leicester} \par
    First class degree -- Four year study 
    % \par
    % Award date: 23rd July 1999 % When needed
  \\
   
  1995 &
    \headline{A levels} \par
    \vspace{0.25 em}
    \begingroup
      \renewcommand*\arraystretch{1}
      \begin{tabular}
        {@{}>{\textbullet~}l<{:}l>{\textbullet~}l<{:}l}
        Chemistry & A grade & Physics       & A grade \\
        Maths     & A grade & Further maths & A grade \\
      \end{tabular}
    \endgroup   
\end{CVtable}

\section{Teaching experience}

\begin{CVtable}
  Lectures &
    Two-part course on X-ray crystallography \par
    Introductory talk \enquote{Skills for new researchers}
  \\
  
  Tutorials &
    Significant experience of giving both organic and inorganic
      tutorials \par
    Advanced tutorials on crystal symmetry  
  \\

  Supervision &
    Directly supervised several final year project students \par
    Planning project aims and direction for Ph.D. students \par
    Leadership role in laboratory environment 
  \\
  
  Assessment &
    Oral and written assessment of M.Chem. practical classes \par
    Final-year project student presentation marking \par
    Development of a series of model answers for practical course 
  \\
    
  Skills transfer &
    Training students in the use of X-ray diffraction \par
    Lead a successful series of \enquote{\LaTeX{} for Beginners} courses
  \\
    
\end{CVtable}

\section{Skills}

\begin{CVtable}
  NMR & 
    Hands on experience with Bruker systems \par
    Use of a range of multi-nuclear and variable-temperature methods
  \\
    
  FT-IR & 
    Analysis of time-resolved spectra from ReactIR and Bruker Vertex 
      systems
  \\
    
  X-Ray diffraction &
    Data collection with both Bruker--Nonius and Oxford Diffraction 
      instruments \par
    Structure solution using \textsc{shelx-97} \par
    Data analysis using \textsc{mercury} and \textsc{platon}
  \\  
  
  Computing & 
    Advanced document preparation in both Word and \LaTeX \par
    Web site management, including on-line conference registration \par
    Programming in \AllTeX{}, (X)HTML and PHP \par
    Windows and Unix administration
  \\
\end{CVtable}

\section{Additional information}

\begin{CVtable}
  Professional bodies &
    American Chemical Society \par
    British Crystallography Association  \par
    Royal Society of Chemistry
  \\
  
  Membership &
    Co-ordination Chemistry Discussion Group 2011 committee \par
    \LaTeX3 Project \par
    \TeX{} Users Group \par
    UK \TeX{} Users' Group (Secretary) 
  \\
    
%  Languages & Certificate in German (2005) \par
%    French to GCSE 
%  \\
%    
%  Ongoing & 
%    Open University Mathematics degree (2005--) 
%  \\ 
\end{CVtable}

\newpage

\section{List of Publications}

% Keep this file relatively short
% !TeX program = LuaLaTex
% !TeX root    = CV.tex

\hyphenation{Bos-ch-ert}
\hyphenation{Hugh-es}
\hyphenation{Wri-ght}

\begin{etaremune}[start = 45] % Remember to adjust this
  \paper{10.1021/ja2087536}{
    \papertitle{Paramagnetic Bridging Hydrides of Relevance to Catalytic
    Hydrogen Evolution at Metallo-sulfur Centers}, A.~Jablonskytė,
    J.\,A.~Wright, S.\,A.~Fairhurst, J.\,N.\,T.~Peck, S.\,K.~Ibrahim,
    V.\,S.~Oganesyan, and C.\,J.~Pickett, \emph{J.~Am. Chem. Soc.},
    in press
  }

  \paper{10.1039/c1cc11320h}{
    \papertitle{The role of CN and CO ligands in the vibrational relaxation
    dynamics of model compounds of the [FeFe]-\break hydrogenase enzyme},
    S.~Kaziannis, J.\,A.~Wright, M.~Candelaresi, R.~Kania, G.\,M.~Greetham,
    A.\,W.~Parker, C.\,J.~Pickett and N.\,T.~Hunt 
    \emph{Phys. Chem. Chem. Phys.}, 2011, \textbf{13}, 10295--10305
  }

  \paper{10.1039/c1cc11320h}{
    \papertitle{The hafnium-mediated NH activation of an amido-borane},
    E.\,A.~Jacobs, A.-M.~Fuller, S.\,J.~Lancaster and J.\,A.~Wright,
    \emph{Chem. Commun.}, 2011, \textbf{47}, 5870--5872
  }

  \paper{10.1002/ejic.201001085}{
    \papertitle{Density Functional Calculations on Protonation of the
    [FeFe]-Hydrogenase Model Complex \ce{Fe2($\mu\text{-}$pdt)(CO)4}%
    -\break\ce{(PMe3)2} and Subsequent Isomerization Pathways}, C.~Liu,
    J.\,N.~T.~Peck, J.\,A.~Wright, C.\,J.~Pickett and\break M.\,B.~Hall, 
    \emph{Eur.~J. Inorg. Chem.}, 2011, 1080--1093
  }
  
  \paper{10.1002/ejic.201001072}{
    \papertitle{[FeFe]-Hydrogenase models: unexpected variation in
    protonation rate between dithiolate bridge analogues},
    \break
    A.~Jablonskytė, J.\,A.~Wright and C.\,J.~Pickett, \emph{Eur.~J.
    Inorg. Chem.}, 2011, 1033--1037
  }

  \paper{10.1021/jp107618n}{
    \papertitle{Femtosecond to Microsecond Photochemistry of a 
    [FeFe]hydrogenase Enzyme Model Compound}, S.~Kaziannis, 
    S.~Santabarbara, J.\,A.~Wright, G.\,M.~Greetham, M.~Towrie, 
    A.\,W.~Parker, C.\,J.~Pickett and N.\,T.~Hunt, \emph{J.~Phys.
    Chem.~B}, 2010, \textbf{114}, 15370--15379
  }

  \paper{10.1107/S0108270110049371}{
    \papertitle{The mixed diol-dithiol 
    2,2-bis(sulfanylmethyl)propane-1,3-diol: characterization of key
    intermediates on a new synthetic pathway}, T.\,R.~Simmons, 
    C.\,J.~Pickett and J.\,A.~Wright, \emph{Acta Cryst.~C}, 2011,
    \textbf{67}, o1--o5
  }
  
  \paper{10.1039/c004692b}{
    \papertitle{Protonation of [FeFe]-hydrogenase sub-site analogues:
    revealing mechanism using FTIR stopped-flow techniques},
    J.\,A.~Wright, L.~Webster, A.~Jablonskytė, P.\,M.~Woi,
    S.~Ibrahim and C.\,J.~Pickett, \emph{Faraday Discuss.}, 2011,
    \textbf{148}, 359--371
  }
  
  \paper{10.1021/om1008567}{
    \papertitle{The Third Hydrogenase: More Natural Organometallics},
    J.\,A.~Wright, P.\,J.~Turrell and C.\,J.~Pickett,
    \emph{Organometallics}, 2010, \textbf{49}, 6146--6156
  }
  
  \paper{10.1002/anie.201004189}{
    \papertitle{The Third Hydrogenase: A Ferracyclic Carbamoyl with Close
    Structural Analogy to the Active Site of Hmd}, P.\,J.~Turrell,
    J.\,A.~Wright, J.\,N.~T.~Peck, V.~Oganesyan and C.\,J.~Pickett,
    \emph{Angew. Chem. Int. Ed.}, 2010, \textbf{49}, 7508--7511
  }
  
  \paper{10.1021/ic101289s}{
    \papertitle{Determination of the Photolysis Products of
    [FeFe]Hydrogenase Enzyme Model Systems using Ultrafast
    Multidimensional Infrared Spectroscopy}, A.\,I.~Stewart,
    J.\,A.~Wright, G.\,M.~Greetham, S.~Kaziannis, S.~Santabarbara,
    M.~Towrie, A.\,W.~Parker, C.\,J.~Pickett and N.\,T.~Hunt,
    \emph{Inorg. Chem.}, 2010, \textbf{49}, 9563--9573
  }
  
  \paper{10.1107/S0108270110018433}{
    \papertitle{1-[2-(2,6-Diisopropylanilino)-1-naphthyl]isoquinoline},
    R.\,H.~Howard, N.~Theobald, M.~Bochmann, \break J.\,A.~Wright,
    \papertitle{Acta Cryst.~C}, 2010, \textbf{66}, o310--o312
  }
  
  \paper{10.1039/b923191a}{%
    \papertitle{Mechanistic aspects of the protonation of
    [FeFe]-hydrogenase subsite analogues}, A.~Jablonskytė,
    J.\,A.~Wright and C.\,J.~Pickett, \emph{Dalton Trans.}, 2010,
    \textbf{39}, 3026--3034%
  }
  
  \paper{10.1107/S0108270110004506}{
    \papertitle{2-(Diphenylphosphinoylmethyl)pyrrole-2-%
    (diphenylphosphinomethyl)pyrrole (0.43/0.57) and
    tetrachlorido-\break (5-diphenylphosphinomethyl-2H-pyrrole-%
    κ\textsuperscript{2}N,P)titanium(IV)}, L.\,M.~Broomfield,
    M.~Bochmann and \break J.\,A.~Wright, \emph{Acta Cryst.~C}, \textbf{2010},
    m79--m82
  }
  
  \paper{10.1016/j.jorganchem.2009.08.033}{%
    \papertitle{Synthesis of neutral and zwitterionic
    phosphinomethylpyrrolato complexes of nickel}, L.\,M.~Broomfield,
    D.~Boschert, J.\,A.~Wright, D.\,L.~Hughes and M.~Bochmann,
    \emph{J.~Organomet. Chem.}, 2009, \textbf{694}, 4084--4089%
  }
  
  \paper{10.1039/b907982c}{
    \papertitle{Synthesis and structures of complexes with axially chiral
    isoquinolinyl-naphtholate ligands}, R.\,H.~Howard, C.~Alonso-Moreno,
    L.\,M.~Broomfield, D.\,L.~Hughes, J.\,A.~Wright, M.~Bochmann,
    \emph{Dalton Trans.}, 2009, \textbf{38}, 8667--8682
  }
  
  \paper{10.1039/b901810g}{
    \papertitle{Synthesis, structure and ethylene polymerisation 
    behaviour of vanadium(IV and V) complexes bearing chelating
    aryloxides}, D.~Homden, C.~Redshaw, L.~Warford, D.\,L.~Hughes,
    J.\,A.~Wright, S.\,H.~Dale and M.\,R.~J.~Elsegood, \emph{Dalton
    Trans.}, 2009, \textbf{38}, 890--8910
  }
  
  \paper{10.1039/b902402f}{
    \papertitle{Vanadium-based imido-alkoxide pro-catalysts bearing
    bisphenolate ligands for ethylene and epsilon-caprolactone
    polymerisation}, A.~Arbaoui, C.~Redshaw, D.~Homden, J.\,A.~Wright
    and M.\,R.~J.~Elsegood, \emph{Dalton Trans.}, 2009, \textbf{38},
    8911--8922
  }
  
  \paper{10.1039/b908394d}{
    \papertitle{Synthesis, structures and reactivity of
    2-phosphorylmethyl-1H-pyrrolato complexes of titanium, yttrium and
    zinc}, L.\,M.~Broomfield, J.\,A.~Wright and M.~Bochmann, \emph{Dalton
    Trans.}, 2009, \textbf{38}, 8269--8279
  }
  
  \paper{10.1039/b912499c}{
    \papertitle{Protonation of a subsite analogue of [FeFe]-hydrogenase:
    mechanism of a deceptively simple reaction revealed by
    time-resolved IR spectroscopy}, J.\,A.~Wright and C.\,J.~Pickett,
    \emph{Chem. Commun.}, \textbf{45}, 5719--5721
  }
  
  \paper{10.1039/b813313a}{
    \papertitle{New structural motifs in chromium(III)
    calix[4 and 6]arene chemistry}, C.~Redshaw, D.~Homden, D.\,L.~Hughes,
    J.\,A.~Wright and M.\,R.~J.~Elsegood, \emph{Dalton Trans.},
    \textbf{38}, 1231--1242
  }
  
  \paper{10.1039/b821301a}{
    \papertitle{α-Zirconium phosphonates: versatile supports for
    N-heterocyclic carbenes}, S.~Chessa, N.\,J.~Clayden, M.~Bochmann 
    and J.\,A.~Wright, \emph{Chem. Commun.}, 2009, \textbf{45}, 797--799
  }
  
  \paper{10.1002/anie.200804573}{
    \papertitle{``Pincer'' Pyridine-Dicarbene-Iridium Complexes: Facile
    \ce{C-H} Activation and Unexpected
    η\textsuperscript{2}-Imidazol-2-yl-\break idene Coordination},
    J.\,A.~Wright,
    A.\,A.~Danopoulos, W.\,B.~Motherwell, R.\,J.~Carroll, S.~Ellwood,
    J.~Saßmannshausen, \emph{Angew. Chem. Int. Ed.}, 2008, 
    \textbf{47}, 9765--9767
  }
  
  \paper{10.1021/om800783h}{
    \papertitle{Structural Characterization of a Cationic Zirconocene
    Dimethylaniline Complex and Related Catalytically Relevant Species},
    P.\,A.~Wilson, J.\,A.~Wright, V.\,S.~Oganesyan, S.\,J.~Lancaster
    and M.~Bochman, \emph{Organometallics}, 2008, \textbf{27},
    6371--6374
  }
  
  \paper{10.1021/om800486p}{
    \papertitle{Ligand Mobility and Solution Structures of the
    Metallocenium Ion Pairs [\ce{Me2C}(Cp)(fluorenyl)\ce{MCH2SiMe3+}%
    \(\cdots\)\break \ce{X-}] (M = Zr, Hf; X = \ce{MeB(C6F5)3}, 
    \ce{B(C6F5)4})}, C.~Alonso-Moreno, S.\,J.~Lancaster, J.\,A.~Wright, 
    D.\,L. Hughes, C.~Zuccaccia, A.~Correa, A.~Macchioni, L.~Cavallo
    and M.~Bochmann, \emph{Organometallics}, 2008, \textbf{27}, 
    5474--5487
  }
  
  \paper{10.1021/ic702506w}{
    \papertitle{Early Transition Metal Complexes Bearing a C-Capped
    Tris(phenolate) Ligand Incorporating a Pendant Imine Arm: Synthesis,
    Structure, and Ethylene Polymerization Behavior}, D.~Homden,
    C.~Redshaw, J.\,A.~Wright, D.\,L.~Hughes and M.\,R.~J.~Elsegood,
    \emph{Inorg. Chem.}, 2008, \textbf{47}, 5799--5814
  }
  
  \paper{10.1016/j.jorganchem.2007.10.043}{
    \papertitle{Synthesis and structures of new binuclear zinc alkyl,
    aryl and aryloxo complexes}, Y.~Sarazin, J.\,A.~Wright, D.\,A.~J.
    Harding, E.~Martin, T.\,J.~Woodman, D.\,L.~Hughes and M.~Bochmann,
    \emph{J.~Organomet. Chem.}, 2008, \textbf{693}, 1494--1501
  }
  
  \paper{10.1016/j.jorganchem.2007.04.047}{%
    \papertitle{Mixed-ligand iminopyrrolato-salicylaldiminato group~4 
    metal complexes: Optimising catalyst structure for \break
    ethylene/propylene
    copolymerisations}, L.\,M.~Broomfield, Y.~Sarazin, J.\,A.~Wright,
    D.\,L.~Hughes, W.~Clegg, R.\,W.~Harrington and M.~Bochmann,
    \emph{J.~Organomet. Chem.}, 2007, \textbf{692}, 4603--4611
  }
  
  \paper{10.1002/chem.200601751}{%
    \papertitle{The Synthesis, Structure and Reactivity of 
    \ce{B(C6F5)3}-Stabilised Amide (\ce{M-NH2}) Complexes of the Group~4
    Metals}, A.\,J.~Mountford, W.~Clegg, S.\,J.~Coles, R.\,W.~Harrington,
    P.\,N.~Horton, S.\,M.~Humphrey, M.\,B. Hursthouse, J.\,A.~Wright
    and S.\,J.~Lancaster, \emph{Chem. Eur.~J.}, 2007, \textbf{13},
    4535--4547
  }
  
  \paper{10.1107/S1600536807007052}{%
    \papertitle{Redetermination of 
    catena-poly[[sodium(I)-tri-μ-dimethylformamide-%
    κ\textsuperscript{6}O:O] iodide] at 140 K}, S.~Chessa and
    J.\,A.~Wright, \emph{Acta Cryst.~C}, 2007, \textbf{63}, m787--m789
  }
  
  \paper{10.1021/om061154b}{%
    \papertitle{Mono(arene) Complexes of Thallium(I) Supported by a Weakly
    Coordinating Anion}, Y.~Sarazin, N.~Kaltsoyannis, J.\,A.~Wright,
    M.~Bochmann, \emph{Organometallics}, 2007, \textbf{26}, 1811--1815
  }
  
  \paper{10.1021/ja0657105}{%
    \papertitle{Thallium(I) Sandwich, Multidecker, and Ether Complexes
    Stabilized by Weakly-Coordinating Anions: A Spectroscopic,
    Structural, and Theoretical Investigation}, Y.~Sarazin, 
    D.\,L.~Hughes, N.~Kaltsoyannis, J.\,A.~Wright and M.~Bochmann,
    \emph{J.~Am. Chem. Soc.}, 2007, \textbf{129}, 881--884
  }
  
  \paper{10.1016/j.jorganchem.2006.08.023}{%
    \papertitle{`Pincer' pyridyl- and bipyridyl-N-heterocyclic carbene
    analogues of the Grubbs' metathesis catalyst}, J.\,A.~Wright, 
    A.\,A.~Danopoulos, W.\,B.~Motherwell, R.\,J.~Carroll, S.~Ellwood,
    \emph{J.~Organomet. Chem.}, 2006, \textbf{691}, 5204--5210
  }
  
  \paper{10.1016/j.jorganchem.2006.09.021}{%
    \papertitle{Synthesis and crystal structure of
    \ce{[C6H5Hg(H2NSiMe3)][H2N\{B(C6F5)3\}2]}, a phenyl--mercury(II)
    cation stabilised by a non-coordinating counter-anion},
    Y.~Sarazin, J.\,A.~Wright and M.~Bochmann, \emph{J.~Organomet.
    Chem.}, 2006, \textbf{691}, 5680--5687
  }
  
  \paper{10.1002/ejic.200600559}{%
    \papertitle{``Pincer'' N-Heterocyclic Carbene Complexes of Rhodium
    Functionalised with Pyridyl and Bipyridyl Donors}, J.\,A.~Wright,
    A.\,A.~Danopoulos, W.\,B.~Motherwell, R.\,J.~Carroll, S.~Ellwood and
    J.~Saßmannshausen, \emph{Eur.~J. Inorg. Chem.}, 2006, 4857--4865
  }
  
  \paper{10.1107/S0108270106018695}{%
    \papertitle{\{Bis(3,5-Di-tert-butyl-2-oxidobenzyl)%
    [2-(N,N-dimethylamino)ethyl]amine-κ\textsuperscript{4}N,N',O,O'\}%
    zinc(II) and \{bis(3-\break tert-butyl-5-methyl-2-oxidobenzyl)%
    [2-(N,N-dimethylamino)ethyl]amine-κ\textsuperscript{4}N,N',O,O'\}%
    (tetrahdyrofuran)-\break zinc(II)}, R.\,H.~Howard, M.~Bochmann and
    J.\,A.~Wright, \emph{Acta Cryst.~C}, 2006, \textbf{62}, m293--m296
  }
  
  \paper{10.1039/b515548g}{%
    \papertitle{The synthesis of new weakly coordinating diborate anions:
    anion stability as a function of linker structure and steric bulk},
    M.\,H.~Hannant, J.\,A.~Wright, S.\,J.~Lancaster, D.\,L.~Hughes, 
    P.\,N.~Horton and M.~Bochmann, \emph{Dalton Trans.}, 2006,
    \textbf{35}, 2415--2426
  }
  
  \paper{10.1002/masy.200690042}{%
    \papertitle{Anion Influence in Metallocene-based Olefin Polymerisation
    Catalysts}, P.\,A.~Wilson, M.\,H.~Hannant,\break J.\,A.~Wright,
    R.~Cannon and M.~Bochmann, \emph{Macromol. Symp.}, 2006,
    \textbf{236}, 100--110
  }
  
  \paper{10.1039/b512133g}{%
    \papertitle{`Pincer' dicarbene complexes of some early transition
    metals and uranium}, D.~Pugh, J.\,A.~Wright, S.~Freeman and
    A.\,A.~Danopoulos, \emph{Dalton Trans.}, 2006, \textbf{35}, 775--782
  }
  
  \paper{10.1002/chem.200400644}{%
    \papertitle{Novel Anti-Markovnikov Regioselectivity in the Wacker 
    Reaction of Styrenes}, J.\,A.~Wright, M.\,J.~Gaunt and J.\,B.~Spencer,
    \emph{Chem.--Eur.~J.}, 2005, \textbf{12}, 949--955
  }
  
  \paper{10.1039/b415562a}{%
    \papertitle{Molecular \ce{N2} complexes of iron stabilised by
    N-heterocyclic pincer dicarbene ligands}, A.\,A.~Danopoulos,
    J.\,A.~Wright and W.\,B.~Motherwell, \emph{Chem. Commun.}, 2005,
    \textbf{41}, 784--786
  }
  
  \paper{10.1016/j.poly.2004.07.002}{%
    \papertitle{Picoline and pyridine functionalised chelate N-heterocyclic
    carbene complexes of nickel: synthesis and structural studies},
    S.~Winston, N.~Stylianides, A.\,A.~D.~Tulloch, J.\,A.~Wright and
    A.\,A.~Danopoulos, \emph{Polyhedron}, 2004, \textbf{23}, 2813--2820
  }
  
  \paper{10.1021/om049489l}{%
    \papertitle{N-Heterocyclic ``Pincer'' Dicarbene Complexes of Cobalt(I),
    Cobalt(II), and Cobalt(III)}, A.\,A.~Danopoulos, J.\,A.~Wright,
    W.\,B.~Motherwell and S.~Ellwood, \emph{Organometallics}, 2004,
    \textbf{23}, 4807--4810
  }
  
  \paper{10.1021/om0341911}{%
    \papertitle{N-Heterocyclic Pincer Dicarbene Complexes of Iron(II):
    C-2 and C-5 Metalated Carbenes on the Same Metal Center},
    A.\,A.~Danopoulos, N.~Tsoureas, J.\,A.~Wright and M.\,E.~Light,
    \emph{Organometallic}, 2004, \textbf{23}, 166--168
  }
  
  \paper{10.1016/S0040-4039(01)00563-9}{%
    \papertitle{Sequential removal of the benzyl-type protecting groups 
    PMB and NAP by oxidative cleavage using CAN and DDQ}, 
    J.\,A.~Wright, J.-Q.~Yu and J.\,B.~Spencer, \emph{Tetrahedron Lett.},
    2001, \textbf{41}, 4033--4036%
  }
\end{etaremune}

\section{Posters}

\begin{itemize}
  \item \emph{Replacing Precious Metals: Iron-Containing Enzymes Show
    the Way}, J.~A. Wright and C.~J. Pickett, \emph{SET for Britain
    2011}, Houses of Parliament, London, 14th March 2011.
    
  \item \emph{[FeFe-Hydrogenase subsite analogues: seeing protonation
    intermediates by stopped-flow spectroscopy]}, J.~A. Wright,
    P.~J. Turrell and C.~J. Pickett, \emph{Faraday Discussions 148},
    Nottingham, UK, 5th to 7th July 2010.
    
  \item \emph{Mechanistic investigations into the Wacker reaction of
    styrenes}, J.~A. Wright, M.~J. Gaunt and J.~B. Spencer,
    \emph{13th International Symposium on Homogeneous Catalysis},
    Tarragona, Spain, 3rd to 7th September 2002.
\end{itemize}

\begin{comment}

\section{Publications}

A total of 44 publications in per-reviewed journals. Recent papers
include
\begin{itemize}
  \paper{10.1039/c1cc11320h}{
    \papertitle{The role of CN and CO ligands in the vibrational relaxation
    dynamics of model compounds of the [FeFe]-\break hydrogenase enzyme},
    S.~Kaziannis, J.~A. Wright, M.~Candelaresi, R.~Kania, G.~M. Greetham,
    A.~W. Parker, C.~J. Pickett and N.~T. Hunt 
    \emph{Phys. Chem. Chem. Phys.}, 2011, \textbf{13}, 10295--10305
  }
  
  \paper{10.1039/c1cc11320h}{
    \papertitle{The hafnium-mediated NH activation of an amido-borane},
    E.~A. Jacobs, A.-M. Fuller, S.~J. Lancaster and J.~A. Wright,
    \emph{Chem. Commun.}, 2011, \textbf{47}, 5870--5872
  }
  
  \paper{10.1002/ejic.201001072}{
    \papertitle{[FeFe]-Hydrogenase models: unexpected variation in
    protonation rate between dithiolate bridge analogues},
    A.~Jablonskytė, J.~A. Wright and C.~J. Pickett, \emph{Eur.~J.
    Inorg. Chem.}, 2011, 1033--1037
  }

  \paper{10.1002/anie.201004189}{
    \papertitle{The Third Hydrogenase: A Ferracyclic Carbamoyl with Close
    Structural Analogy to the Active Site of Hmd}, P.~J. Turrell,
    J.~A. Wright, J.~N.~T. Peck, V.~Oganesyan and C.~J. Pickett,
    \emph{Angew. Chem. Int. Ed.}, 2010, \textbf{49}, 7508--7511
  }
\end{itemize}

\end{comment}

\section{Talks}

\begin{itemize}
  \item \emph{The mono-iron hydrogenase: accurate active site models},
    Coordination Chemistry Discussion Group Meeting, University of
    East Anglia, 8th July 2011.
  \item \emph{Model hydrogenase subsites: seeing hydride formation as
    it happens}, Joe Spencer Memorial Lectures, University of
    Cambridge, 23rd April 2010.
  \item \emph{Mechanism studies on hydrogenase models},
    Dibrugarh University, Dibrugarh, India, 4th December 2009.
  \item \emph{Mechanism studies on hydrogenase models},
    Second Young Indian Scientists Networking Conference,
    Kolkata, India, 2nd December 2009.
\end{itemize}

\end{document}