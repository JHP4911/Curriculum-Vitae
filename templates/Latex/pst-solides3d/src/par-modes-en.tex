\section{The \Index{modes}}

For some solids, there are certain gratings predefined.
We can setup the key values to \texttt{\Lkeyword{mode}=0, 1, 2, 3 or 4} which allows to have some some gratings from very coarse  \texttt{\Lkeyword{mode}=0} up to very fine \texttt{\Lkeyword{mode}=4}.

This permits us to have a draft version of a solid with \texttt{\Lkeyword{mode}=0} (fewer calculations) and then refine it with \texttt{\Lkeyword{mode}=4} for the final version.

\psResetSolidKeys
%% avec mode = 0
\begin{center}
\psset{lightsrc=10 5 0,viewpoint=50 20 -40 rtp2xyz,Decran=35,unit=0.5cm,%
       incolor=white,fillcolor=green!50,r0=5,r1=2,h=5,object=troncconecreux,r0=5,r1=2,h=5}
\begin{pspicture}(-5,-5)(5,5)
\psframe(-5,-5)(5,5)
\psSolid[mode=0]
\rput(0,-4.5){\psframebox[fillstyle=solid,fillcolor=black]{\small \textcolor{white}{\texttt{[mode=0]}}}}
\end{pspicture}
%
\begin{pspicture}(-5,-5)(5,5)
\psframe(-5,-5)(5,5)
\psSolid[mode=1]%
\rput(0,-4.5){\psframebox[fillstyle=solid,fillcolor=black]{\small\textcolor{white}{\texttt{[mode=1]}}}}
\end{pspicture}
%
\begin{pspicture}(-5,-5)(5,5)
\psframe(-5,-5)(5,5)
\psSolid[mode=2]%
\rput(0,-4.5){\psframebox[fillstyle=solid,fillcolor=black]{\textcolor{white}{\texttt{[mode=2]}}}}
\end{pspicture}
%
\begin{pspicture}(-5,-5)(5,5)
\psframe(-5,-5)(5,5)
\psSolid[mode=3]%
\rput(0,-4.5){\psframebox[fillstyle=solid,fillcolor=black]{\textcolor{white}{\texttt{[mode=3]}}}}
\end{pspicture}
%
\begin{pspicture}(-5,-5)(5,5)
\psframe(-5,-5)(5,5)
\psSolid[mode=4]%
\rput(0,-4.5){\psframebox[fillstyle=solid,fillcolor=black]{\textcolor{white}{\texttt{[mode=4]}}}}
\end{pspicture}
%
\begin{pspicture}(-5,-5)(5,5)
\psframe(-5,-5)(5,5)
\psSolid[mode=5]%
\rput(0,-4.5){\psframebox[fillstyle=solid,fillcolor=black]{\small\textcolor{white}{\texttt{[mode=5] => [mode=4] forced}}}}
\end{pspicture}
\end{center}
%\newpage

\endinput
