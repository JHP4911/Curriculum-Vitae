\section{The \Index{jps code}}

\textit{jps code} contains all the  PostScript code that is used by the library
developed for the software \textit{jps2ps}.

The \texttt{solides.pro} file of the \texttt{solides3d} package
contains all the elements native
to that library, which contains about $4\, 500$~functions and
procedures.

It allows us to have available some adapted commands in mathematical form, without
having to construct them with the primitives
 \verb+moveto+, \verb+lineto+, \verb+curveto+, etc.

For example, we can define a function $F$ with $F(t) =
(3\cos^3 t, 3\sin^3 t)$, and draw its curve
%I'm not sure of the distinction between the curve and the path, as you use it
with the \textit{jps code} \verb+0 360 {F} CourbeR2+.

If we only want to have the path of that curve, we use the code
\verb+0 360 {F} CourbeR2_+,
and if we want to add this to the stack of points of the curve,
we use \verb+0 360 {F} CourbeR2++.

In all of the $3$~examples below, the number of points is declared by the global
variable \Lkeyword{resolution}.

In other words, with the \Index{function} $F$ named above and a fixed resolution of 36, the \textit{jps code}
\begin{verbatim}
   0 360 {F} CourbeR2+
\end{verbatim}
is equivalent to the PostScript code
\begin{verbatim}
   0 10 360 {
      /angle exch def
      3 angle cos 3 exp mul
      3 angle sin 3 exp mul
   } for
\end{verbatim}

We haven't yet developed documentation for the library hidden in the
\texttt{solides.pro} file. For the moment we refer the
\textit{Guide de l'utilisateur de jps2ps} for the interested user
available at the website  \url{melusine.eu.org/syracuse/bbgraf}.

\endinput
