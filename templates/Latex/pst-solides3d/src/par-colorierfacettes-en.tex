\section{\Index{Colouring} some single faces}

The key value \texttt{\Lkeyword{fcol}=$i_0$~($c_0$) $i_1$~($c_1$) \dots $i_n$~($c_n$)},
where $i_k$ are integers and $c_k$ the names of the colours, permits to
specify a \Index{colour} for special \Index{faces}.
To the face with the index $i_k$ corresponds the colour $c_k$. The
integer $n$ must be lower than the maximum of the number of faces of the chosen solid.


%% L'option \texttt{[fcol=1 (OliveGreen) 0 (color1) 4 (color2) etc.]}
%% permet de sp\'{e}cifier dans l'ordre :
%% \begin{compactitem}
%%   \item le num\'{e}ro de la facette  de \texttt{0} \`{a} \texttt{n-1}, pour \texttt{n} facettes ;
%%   \item la couleur de la facette.
%% \end{compactitem}

The colour names $c_k$, there are $68$~predefined values, are defined names in the
\texttt{color.pro}. These values are:
\textsl{GreenYellow},
\textsl{Yellow},
\textsl{Goldenrod},
\textsl{Dandelion},
\textsl{Apricot},
\textsl{Peach},
\textsl{Melon},
\textsl{YellowOrange},
\textsl{Orange},
\textsl{BurntOrange},
\textsl{Bittersweet},
\textsl{RedOrange},
\textsl{Mahogany},
\textsl{Maroon},
\textsl{BrickRed},
\textsl{Red},
\textsl{OrangeRed},
\textsl{RubineRed},
\textsl{WildStrawberry},
\textsl{Salmon},
\textsl{CarnationPink},
\textsl{Magenta},
\textsl{VioletRed},
\textsl{Rhodamine},
\textsl{Mulberry},
\textsl{RedViolet},
\textsl{Fuchsia},
\textsl{Lavender},
\textsl{Thistle},
\textsl{Orchid},
\textsl{DarkOrchid},
\textsl{Purple},
\textsl{Plum},
\textsl{Violet},
\textsl{RoyalPurple},
\textsl{BlueViolet},
\textsl{Periwinkle},
\textsl{CadetBlue},
\textsl{CornflowerBlue},
\textsl{MidnightBlue},
\textsl{NavyBlue},
\textsl{RoyalBlue},
\textsl{Blue},
\textsl{Cerulean},
\textsl{Cyan},
\textsl{ProcessBlue},
\textsl{SkyBlue},
\textsl{Turquoise},
\textsl{TealBlue},
\textsl{Aquamarine},
\textsl{BlueGreen},
\textsl{Emerald},
\textsl{JungleGreen},
\textsl{SeaGreen},
\textsl{Green},
\textsl{ForestGreen},
\textsl{PineGreen},
\textsl{LimeGreen},
\textsl{YellowGreen},
\textsl{SpringGreen},
\textsl{OliveGreen},
\textsl{RawSienna},
\textsl{Sepia},
\textsl{Brown},
\textsl{Tan},
\textsl{Gray},
\textsl{Black},
\textsl{White}.
The list of these $68$ colours is available in the command
\verb+\colorfaces+ (see an example in the section about
the grating  of a cube).

Thinking on that case, the number of the faces
$\mathtt{n_1\times n_2}+2\texttt{(outer faces  inner faces)}$
must be lower than 68!

However users can define their own \Index{colours}. There are two methods:

\begin{compactitem}
\item They can use one of the $4$~optional arguments  \texttt{color1},
  \texttt{color2}, \texttt{color3}, \texttt{color4} from
  \Lcs{psSolid}, then transmit to \Lkeyword{fcol} a pair of the type
  $i$~\verb+(color1)+, where $i$ is the index of the chosen face. The
  arguments \texttt{color1}, etc.  are used in the same way as the
  arguments from \Lkeyword{color} and \Lkeyword{incolor}.\hfill \break
  A possible command could be the following:
  \begin{verbatim}
   \psSolid[a=1,object=cube,color1=red!60!yellow!20,fcol=0 (color1)]%
  \end{verbatim}
\item They define their own colour names with the command
  \verb+\pstVerb+, and then use these names with the argument
  \Lkeyword{fcol}. For example:
\begin{verbatim}
\pstVerb{/hetre {0.764 0.6 0.204 setrgbcolor} def
         /chene {0.568 0.427 0.086 setrgbcolor} def
         /cheneclair {0.956 0.921 0.65 setrgbcolor} def
         }%
\end{verbatim}
And therefore:
\begin{verbatim}
fcol=0 (hetre) 1 (chene)  2 (cheneclair)
\end{verbatim}
\end{compactitem}


The $4$~arguments
\verb+color1+,
\verb+color2+,
\verb+color3+,
\verb+color4+ have default values:
\begin{compactitem}
    \item \textcolor{cyan!50}{color1=cyan!50}
    \item \textcolor{magenta!60}{color2=magenta!60}
    \item \textcolor{blue!30}{color3=blue!30}
    \item \textcolor{red!50}{color4=red!50}
\end{compactitem}




\begin{LTXexample}[width=5cm]
\psset{Decran=20,viewpoint=10 5 10,unit=0.5}
\begin{pspicture}(-5,-5)(5,5)
\psSolid[
   fcol=0 (Apricot) 1 (Aquamarine)  2 (Bittersweet)
        3 (ForestGreen) 4 (Goldenrod)
        13 (GreenYellow)
        40 (Mulberry),
   object=cube,mode=3]%
\end{pspicture}
\end{LTXexample}

\begin{LTXexample}[width=4.9cm]
\psset{Decran=20,viewpoint=10 5 10,unit=0.5}
\begin{pspicture}(-5,-5)(5,5)
\psSolid[
   fcol=0 (Apricot) 2 (Lavender) 3 (SkyBlue)  11 (LimeGreen) 12 (OliveGreen),
   object=cylindre,
   h=4,
   ngrid=4 10](0,0,-2)
\end{pspicture}
\end{LTXexample}

The choice of the faces to be coloured can be specified with some PostScript code,
\begin{verbatim}
fcol=48 {i (Black) i 1 add (LimeGreen) i 2 add (Yellow) /i i 3 add store} repeat
\end{verbatim}
which will alternately colour the faces in black, green and yellow.
\begin{center}
\begin{LTXexample}[width=6.5cm]
\begin{pspicture}(-3,-3)(3.5,2.5)
\psset{Decran=7.5,viewpoint=10 10 5}
\pstVerb{/iface 0 store}%
\psSolid[
fcol=48 {iface (Black)
  iface 1 add (LimeGreen)
  iface 2 add (Yellow) /iface
  iface 3 add store} repeat,
  r1=4,r0=1,
  object=tore,
  ngrid=8 18,
  RotY=30]
\end{pspicture}
\end{LTXexample}
\end{center}

When the option \Lkeyword{hue} is activated, the faces of the solid are coloured with the nuance of the rainbow colours.

\begin{LTXexample}[width=5.9cm]
\begin{pspicture}(-3,-2.5)(3,2.5)
\psset[pst-solides3d]{viewpoint=50 50 50,Decran=40,lightsrc=50 20 1e2}
\psSolid[r1=5,r0=1,object=tore,ngrid=16 18,hue=0 1]%
\end{pspicture}
\end{LTXexample}

\endinput
