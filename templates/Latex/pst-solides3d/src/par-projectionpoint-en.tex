\section{Points}

\subsection{Direct definition}

The object \Lkeyword{point} defines a \Index{point}. The values $(x,y)$ of
its coordinates can be passed directly to the macro
\Lcs{psProjection} or indirectly via the option \Lkeyword{args}.

Thus the two commands \verb+\psProjection[object=point](1,2)+ and
\verb+\psProjection[object=point,arg=1 2]+ are equivalent and lead
to the projection of the point with coordinates $(1,2)$ onto the
chosen plane.

\subsection{Labels}

The option \texttt{\Lkeyword{text}=my text} allows us to project a string of
characters onto the chosen plane next to a chosen point. The
positioning is made with the argument \texttt{\Lkeyword{pos}=value} where
\texttt{value} is one of the following $\{$ul, cl, bl, dl, ub, cb, bb,
db, uc, cc, bc, dc, ur, cr, br, dr$\}$.

The details of the parameter \Lkeyword{pos} will be discussed in a
later paragraph.

\begin{LTXexample}[width=7.5cm]
\begin{pspicture}(-3,-3)(4,3.5)%
\psframe*[linecolor=blue!50](-3,-3)(4,3.5)
\psset{viewpoint=50 30 15,Decran=60}
\psset{solidmemory}
%% definition du plan de projection
\psSolid[object=plan,
   definition=equation,
   args={[1 0 0 0] 90},
   name=monplan,
   planmarks,
   showBase]
\psset{plan=monplan}
%% definition du point A
\psProjection[object=point,
   args=-2 1,
   text=A,
   pos=ur]
\psProjection[object=point,
   text=B,
   pos=ur](2,1)
\composeSolid
\axesIIID(4,2,2)(5,4,3)
\end{pspicture}
\end{LTXexample}



\subsection{Naming and memorising a point}

If the option \texttt{\Lkeyword{name}=myName} is given, the coordinates
$(x,y)$ of the chosen point are saved under the name \texttt{myName} and so
can be reused.

\subsection{Some other definitions}

There are other methods to define a point in 2D. The options
\Lkeyword{definition} and \Lkeyword{args} support the following
methods:

\begin{itemize}

\item \texttt{\Lkeyword{definition}=\Lkeyval{milieu}};
\texttt{\Lkeyword{args}=$A$ $B$}.

The midpoint of the line segment $[AB]$

\item \texttt{\Lkeyword{definition}=\Lkeyval{parallelopoint}};
\texttt{\Lkeyword{args}=$A$ $B$ $C$}.

The point $D$ for which $(ABCD)$ is a
parallelogram.

\item \texttt{\Lkeyword{definition}=\Lkeyval{translatepoint}};
\texttt{\Lkeyword{args}=$M$ $u$}.

The image of the point $M$ shifted by the vector
$\vec u$


\item \texttt{\Lkeyword{definition}=\Lkeyval{rotatepoint}};
\texttt{\Lkeyword{args}=$M$ $I$ $r$}.

The image of the point $M$ under a
rotation about the point $I$ through an angle $r$ (in degrees)

\item \texttt{\Lkeyword{definition}=\Lkeyval{hompoint}};
\texttt{\Lkeyword{args}=$M$ $A$ $k$}.

The point $M'$ satisfying
$\overrightarrow {AM'} = k \overrightarrow {AM}$

\item \texttt{\Lkeyword{definition}=\Lkeyval{orthoproj}};
\texttt{\Lkeyword{args}=+$M$ $d$}.

The orthogonal projection of the point
$M$ onto the line $d$.

\item \texttt{\Lkeyword{definition}=\Lkeyval{projx}};
\texttt{\Lkeyword{args}=$M$}.

The projection of the point $M$ onto the $Ox$
axis.

\item \texttt{\Lkeyword{definition}=\Lkeyval{projy}};
\texttt{\Lkeyword{args}=$M$}.

The projection of the point $M$ onto the $Oy$
axis.

\item \texttt{\Lkeyword{definition}=\Lkeyval{sympoint}};
\texttt{\Lkeyword{args}=$M$ $I$}.

The point of symmetry of $M$ with respect
to the point $I$.

\item \texttt{\Lkeyword{definition}=\Lkeyval{axesympoint}};
\texttt{\Lkeyword{args}=$M$ $d$}.

The axially symmetrical point of $M$ with
respect to the line $d$.

\item \texttt{\Lkeyword{definition}=\Lkeyval{cpoint}};
\texttt{\Lkeyword{args}=$\alpha $ $C$}.

The point corresponding to the
angle $\alpha $ on the circle $C$

\item \texttt{[definition=xdpoint]};
\verb+args=+$x$ $d$.

The $Ox$ intercept $x$ of the line $d$.

\item \texttt{\Lkeyword{definition}=\Lkeyval{ydpoint}};
\texttt{\Lkeyword{args}=$y$ $d$}.

The $Oy$ intercept $y$ of the line $d$.

\item \texttt{\Lkeyword{definition}=\Lkeyval{interdroite}};
\texttt{\Lkeyword{args}=$d_1$ $d_2$}.

The intersection point of the lines
$d_1$ and $d_2$.

\item \texttt{\Lkeyword{definition}=\Lkeyval{interdroitecercle}};
\texttt{\Lkeyword{args}=$d$ $I$ $r$}.

The intersection points of the line
$d$ with a circle of centre $I$ and radius $r$.

\end{itemize}

In the example below, we define and name three points $A$, $B$ and
$C$, and then calculate the point $D$ for which $(ABCD)$ is a
parallelogram together with the centre of this parallelogram.

\begin{LTXexample}[width=7.5cm]
\begin{pspicture}(-3,-3)(4,3.5)%
\psframe*[linecolor=blue!50](-3,-3)(4,3.5)
\psset{viewpoint=50 30 15,Decran=60}
\psset{solidmemory}
%% definition du plan de projection
\psSolid[object=plan,
   definition=equation,
   args={[1 0 0 0] 90},
   name=monplan,
   planmarks,
   showbase]
\psset{plan=monplan}
%% definition du point A
\psProjection[object=point,
   text=A,pos=ur,name=A](-1,.7)
%% definition du point B
\psProjection[object=point,
   text=B,pos=ur,name=B](2,1)
%% definition du point C
\psProjection[object=point,
   text=C,pos=ur,name=C](1,-1.5)
%% definition du point D
\psProjection[object=point,
   definition=parallelopoint,
   args=A B C,
   text=D,pos=ur,name=D]
%% definition du point G
\psProjection[object=point,
   definition=milieu,
   args=D B]
\composeSolid
\axesIIID(4,2,2)(5,4,3)
\end{pspicture}
\end{LTXexample}

\endinput

