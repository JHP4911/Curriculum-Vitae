\section{Positioning a named point}

\begin{verbatim}
\psPoint(x,y,z){name}
\end{verbatim}
This is a command similar to \verb+\pnode(! x y){name}+. It places
the node \texttt{(name)} at the point with the coordinates $(x,y,z)$,
viewed with the chosen point of view \texttt{\Lkeyword{viewpoint}=vx vy vz}. We can
now use the point to mark it, draw lines, polygons, etc.

Let's place the centres of the atoms of the methanol molecule $\mathrm{CH_3COH}$.

\begin{LTXexample}[width=8cm]
\begin{pspicture}(-4,-4)(4,5)
\psset{viewpoint=100 50 20 rtp2xyz,Decran=20}
\axesIIID(3,3,3)(20,20,20)
\psPoint(-4.79,2.06,0){C1}
\psPoint(-4.79,15.76,0){Ox}
\psPoint(8.43,5.57,0){C2}
\psPoint(-14.14,3.34,0){H3}
\psPoint(14.14,-2.94,8.90){H6}
\psPoint(14.14,-2.94,-8.90){H7}
\psPoint(6.43,-16.29,0){H8}
\psline(C1)(H3)\psline(C2)(H7)
\psline(C2)(H8)\psline(C1)(C2)
\psline[doubleline=true](C1)(Ox)
\psline(C2)(H6)
\uput[r](H3){$\mathrm{H_1}$}
\uput[l](H6){$\mathrm{H_2}$}
\uput[l](H7){$\mathrm{H_3}$}
\uput[l](H8){$\mathrm{H_4}$}
\uput{0.25}[u](C1){$\mathrm{C_1}$}
\uput{0.25}[d](C2){$\mathrm{C_2}$}
\uput{0.25}[r](Ox){$\red\mathrm{O}$}
\psdots[dotstyle=o,dotsize=0.3](H3)(H6)(H7)(H8)
\psdots[dotsize=0.4](C1)(C2)
\psdot[linecolor=red,dotsize=0.4](Ox)
\end{pspicture}
\end{LTXexample}


\endinput
