\documentclass[a4papers]{scrartcl}
\usepackage[latin1]{inputenc}
\usepackage[T1]{fontenc}
\usepackage[ngerman]{babel}

\usepackage[sf,center]{titlesec}
\usepackage[usenames,dvipsnames]{color}
\usepackage{blindtext}
\usepackage{lmodern}
\usepackage{fancyhdr}
\usepackage{graphicx}
%\usepackage{picins}
\usepackage{wrapfig}
\usepackage{marvosym}

\pagestyle{fancy}

%Kopf- und Fu�zeile definieren
\lhead{}
\rhead{}
\renewcommand{\headrulewidth}{0pt}
% \lfoot{\sf \tiny Pascal Huber, Euskirchnerstra�e 73, 53121 Bonn, \Mobilefone\, 0176 84007003, \Letter\, pascal.huber@aol.de}{\bfseries \rightmark}
\cfoot{\thepage}
\rfoot{}


%Definition der �berschriften
\titleformat{\section}[block]{\sf\Huge}{}{0pt}{\color{Gray}}[\color{Black}\titlerule]
\titleformat{\subsection}[block]{\sf\large}{}{0pt}{\color{RoyalBlue}
 \rule[0.5ex]{2.1cm}{2pt}\rule[0.5ex]{1cm}{2pt}\quad}
\titleformat{\subsubsection}[block]{\sf}{}{0pt}{\color{RoyalBlue}{\fontfamily{cmss}\fontseries{bx}\fontshape{n}\selectfont}\rule[0.5ex]{2.1cm}{0pt}\rule[0.5ex]{0.5cm}{0pt}\rule[0.5ex]{0.5cm}{0.5pt}\quad}

\begin{document}


\section{Lebenslauf}

%Abstand zum Titel

%Bild einf�gen
%\parpic(0cm,6.3cm)[r][]{\fbox{\includegraphics[width=3.7cm]{Pascal_Lebenslaufbild.jpg}}}


\subsection{Zur eigenen Person}

\begin{list}{}
{\setlength{\topsep}{1.0cm}
\setlength{\itemsep}{0cm}
\setlength{\leftmargin}{3.6cm}
\setlength{\labelwidth}{3.2cm}
\setlength{\labelsep}{0.4cm}
\renewcommand{\makelabel}[1]{\fontfamily{cmss}\fontseries{x}\fontshape{n}
\selectfont #1}}
\item[Name] Pascal Huber
\item[Anschrift] Euskirchner Stra�e 73
\item 53121 Bonn
\item[Mobil] 0176 84007003
\item[E-Mail] pascal.huber@aol.de
\item[Geburtsdatum] 20.08.1989, Freiburg im Breisgau
\item[Familienstand] ledig
\end{list}


\vspace{-2.5\baselineskip}
\subsection{Studium \& Schule}

\begin{list}{}
{\setlength{\topsep}{1.0cm}
\setlength{\itemsep}{0cm}
\setlength{\leftmargin}{3.6cm}
\setlength{\labelwidth}{3.2cm}
\setlength{\labelsep}{0.4cm}
\renewcommand{\makelabel}[1]{\fontfamily{cmss}\fontseries{x}\fontshape{n}
\selectfont #1}}
\item[10/2013 - heute] Masterstudium der Mathematik, \\
  Rheinische Friedrich-Wilhelms-Universit�t, Bonn \\
  Schwerpunkt in Numerischer Mathematik
\item[10/2012 - 09/2013] Masterstudium der Mathematik, \\
  Technischen Universit�t, M�nchen
\item[10/2009 - 09/2012] Bachelorstudium der Mathematik, \\
  Rheinischen Friedrich-Wilhelms-Universit�t, Bonn \\
  Schwerpunkte in Analysis und Stochastik \\
  Abschlussnote 1.1
\item[02/2012 - 08/2012] Bachelorarbeit zum Thema ``\textit{Bulk universality of
    Wigner Matrices}'' \\
  Betreuer: Prof. Benjamin Schlein
\item[09/2000 - 06/2008] Deutsch-Franz�sisches Gymnasium, Freiburg \\
  Abiturnote 1.0
\end{list}


\vspace{-2.5\baselineskip}
\subsection{Praktische Erfahrungen}

\begin{list}{}
{\setlength{\topsep}{1.0cm}
\setlength{\itemsep}{0cm}
\setlength{\leftmargin}{3.6cm}
\setlength{\labelwidth}{3.2cm}
\setlength{\labelsep}{0.4cm}
\renewcommand{\makelabel}[1]{\fontfamily{cmss}\fontseries{x}\fontshape{n}
\selectfont #1}}
\item[12/2013 - heute] SHK-T�tigkeit am Fraunhofer-Institut f�r
  Algorithmen und Wissenschaftliches Rechnen SCAI, Bonn \\
  GUI-Programmierung und Mitarbeit an einem Softwarepaket zur
  numerischen Simulation in der Molek�ldynamik (``\textit{Tremolo-X}'')
\item[04/2013 - 09/2013] Praktikum beim Fraunhofer-Institut f�r
  Eingebettete Systeme und Kommunikationstechnik ESK in  M"unchen zum
  Thema ``\textit{Implementierung von Kartenbezogenen Algorithmen}''\\
  Implementierung von Routing- und Map-Matching-Algorithmen im Rahmen
  eines Software-Frameworks f�r vernetzte Verkehrstechnik
\item[09/2011 - 03/2012] C++-Programmierpraktikum zum Thema
  ``\textit{Partikelmethoden und gitterlose Diskretisierung}'', \\
  Rheinische Friedrich-Wilhelms-Universit�t, Bonn\\
  Programmierung, Parallelisierung und Visualisierung am Beispiel von
  numerischen Simulationen in der Molek�ldynamik
\item[09/2011 - 03/2012] Stelle als Tutor der Vorlesung
  ``\textit{Einf�hrung in die Wahrscheinlichkeitstheorie}''
\item[09/2010 - 03/2012] Stelle als Tutor der Vorlesung
  ``\textit{Algorithmische Mathematik I}''
\end{list}


\vspace{-2.5\baselineskip}
\subsection{Auslandsaufenthalt}

\begin{list}{}
{\setlength{\topsep}{1.0cm}
\setlength{\itemsep}{0cm}
\setlength{\leftmargin}{3.6cm}
\setlength{\labelwidth}{3.2cm}
\setlength{\labelsep}{0.4cm}
\renewcommand{\makelabel}[1]{\fontfamily{cmss}\fontseries{x}\fontshape{n}
\selectfont #1}}
\item[09/2008-07/2009] Auslandsaufenthalt in Australien und Neuseeland
  ``Work\&Travel''\\
  Mitarbeit auf landwirtschaftlichen Betrieben
\end{list}


\vspace{-2.5\baselineskip}
\subsection{Zusatzqualifikationen}

\subsubsection{Sprachkenntnisse}

\begin{list}{}
{\setlength{\topsep}{1.0cm}
\setlength{\itemsep}{0cm}
\setlength{\leftmargin}{3.6cm}
\setlength{\labelwidth}{3.2cm}
\setlength{\labelsep}{0.4cm}
\renewcommand{\makelabel}[1]{\fontfamily{cmss}\fontseries{x}\fontshape{n}
\selectfont #1}}
\item[Franz�sisch] flie�end in Wort und Schrift
\item[Englisch] gute Kenntnisse Wort und Schrift
\end{list}

\vspace{-2.5\baselineskip}
\subsubsection{EDV-Kenntnisse}

\begin{list}{}
{\setlength{\topsep}{1.0cm}
\setlength{\itemsep}{0cm}
\setlength{\leftmargin}{3.6cm}
\setlength{\labelwidth}{3.2cm}
\setlength{\labelsep}{0.4cm}
\renewcommand{\makelabel}[1]{\fontfamily{cmss}\fontseries{x}\fontshape{n}
\selectfont #1}}
\item[C und C++] sehr gute Kenntnisse
\item[Qt] Grundkenntnisse
\item[Java] Grundkenntnisse
\item[LaTeX] sehr gute Kenntnisse
\end{list}

\vspace{-2.5\baselineskip}
\subsection{Interessen}

\begin{list}{}
{\setlength{\topsep}{1.0cm}
\setlength{\itemsep}{0cm}
\setlength{\leftmargin}{3.6cm}
\setlength{\labelwidth}{3.2cm}
\setlength{\labelsep}{0.4cm}
\renewcommand{\makelabel}[1]{\fontfamily{cmss}\fontseries{x}\fontshape{n}
\selectfont #1}}
\item[] Tennis, Schwimmen, Badminton
\end{list}


\vspace{3cm}
Bonn, den \today


\end{document}