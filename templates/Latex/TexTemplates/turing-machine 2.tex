% Author: Ludger Humbert <https://haspe.homeip.net/cgi-bin/pyblosxom.cgi>
% Source: 
% https://haspe.homeip.net/projekte/ddi/browser/tex/pgf2/turingmaschine-schema.tex
%
\documentclass{article}
\usepackage{tikz}
\usetikzlibrary{shapes.arrows,chains}
\usepackage{verbatim}

\begin{comment}
:Title: Turing machine
:Slug: turing-machine

An illustration of an `universal Turing machine`_.

.. _universal Turing machine: http://en.wikipedia.org/wiki/Universal_Turing_machine

:Source: `https://haspe.homeip.net/projekte/ddi/browser/tex/pgf2`__

.. __: https://haspe.homeip.net/projekte/ddi/browser/tex/pgf2
.. _Ludger Humbert: https://haspe.homeip.net/cgi-bin/pyblosxom.cgi
\end{comment}

\usepackage[ngerman]{babel}


% serifenfreier Font -- fuer Praesentation geeignet/er
\renewcommand\familydefault{\sfdefault} 

\listfiles % damit im Log alle benutzten Pakete aufgelistet werden

\begin{document}

\begin{tikzpicture}[
      start chain=1 going right,start chain=2 going below,node distance=-0.15mm
    ]
    \node [on chain=2] {Tape};
    \node [on chain=1] at (-1.5,-.4) {\ldots};  
    \foreach \x in {1,2,...,11} {
        \x, \node [draw,on chain=1] {};
    } 
    \node [name=r,on chain=1] {\ldots}; 
    \node [name=k, arrow box, draw,on chain=2,
        arrow box arrows={east:.25cm, west:0.25cm}] at (-0.335,-.65) {};    
    \node at (1.5,-.85) {Read/write head};
    \node [on chain=2] {};
    \node [draw,on chain=2] {Program};
    \chainin (k) [join]; % Verbindung vom Programm zum Leseschreibkopf
\end{tikzpicture}

\end{document}

