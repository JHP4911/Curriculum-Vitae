% Hypercycle
% Author: Florian Hollandt
\documentclass{article}

\usepackage{tikz}
%%%<
\usepackage{verbatim}
\usepackage[active,tightpage]{preview}
\PreviewEnvironment{tikzpicture}
\setlength\PreviewBorder{5pt}%
%%%>

\begin{comment}
:Title:  Hypercycle
:Grid: 2x1

This example shows the power and flexibility of PGF's style mechanism.
The same paths are used to draw the diagrams, but by altering styles
different visual results are achieved. 

\end{comment}
\usetikzlibrary{shapes}

\setlength\oddsidemargin{0in}


\begin{document}

\pagestyle{empty}


\def\n{5}   % Number of Information carrier / Enzyme pairs
\def\d{2}   % Width of the figure


\tikzstyle{every picture}=[execute at end picture={
\foreach \x in {1,...,\n}{
        \path (360*\x/\n:\d)
        node [info carrier] (I\x) {$I_{\x}$};
    \draw [I to I]
        (I\x.200+360*\x/\n) arc (140:360:\r/6) arc (0:20:\r/6);
        \path (360*\x/\n:2*\d) 
        node [enzyme] (E\x) {$E_{\x}$};
    \draw  [I to E]
        (I\x) to (E\x);
    \draw [E to I]
        (360*\x/\n-360/\n:2*\d) to (I\x.center);
};
}]


\paragraph{Plain}

\begin{center}
\tikzstyle{radius}=[
    execute at begin picture={
        \def\r{2*3/4}
    }
]
\tikzstyle{every node}=[
    node distance=\d cm
]
\tikzstyle{info carrier}=[
    inner sep=2pt
]
\tikzstyle{enzyme}=[]
\tikzstyle{every path}=[
    shorten >=2pt,
    shorten <=2pt
]
\tikzstyle{I to E}=[
    ->,
    shorten >=2pt,
    shorten <=2pt
]
\tikzstyle{E to I}=[
    ->,
    densely dotted,
    shorten >=6mm,
    shorten <=4mm
]
\tikzstyle{I to I}=[
    ->,
    rotate=360*\x/\n
]
\begin{tikzpicture}[radius]
\end{tikzpicture}
\end{center}

\paragraph{Fancy}

\begin{center}
\tikzstyle{radius}=[
    execute at begin picture={
        \def\r{2.1}
    }
]
\tikzstyle{every node}=[
    node distance=\d cm
]
\tikzstyle{info carrier}=[
    shape=rectangle,
    draw=black,
    rounded corners=0.5,
    shading=radial,
    outer color=orange,
    inner color=white
]
\tikzstyle{enzyme}=[
    inner sep=2pt,
    shape=ellipse,
    draw=black,
    rounded corners=0.5,
    ball color=green
]
\tikzstyle{every path}=[
    shorten >=2pt,
    shorten <=2pt,
    thick
]
\tikzstyle{I to E}=[
    -stealth,
    controls=(10+360*\x/\n:\d*4/3) and (-10+360*\x/\n:\d*5/3),
    shorten >=2pt,
    shorten <=2pt,
    in=360*\x/\n+225
]
\tikzstyle{E to I}=[
    -stealth,
    controls=(360*\x/\n-360/\n+30:\d*5/3) and (360*\x/\n-360/\n-00:\d*4/3),
    densely dotted,
    shorten >=8mm,
    shorten <=5mm
]
\tikzstyle{I to I}=[
    -stealth,
    shorten >=0pt,
    shorten <=2pt,
    rotate=360*\x/\n
]
\begin{tikzpicture}[radius]
\end{tikzpicture}
\end{center}

\end{document}
