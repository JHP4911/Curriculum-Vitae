% Copyright 2009-2011 Dominik Wagenfuehr <dominik.wagenfuehr@deesaster.org>
% Dieses Dokument unterliegt der Creative-Commons-Lizenz
% "Namensnennung-Weitergabe unter gleichen Bedingungen 3.0 Deutschland"
% [http://creativecommons.org/licenses/by-sa/3.0/de/].

\documentclass[12pt,parskip=half-]{scrartcl}

% Copyright 2009-2011 Dominik Wagenfuehr <dominik.wagenfuehr@deesaster.org>
% Dieses Dokument unterliegt der Creative-Commons-Lizenz
% "Namensnennung-Weitergabe unter gleichen Bedingungen 3.0 Deutschland"
% [http://creativecommons.org/licenses/by-sa/3.0/de/].

% Pakete
\usepackage[utf8]{inputenc}   % UTF8-Kodierung für Umlaute
\usepackage[T1]{fontenc}      % use TeX encoding then Type 1.
\usepackage{lmodern}          % Bessere Schriftart (ersetzt CM-Schriften)
\usepackage{ngerman}          % deutsche Silbentrennung
\usepackage{graphicx}         % Einbindung von Grafiken
\usepackage{fancyhdr}         % eigenes Layout einbinden
\usepackage{pdfpages}         % PDF-Seiten einbinden
\usepackage{xifthen}          % Wenn-dann-Abfragen
\usepackage{charter}          % Charter-Schrift
\usepackage{titlesec}         % Anpassung der Überschriften
\usepackage{longtable}        % Tabellen über Seitenumbruch hinweg
\usepackage{setspace}         % Zeilenabstand festlegen
\usepackage{hyperref}         % Hyperlinks und interne PDF-Verweise

% Layout
\hbadness=10000                % unterdrückt unwichtige Fehlermeldungen
\clubpenalty = 10000           % Keine "Schusterjungen"
\widowpenalty = 10000          % Keine "Hurenkinder"
\displaywidowpenalty = 10000

% Linie im Kopf ausblenden und im Fuß eine feine Linie
\renewcommand*{\headrulewidth}{0pt}
\renewcommand*{\footrulewidth}{0.4pt}

% Überschriftenlayout verändern
\titlespacing{\section}{0mm}{2em}{2em}
\titlespacing{\subsection}{0em}{0em}{0em}
\titleformat{\section}{\normalfont\LARGE\scshape}{}{0mm}{\hspace*{\fill}}
\titleformat{\subsection}{\normalfont\Large\bfseries\scshape}{}{0mm}{}[\vskip-1.5em\hrulefill]

% Eigene Fusszeile festlegen, Kopfzeile leeren
\pagestyle{fancy}
\fancyhead{}
\cfoot{\VollerName, \AbsenderStrasse, \AbsenderPLZOrt, \Telefon}

% Absender
\newcommand{\Absender}[1][\normalsize]{%
    \rightline{\textbf{#1\VollerName}}
    \rightline{\AbsenderStrasse}
    \rightline{\AbsenderPLZOrt}
    \rightline{\Telefon}
    \rightline{\EMail}
}

% Adressat
\newcommand{\Anschrift}{%
    \textbf{\Firma}\\
    \AnredeKopf{} \AdressatVorname{} \AdressatNachname\\
    \AnschriftStrasse\\
    \AnschriftPLZOrt\\
}

% Makro für Ort und Unterschrift
\newcommand*{\UnterschriftOrt}{%
    \Unterschrift\\
    \OrtDatum\\
}

% Unterschrift als Bilddatei einfügen
\newcommand*{\Unterschrift}[1][0.1cm]{%
    \hspace*{#1}\includegraphics{\UnterschriftenDatei}%
}

% Tabelle für Lebenslauf
\newlength{\AbstandAbschnitt}
\newboolean{UnterabschnittBegonnen}
\newenvironment{Abschnitt}[1][0em]
{%
    \setlength{\AbstandAbschnitt}{#1}%
    \begin{longtable}{p{0.3\linewidth}p{0.64\linewidth}}%
    \setboolean{UnterabschnittBegonnen}{false}%
}
{%
    \end{longtable}%
    \vspace{\AbstandAbschnitt}%
}

% Unterabschnitt in Lebenslauftabelle mit sonstigen Themen
% Achtung, keine Umgebung, sondern ein Kommando!
\newlength{\AbstandUnterabschnitt}
\newcommand{\Unterabschnitt}[3][0em]
{%
    \end{longtable}%
    \ifthenelse{\boolean{UnterabschnittBegonnen}}{%
        \setlength{\AbstandUnterabschnitt}{-5em}%
    }{%
        \setlength{\AbstandUnterabschnitt}{-2em}%
    }%
    \addtolength{\AbstandUnterabschnitt}{#1}%
    \vspace{\AbstandUnterabschnitt}%
    \textit{#2} #3%
    \setboolean{UnterabschnittBegonnen}{true}%
    \begin{longtable}{p{0.3\linewidth}p{0.64\linewidth}}%
    \\[1em]
}

% Anrede im Brief
\newcommand*{\Anrede}{Sehr geehrte\AnredeText{} \AdressatNachname,}

% Die Grußformel am Anschreibenende
\newcommand{\Grussworte}{%
Über die Einladung zu einem persönlichen Gespräch freue ich mich und verbleibe

mit freundlichen Grüßen

\Unterschrift\\[1em]
}

% Betreff
\newcommand*{\Betreff}{\textbf{Bewerbung als \Bewerberstelle}}

% Etwa Abstand vor dem Ort, um die Seite besser aufteilen zu können.
\newlength{\AbstandVorOrt}
\setlength{\AbstandVorOrt}{1em}

% Abstand zwischen den Adressen im Anschreiben.
\newlength{\AbstandZwischenAdressen}
\setlength{\AbstandZwischenAdressen}{0em}

\newcommand{\AnschreibenKopf}
{%
    % Anschreiben ohne Fuss- oder Kopfzeile
    \thispagestyle{empty}

    % Die Anschreibeseite wird bei Bedarf etwas vergrößert, da es hier keine
    % Fußzeile gibt.
    % \enlargethispage{3cm}
    
    \begin{spacing}{1.05}
        % Meine Anschrift
        \Absender

        % ggf. will man etwas Abstand zwischen den Adressen.
        \vspace*{\AbstandZwischenAdressen}

        % Anschrift der Firma
        \Anschrift

        % ggf. braucht man hier etwas Abstand, je nachdem, wie
        % viel Text man im Anschreiben hat.
        \vspace*{\AbstandVorOrt}

        % Ort und Datum
        \rightline{\OrtDatum}
    \end{spacing}

}

\newcommand{\Anschreiben}
{%
    \AnschreibenKopf
    \begin{spacing}{1.15}
        \Betreff
        \begin{flushleft}
            \Anrede

            \AnschreibenText

            \Grussworte

            \AnschreibenAnlage
        \end{flushleft}
    \end{spacing} 
}

\newcommand{\MeineSeite}
{%
    \section{Zu meiner Person}

    \begin{spacing}{1.3}
        \Absender[\large]
        \vskip2em

        \rightline{geboren am \Geburtstag}
        \rightline{in \Geburtsort}
        \vskip1em

        \rightline{\Details}

        \rule{10cm}{0.6pt} \\
        {\large \textbf{Ausbildungsgrad}}\\
        \Ausbildungsgrad\\

        \includegraphics[width=5cm]{\BewerberFoto}\\

    \end{spacing}
}

\newcommand{\MeineMotivation}
{%
    \section{Über mich und meine Motivation}
 
    \begin{spacing}{1.3}
        \begin{flushleft}

        \Motivationstext

        \end{flushleft}
    \end{spacing}
    
    \UnterschriftOrt
}


%%%%%%%%%%%%%%%%%%%%%%%%%%%%%%%%%%%%%%%%%%%%%%%%%%%
% Eigene Definitionen
% Ab hier sollte man selbst Änderungen vornehmen.
%%%%%%%%%%%%%%%%%%%%%%%%%%%%%%%%%%%%%%%%%%%%%%%%%%%

% Eigene Adresse festgelegt.
\newcommand*{\VollerName}{Max Mustermann}
\newcommand*{\AbsenderStrasse}{Musterstraße 42}
\newcommand*{\AbsenderPLZOrt}{12345 Berlin}
\newcommand*{\Telefon}{Tel.: 01234 -- 567890}
\newcommand*{\EMail}{mmustermann@musterstadt.de}
\newcommand*{\OrtDatum}{Berlin, \today}
\newcommand*{\Geburtstag}{1. Januar 1970}
\newcommand*{\Geburtsort}{Berlin}
\newcommand*{\Details}{ledig, ortsungebunden} % verheiratet, 2 Kinder, etc.
\newcommand*{\Ausbildungsgrad}{Promovierter Veterinärmediziner}

% Die Unterschrift sollte als Bilddatei vorliegen.
\newcommand*{\UnterschriftenDatei}{signatur-bewerber.png}

% Das Bewerberfoto.
\newcommand*{\BewerberFoto}{foto-bewerber.png}

% Adressat festlegen
\newcommand*{\Firma}{Hell AG}
\newcommand*{\AdressatVorname}{Ludwig-Christofer}
\newcommand*{\AdressatNachname}{Funkel}
\newcommand*{\AnschriftStrasse}{Route 66}
\newcommand*{\AnschriftPLZOrt}{00000 Havenfürst}

% Die beiden Anreden im Briefkopf und im Text müssen
% unterschieden werden.
\newcommand*{\AnredeKopf}{Herrn}   % alternativ: Frau
\newcommand*{\AnredeText}{r Herr}  % alternativ: Frau

% Die Stelle, auf die man sich bewirbt
\newcommand*{\Bewerberstelle}{Veterinär-Mediziner (m/w)}

% Der Anschreibentext.
\newcommand{\AnschreibenText}
{%
    {\itshape
        Hier steht nun der Bewerbungstext. Nicht zu lang, aber mit den
        wichtigen Details, wieso man sich bewirbt und in der Firma
        anfangen will.
        
        Den Hinweis hier natürlich löschen! Das heißt, alles, was in
        den geschweiften Klammern steht, kann weg.
    }
}

% Hinweis auf Anlagen
% Wenn nicht benötigt, dann einfach auskommentieren.
\newcommand{\AnschreibenAnlage}
{%
    Anlagen
}

% Text für die Motivationsseite
\newcommand{\Motivationstext}
{%
    {\itshape
        Hier steht nun der Motivationstext. Man sollte ein bisschen mehr
        über sich erzählen und wieso man in der Firma anfangen will.
        
        Dabei kann man etwas weiter ausholen, frühere Tätigkeiten
        beschreiben, aber auch auf private nützliche Dinge eingehen,
        die bei der Arbeit helfen können.
        
        In dem Beispiel hier zum Beispiel die Erfahrungen der
        Tierartzpraxis und die eigenen Haustiere.
        
        Den Hinweis hier natürlich löschen! Das heißt, alles, was in
        den geschweiften Klammern steht, kann weg.        
    }
}


% Titel und Autor des PDFs werden automatisch festgelegt,
% was aber erst nach der Definition des Namens geht.
\hypersetup{%
    pdftitle={Bewerbung bei \Firma},
    pdfauthor={\VollerName},
    pdfcreator={\VollerName}
}

\begin{document}

%%%%%%%%%%%%%%%%%%%%%%%%%%%
% Das Anschreiben
%%%%%%%%%%%%%%%%%%%%%%%%%%%

% ggf. braucht man zwischen Anschrift und text etwas Abstand,
% je nachdem, wie viel man im Anschreiben hat. (Standard ist 1em.)
% \setlength{\AbstandVorOrt}{2em}

% Normalerweise gibt es zwischen den zwei Adressen im Anschreiben
% keinen Abstand. Wer dennoch einen will, einfach den Abstand
% neu setzen. (Standard ist 0em.)
%\setlength{\AbstandZwischenAdressen}{1em}

\Anschreiben
\clearpage

%%%%%%%%%%%%%%%%%%%%%%%%%%%%%%%%%%%%%%%%%%%
% Meine Seite
%%%%%%%%%%%%%%%%%%%%%%%%%%%%%%%%%%%%%%%%%%%

\MeineSeite
\clearpage

%%%%%%%%%%%%%%%%%%%%%%%%%%%%%%%%%%%%%%%%%%%
% Lebenslauf
%%%%%%%%%%%%%%%%%%%%%%%%%%%%%%%%%%%%%%%%%%%

% Ab hier muss man selbst tätig werden und Inhalt einfüllen!

\section{Lebenslauf}

\subsection{Promotion}

\begin{Abschnitt}
Mai 2000 & Doktor der Veterinärmedizin Dr.\,med. \\
         & Doktorarbeit an der Freien Universität Berlin: \\
	     & "`\textit{Habitatsverhalten von Meerschweinchen unter Einfluss von Halluzinogenen}"'
\end{Abschnitt}

\subsection{Studium}

\begin{Abschnitt}
September 1997 & Abschluss als Diplom-Veterinärmediziner (Bewertung "`sehr gut"') \\
               & Diplomarbeit an der Freien Universität Berlin: \\
	           & "`\textit{Motorische Analyse der Klapperschlange -- Bewegung und Fortpflanzung}"'\\
\\
1991 -- 1997  & Studium Diplom-Veterinärmedizin\\
              & an der Freien Universität Berlin \\
              & Schwerpunkte: Säugende Reptilien
\end{Abschnitt}

\subsection{Zivildienst}

\begin{Abschnitt}
1989 -- 1991 & Zivildienst Malteser-Krankenhaus Berlin
\end{Abschnitt}

\subsection{Schulbildung}

\begin{Abschnitt}
Juni 1989    & Abiturprüfung bestanden mit Note 1,3 \\
1982 -- 1989 & Gymnasium Pestalozzischule Berlin
\end{Abschnitt}

% Neue Seite, die Überschrift muss jedesmal eingefügt werden!
%\clearpage
%\section{Lebenslauf}

\subsection{Veröffentlichungen}

\begin{Abschnitt}
1995 & "`\textit{Cutting Glasses with the Teeth of Crotalus adamanteus}"' für "`Ninth International Conference of World Wide Veterinary Medicine"' in Barcelona (Spanien), Januar 1996
\end{Abschnitt}

\subsection{Praktische Tätigkeiten}

\begin{Abschnitt}
1988 -- 1989 und & Tierarzt-Praxis Dr. Hauser in Berlin \\
\Unterabschnitt{Kernarbeitsgebiete:}{%
\begin{itemize}
\item Hunden die Pfötchen halten
\item Katzen kraulen
\end{itemize}
}
\Unterabschnitt{Weitere Aufgaben:}{Kaffee kochen}
1991 -- 1997     & Aushilfe und Betreuung der Patienten \\
\end{Abschnitt}

\subsection{Besondere Kenntnisse}

\begin{Abschnitt}
\textbf{Sprachen} & Englisch: sehr gut in Schrift und gut in Sprache \\
                  & Französisch: Grundkenntnisse \\
                  & Latein: Abschluss mit Latinum \\
\\
\textbf{Medizin} & Allgemeine Veterinärmedizin, \\
                 & Reptilien\\
\end{Abschnitt}

\subsection{Hobbys und Interessen}

\begin{Abschnitt}
 & Literatur \\
 & Badminton
\end{Abschnitt}

\UnterschriftOrt

\clearpage

%%%%%%%%%%%%%%%%%%%%%%%%%%%%%%%%%%%%%%%%%%%
% Motivation
%%%%%%%%%%%%%%%%%%%%%%%%%%%%%%%%%%%%%%%%%%%

\MeineMotivation
\clearpage

%%%%%%%%%%%%%%%%%%%%%%%%%%%%%%%%%%%%%%%%%%%
% Anlagen
% Wenn man keine Anlagen hat, dann alles
% auskommentieren oder löschen.
%%%%%%%%%%%%%%%%%%%%%%%%%%%%%%%%%%%%%%%%%%%

\section{Anlagenverzeichnis}

\begin{spacing}{1.1}
    \subsection{Arbeitszeugnisse}
    \begin{itemize}
        \item Tierarzt-Praxis Dr. Hauser
        \item Dienstzeugnis, Zivildienst Malteser-Krankenhaus Berlin
    \end{itemize}
    \vskip2em

    \subsection{Zeugnisse}
    \begin{itemize}
        \item Doktorurkunde
        \item Diplomurkunde
        \item Diplomzeugnis
        \item Abiturzeugnis
    \end{itemize}
\end{spacing}
\clearpage


% Hier folgen nun die einzelnen Anlagen, die als PDF vorliegen
% \includepdf[pages=-]{zeugnis.pdf}


\end{document}
