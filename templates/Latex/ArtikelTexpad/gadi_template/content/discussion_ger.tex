Monk ist eine erfolgreiche US-amerikanische Krimiserie. Hauptperson ist der neurotische Privatdetektiv Adrian Monk, der in San Francisco lebt.

\subsection{Überschrift 2. Ebene}
\label{subsec:ueberschrift_zweiter_ebene_01}
Adrian Monk, verkörpert von Tony Shalhoub, war früher Polizist beim Mordderzernat des San Francisco Police Department. Als seine Ehefrau Trudy Anne bei einem Anschlag ums Leben kam, verschlechterten sich seine zahlreichen ``Macken'' enorm und wurden zu einer psychischen Störung. Er verließ drei Jahre lang seine Wohnung nicht und wurde aus dem Polizeidienst entlassen.

Zu Beginn der Serienhandlung hat sich sein Zustand gebessert, allerdings bleiben zahlreiche Phobien (Angststörungen). Er benötigt daher ständig Hilfe durch einen persönlichen Assistenten; bis in die dritte Staffel ist dies Sharona Flemming (kongenial gespielt von Bitty Schramm), eine ehemalige Krankenschwester. Ihre Aufgabe ist es vor allem, alles von Monk fernzuhalten, was ihm Angst macht - und das ist eine Menge: Er leidet unter anderem an Angst vor Höhen (Akrophobie), Enge (Klaustrophobie), Dunkelheit (Achluophobie), Berührungen (Aphephosmophobie), Bakterien (Bacteriophobie) und vielem mehr.

Außerdem kann er Unordnung nicht ertragen; ein schief hängendes Bild muss sofort gerade gerückt, asymmetrisch angeordnete Gegenstände unverzüglich an den "richtigen" Platz gestellt werden, vorher ist Monk (Autor, 1801) nicht in der Lage, sich etwas anderem zuzuwenden, selbst wenn dadurch extreme Situationen entstehen.

Monk arbeitet als Privatdetektiv und freier Berater für die Polizei in schwierigen Fällen. Dabei hat er gerade durch seine Phobien einen guten Spürsinn für Dinge, die nicht in Ordnung sind. Meist wird er von seinem früheren Vorgesetzten, Captain Leland Stottlemeyer, beauftragt, der Monks unorthodoxer Methode skeptisch gegenübersteht, letztlich aber meist keine andere Wahl hat als sich auf Monk zu verlassen.

Monks Ziel ist die Wiederaufnahme in den Polizeidienst. In einer Folge nimmt er an einem Einstellungstest teil, scheitert jedoch tragisch am Ausfüllen eines Fragebogens

\subsection{Überschrift  2. Ebene}
\label{subsec:ueberschrift_zweiter_ebene_02}
Monk ist natürlich eine fiktive Filmfigur mit bunt zusammen gemischten Charaktermerkmalen und pathologischen Merkmalen aus vielen unterschiedlichen Bereichen. Neben den zahlreichen Phobien könnten seine Inselbegabungen z.B. auch als Asperger-Syndrom gedeutet werden; letztendlich entspricht seine Ansammlung von Symptomen jedoch keinem realen Krankheitsbild.

\subsubsection{Überschrift  3. Ebene}
\label{subsec:ueberschrift_dritter_ebene_01}
\begin{compactitem}
  \item Natalie Teeger (Traylor Howard) ist Monks Assistentin, nachdem die Darstellerin Sharonas, Bitty Schramm, die Serie verließ. In der Handlung wird das so gelöst, dass Sherona ihren Ex-Mann heiratet und nach New Jersey zieht. Natalie hat weniger Verständnis für Monks Phobien als Sharona, versucht sie ihm sogar oft „auszutreiben“, was stets in einem Fiasko endet. Sie hat eine Tochter, Julie (Emmy Clarke). 
  \item Lieutenant Randall Disher (Jason Gray-Stanford) ist Stottlemeyers Assistent; seine absurd-naiven Ideen und Theorien tragen zur Belustigung bei. 
  \item Dr. Charles Kroger (Stanley Kamel) ist Monks Psychiater; obwohl ihn sein Patient oft über Gebühr beansprucht, verliert er nie die Fassung. 
  \item Trudy Anne Monk (Stellina Rusich / Melora Hardin) ist Monks verstorbene Ehefrau. Sie wurde in einem Parkhaus von einer Autobombe getötet. In Folge 51 (Mr. Monk und Mrs. Monk) taucht sie vermeintlich wieder auf. Es handelt sich jedoch nur um eine Doppelgängerin, die obendrein am Ende der Episode ums Leben kommt. 
  \item Adrians Bruder Ambrose Monk (gespielt von John Turturro) ist ebenfalls hochgradig neurotisch. Er schreibt beruflich Gebrauchsanleitungen, überwiegend für Elektrogeräte, und beherrscht daher acht Sprachen, inklusive Hochchinesisch. Der Vater der beiden, Jake Monk (Dan Hedaya), verschwand, als beide noch jung waren. Seitdem wartet Ambrose auf ihn und hat das Haus nicht verlassen. Erst als er durch ein Feuer umgebracht werden soll, überwindet er diese Phobie und wird von seinem Bruder gerettet.
\end{compactitem}

Zitieren von Literatureinträgen funktioniert so: \cite{sample1}