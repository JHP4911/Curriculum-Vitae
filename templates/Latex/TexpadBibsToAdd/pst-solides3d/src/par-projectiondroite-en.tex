\section{Lines}

\subsection{Direct definition}

The object \texttt{droite} allows us to define and draw a \Index{line}. In
the \texttt{pst-solides3d} package, a line in 2D is defined by its
two end-points.

We use the option \Lkeyword{args} to specify the end-points of the
chosen line. We can use coordinates or named points.

As with points and vectors, we can save the coordinates of the
line with the option \Lkeyword{name}.

\begin{LTXexample}[width=7.5cm]
\begin{pspicture}(-3,-3)(4,3.5)%
\psframe*[linecolor=blue!50](-3,-3)(4,3.5)
\psset{viewpoint=50 30 15,Decran=60}
\psset{solidmemory}
%% definition du plan de projection
\psSolid[object=plan,
   definition=equation,
   args={[1 0 0 0] 90},
   planmarks,name=monplan]
\psset{plan=monplan}
%% definition du point A
\psProjection[object=point,
   name=A,text=A,
   pos=ur](-2,1.25)
\psProjection[object=point,
   name=B,text=B,
   pos=ur](1,.75)
\psProjection[object=droite,
   linecolor=blue,
   args=0 0 1 .5]
\psProjection[object=droite,
   linecolor=orange,
   args=A B]
\composeSolid
\end{pspicture}
\end{LTXexample}


\subsection{Some other definitions}

There are other methods to define a line in 2D. The options
\Lkeyword{definition} and \Lkeyword{args} are used in these variants:



\begin{itemize}

\item \texttt{\Lkeyword{definition}=\Lkeyval{horizontale}};
\texttt{\Lkeyword{args}=$b$}.

The line with equation $y=b$.

\item \texttt{\Lkeyword{definition}=\Lkeyval{verticale}};
\texttt{\Lkeyword{args}=$a$}.

The line with equation $x=a$.

\item \texttt{\Lkeyword{definition}=\Lkeyval{paral}};
\texttt{\Lkeyword{args}=$d$ $A$}.

A line parallel to $d$ passing through
$A$.

\item \texttt{\Lkeyword{definition}=\Lkeyword{perp}};
\texttt{\Lkeyword{args}=$d$ $A$}.

A line perpendicular to $d$ passing
through $A$.

\item \texttt{\Lkeyword{definition}=\Lkeyval{mediatrice}};
\texttt{\Lkeyword{args}=$A$ $B$}.

The perpendicular bisector of the line
segment $[AB]$.

\item \texttt{\Lkeyword{definition}=\Lkeyword{bissectrice}};
\texttt{\Lkeyword{args}=$A$ $B$ $C$}.

The bisector of the angle $\widehat
{ABC}$.

\item \texttt{\Lkeyword{definition}=\Lkeyword{axesymdroite}};
\texttt{\Lkeyword{args}=$d$ $D$}.

The reflection of the line $d$ in the
line $D$.

\item \texttt{\Lkeyword{definition}=\Lkeyword{rotatedroite}};
\texttt{\Lkeyword{args}=$d$ $I$ $r$}.

The image of the line $d$ after a
rotation with centre $I$ through an angle $r$ (in degrees)

\item \texttt{\Lkeyword{definition}=\Lkeyword{translatedroite}};
\texttt{\Lkeyword{args}=$d$ $u$}.

The image of the line $d$ shifted by the vector $\vec u$.

\end{itemize}

\endinput
