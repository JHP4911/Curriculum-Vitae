\section {The predefined solids and their parameters}

The basic command is:~
\texttt{\Lcs{psSolid}[object=\textsl{name}]$(x, y ,z)$} which allows us to translate the chosen object to the point with the coordinates $(x, y,
z)$.

The available predefined names for the objects are:
\begin{sloppypar}
{\ttfamily%\flushleft \hyphenchar\font`\-%
point, line, vector, plan, grille, cube, cylindre, cylindrecreux, cone, conecreux, tronccone,
troncconecreux, sphere, calottesphere, calottespherecreuse, tore,
tetrahedron, octahedron, dodecahedron,
isocahedron, anneau, prisme, prismecreux, parallelepiped, face, polygonregulier, ruban, surface, surface*, surfaceparamettree, pie, fusion, geode, load, offfile, objfile, datfile, new.}
\end{sloppypar}


The following table gives an example of every one of the above named solids with their specified parameters:

\begin{center}
\begin{tabular}{>{\bfseries\sffamily\color{blue}}lcm{4cm}m{5cm}}
  \hline
\toptableau
\\\hline
 \Index{Point}&
    \begin{tabular}{c}
    \texttt{[args=1 1 0]}\\
     coordinates
    \end{tabular}
    &
 \begin{pspicture}(-2,-2)(2,2)
\psset{lightsrc=10 5 20,viewpoint=50 20 30 rtp2xyz}
\psSolid[object=point,args=1 1 0]%
\axesIIID(1.5,1.5,1)
\end{pspicture}
 &
 \begin{minipage}{5cm}
 \begin{verbatim}
\psSolid[object=point,
args=1 1 0]%
\end{verbatim}
\end{minipage}
\\\hline
 \Index{Line}&
    \begin{tabular}{c}
    \texttt{[args=0 -1 0 1 2 2]}\\
     coordinates of the\\
     end points
    \end{tabular}
    &
 \begin{pspicture}(-2,-2)(2,2)
\psset{lightsrc=10 5 20,viewpoint=50 20 30 rtp2xyz}
\psSolid[object=line,args=0 -1 0 1 2 2]
\axesIIID(1.5,1.5,1)
\end{pspicture}
 &
\begin{minipage}{5cm}
\begin{verbatim}
\psSolid[object=line,
args=0 -1 0 1 2 2]
\end{verbatim}
\end{minipage}
\\\hline
 \Index{Vector}&
    \begin{tabular}{c}
    \texttt{[args=1 2 2]}\\
     components of\\
     the vector
    \end{tabular}
    &
\begin{pspicture}(-2,-2)(2,2)
\psset{lightsrc=10 5 20,viewpoint=50 20 30 rtp2xyz}
\psSolid[object=vecteur,args=1 2 2]
\axesIIID(1.5,1.5,1)
\end{pspicture}
 &
\begin{minipage}{5cm}
\begin{verbatim}
\psSolid[object=vecteur,
args=1 2 2]
\end{verbatim}
\end{minipage}
\\\hline
 \Index{Plane}&
    \begin{tabular}{c}
    \texttt{[base=-x x -y y]}\\
     range of plane\\
     \texttt{args={[0 0 1 0]}}\\
     equation of plane
    \end{tabular}
    &
 \begin{pspicture}(-2,-2)(2,2)
\psset{lightsrc=10 5 20,viewpoint=50 20 30 rtp2xyz}
\psSolid[object=plan,
   definition=equation,
   args={[0 0 1 0]},
   base=-1 1 -1.5 1.5]
\axesIIID(1.5,1.5,1)
\end{pspicture}
 &
\begin{minipage}{5cm}
\begin{verbatim}
\psSolid[object=plan,
definition=equation,
args={[0 0 1 0]},
base=-1 1 -1.5 1.5]
\end{verbatim}
\end{minipage}
\\\hline

\end{tabular}
\end{center}

\begin{center}
\begin{tabular}{>{\bfseries\sffamily\color{blue}}lcm{4cm}m{5cm}}
  \hline
\toptableau
\\\hline
 \Index{Cube}& \begin{tabular}{c}
    \texttt{[a=4]}\\
     edge's length
    \end{tabular}
    &
 \begin{pspicture}(-2,-2)(2,2)
\psset{lightsrc=10 20 30,viewpoint=50 20 30 rtp2xyz}
 \psset{Decran=60}
 \psSolid[
 object=cube,a=2,action=draw*,fillcolor=magenta!20]%
 \axesIIID(1,1,1)(1.5,1.5,1.5)
 \end{pspicture}
 &
 \begin{minipage}{5cm}
 \begin{verbatim}
\psSolid[
   object=cube,
   a=2,
   action=draw*,
   fillcolor=magenta!20]
 \end{verbatim}
 \end{minipage}
\\\hline
 \Index{Cylinder}&
    \begin{tabular}{c}
     \texttt{[h=6,r=2]}\\
     height and radius\\
     grid:\\
     \texttt{[ngrid=n1 n2]}
     \end{tabular}
     &
 \begin{pspicture}(-2,-2.5)(2,3)
\psset{lightsrc=10 20 30,viewpoint=50 20 30 rtp2xyz}
 \psset{Decran=30}
\psSolid[object=cylindre,h=5,r=2,fillcolor=white,ngrid=4 32](0,0,-3)
 \axesIIID(2,2,2.5)(3,3,3.5)
 \end{pspicture}
 &
 \begin{minipage}{5cm}
 \begin{verbatim}
\psSolid[
  object=cylindre,
  h=5,r=2,
  fillcolor=white,
  ngrid=4 32]
  (0,0,-3)
 \end{verbatim}
 \end{minipage}
\\\hline
 \Index{Hollow Cylinder}&
    \begin{tabular}{c}
     \texttt{[h=6,r=2]}\\
     height and radius\\
     grid:\\
     \texttt{[ngrid=n1 n2]}
     \end{tabular}
     &
 \begin{pspicture}(-2,-2.5)(2,3)
\psset{lightsrc=10 20 30,viewpoint=50 20 30 rtp2xyz}
 \psset{Decran=30}
\psSolid[object=cylindrecreux,h=5,r=2,fillcolor=white,mode=4,incolor=green!50](0,0,-2.5)
 \axesIIID(2,2,2.5)(3,3,4.5)
 \end{pspicture}
 &
 \begin{minipage}{5cm}
 \begin{verbatim}
\psSolid[
   object=cylindrecreux,
   h=5,r=2,
   fillcolor=white,
   mode=4,
   incolor=green!50]
   (0,0,-3)
 \end{verbatim}
 \end{minipage}
 \\\hline
\end{tabular}
\end{center}

\begin{center}
\begin{tabular}{>{\bfseries\sffamily\color{blue}}lcm{4cm}m{5cm}}
  \hline
\toptableau
\\\hline
 \Index{Cone}&
     \begin{tabular}{c}
     \texttt{[h=6,r=2]}\\
     height and radius\\
     grid:\\
     \texttt{[ngrid=n1 n2]}
     \end{tabular}
     &
 \begin{pspicture}(-2,-1)(2,4)
\psset{lightsrc=10 20 30,viewpoint=50 20 30 rtp2xyz}
 \psset{Decran=30}
 \psSolid[object=cone,h=5,r=2,fillcolor=cyan,mode=4]%
 \axesIIID(2,2,5)(2.5,2.5,6)
 \end{pspicture}
 &
 \begin{minipage}{5cm}
 \begin{verbatim}
 \psSolid[
    object=cone,
    h=5,r=2,
    fillcolor=cyan,
    mode=4]%
 \end{verbatim}
 \end{minipage}
\\\hline
 \Index{Hollow Cone}&
     \begin{tabular}{c}
     \texttt{[h=6,r=2]}\\
     height and radius\\
     grid:\\
     \texttt{[ngrid=n1 n2]}
     \end{tabular}
     &
 \begin{pspicture}(-2,-1)(2,4)
\psset{lightsrc=10 20 30,viewpoint=50 20 30 rtp2xyz}
 \psset{Decran=30}
 \psSolid[object=conecreux,h=5,r=2,fillcolor=white,mode=4,RotY=-60,incolor=green!50]%
 \axesIIID(2,2,5)(2.5,2.5,6)
 \end{pspicture}
 &
 \begin{minipage}{5cm}
 \begin{verbatim}
 \psSolid[
    object=conecreux,
    h=5,r=2,
    RotY=-60,
    fillcolor=white,
    incolor=green!50,
    mode=4]%
 \end{verbatim}
 \end{minipage}
 \\\hline
 \Index{Truncated Cone}&
     \begin{tabular}{c}
     \texttt{[h=6,r0=4,r1=1.5]}\\
     height and radii\\
     grid:\\
     \texttt{[ngrid=n1 n2]}
     \end{tabular}
     &
 \begin{pspicture}(-2,-1)(2,4)
\psset{lightsrc=10 20 30,viewpoint=50 20 30 rtp2xyz}
 \psset{Decran=30}
 \psSolid[object=tronccone,r0=2,r1=1.5,h=5,fillcolor=cyan,mode=4]%
 \axesIIID(2,2,5)(2.5,2.5,6)
 \end{pspicture}
 &
 \begin{minipage}{5cm}
 \begin{verbatim}
 \psSolid[
    object=tronccone,
    r0=2,r1=1.5,h=5,
    fillcolor=cyan,
    mode=4]%
 \end{verbatim}
 \end{minipage}
\\\hline
     \begin{tabular}{c}
     Truncated \\
     Hollow Cone
     \end{tabular}
     &
     \begin{tabular}{c}
     \texttt{[h=6,r0=4,r1=1.5]}\\
     height and radii\\
     grid:\\
     \texttt{[ngrid=n1 n2]}
     \end{tabular}
     &
 \begin{pspicture}(-2,-1)(2,4)
\psset{lightsrc=10 20 30,viewpoint=50 20 30 rtp2xyz}
 \psset{Decran=30}
 \psSolid[object=troncconecreux,r0=2,r1=1,h=5,fillcolor=white,mode=4]%
 \axesIIID(2,2,5)(2.5,2.5,6)
 \end{pspicture}
 &
 \begin{minipage}{5cm}
 \begin{verbatim}
\psSolid[
   object=troncconecreux,
   r0=2,r1=1,h=5,
   fillcolor=white,
   mode=4]%
 \end{verbatim}
 \end{minipage}
\\\hline
\end{tabular}
\end{center}

%\newpage

%%%%%%%%%%%%%%%%%%%%%%%%%%%%%%%%%%%%%%%%%%%%%%%%%%
\begin{center}
%\begin{tabular}{>{\bfseries\sffamily\color{blue}}lcm{4cm}m{5cm}}
\begin{tabular}{
   >{\bfseries\sffamily\color{blue}} l
   >{\centering} m{4cm} m{4cm} m{5cm}}
  \hline
\toptableau
\\\hline
 \Index{Sphere} &
     \begin{tabular}{c}
     \texttt{[r=2]}\\
     radius\\
     grid:\\
     \texttt{[ngrid=n1 n2]}
    \end{tabular}
     &
 \begin{pspicture}(-2,-2)(2,3)
\psset{lightsrc=10 20 30,viewpoint=50 20 30 rtp2xyz}
 \psset{Decran=30}
 \psSolid[object=sphere,r=3,fillcolor=red!25,ngrid=18 18,linewidth=0.2\pslinewidth]%
 \axesIIID(3,3,3)(4,4,4)
 \end{pspicture}
 &
 \begin{minipage}{5cm}
 \begin{verbatim}
 \psSolid[
    object=sphere,
    r=2,fillcolor=red!25,
    ngrid=18 18]%
 \end{verbatim}
 \end{minipage}
\\\hline
     \begin{tabular}{c}
     Spherical \\
     zone
     \end{tabular} &
     \begin{tabular}{c}
     \texttt{[r=2]} \\
     radius\\
     \texttt{[phi=0,theta=90]} \\
     latitude for slicing\\
     the zone respectively \\
     from the bottom and top \\
    \end{tabular}
     &
\begin{pspicture}(-2,-3)(5,3)
\psset{unit=0.5cm}
\psset{lightsrc=42 24 13,viewpoint=50 30 15 rtp2xyz,Decran=50}
\psSolid[object=calottesphere,r=3,ngrid=16 18,
   fillcolor=cyan!50,incolor=yellow,theta=45,phi=-30,hollow,RotY=-80]%
\axesIIID(0,3,3)(6,5,4)
\end{pspicture}
 &
 \begin{minipage}{5cm}
 \begin{verbatim}
\psSolid[
   object=calottesphere,
   r=3,ngrid=16 18,
   theta=45,phi=-30,
   hollow,RotY=-80]%
 \end{verbatim}
 \end{minipage}
\\\hline
 \Index{Torus} &
     \begin{tabular}{c}
     \texttt{[r0=4,r1=1.5]} \\
     inner radius\\
     mean radius\\
     grid:\\
     \texttt{[ngrid=n1 n2]}
     \end{tabular}
     &
 \begin{pspicture}(-2,-2)(2,2.35)
\psset{lightsrc=42 24 13,viewpoint=50 30 15 rtp2xyz}
 \psset{Decran=30,unit=0.9cm}
 \psSolid[r1=2.5,r0=1.5,object=tore,ngrid=18 36,fillcolor=green!30,action=draw**]%
  \axesIIID(4,4,0)(5,5,4)
 \end{pspicture}
 &
 \begin{minipage}{5cm}
 \begin{verbatim}
 \psSolid[
    r1=2.5,r0=1.5,
    object=tore,
    ngrid=18 36,
    fillcolor=green!30,
    action=draw*]%
 \end{verbatim}
 \end{minipage}
\\\hline
     \begin{tabular}{c}
     Cylindric \\
     Ring
     \end{tabular}
      &
     \begin{tabular}{c}
     \texttt{[R=4,r=3}\\
     inner and outer radius\\
     \texttt{h=6,section=...]}\\
     height\\
     cross \\
     section
     \end{tabular}
     &
 \begin{pspicture}(-2,-2)(2,2.35)
%\psset{unit=0.44cm}
\psset{lightsrc=42 24 13,viewpoint=50 30 15 rtp2xyz}
 \psset{Decran=30}
\psSolid[object=anneau,fillcolor=yellow,h=1.5,R=4,r=3]%
 \axesIIID(4,4,0)(5,5,4)
 \end{pspicture}
 &
 \begin{minipage}{5cm}
 \begin{verbatim}
 \psSolid[
    object=anneau,
    fillcolor=yellow,
    h=1.5,R=4,r=3]%
 \end{verbatim}
 \end{minipage}
\\\hline
\end{tabular}
\end{center}


\begin{center}
%\begin{tabular}{>{\bfseries\sffamily\color{blue}}lcm{4cm}m{6cm}}
\begin{tabular}{
   >{\bfseries\sffamily\color{blue}} l
   >{\centering} m{4cm} m{4cm} m{5cm}}
  \hline
\toptableau
\\\hline
 \Index{Tetrahedron}&
     \begin{tabular}{c}
     \texttt{[r=2]}\\
     radius of the\\
     circumscribed sphere
     \end{tabular}
     &
 \begin{pspicture}(-2,-2)(2,2)
\psset{lightsrc=10 20 30,viewpoint=50 20 30 rtp2xyz}
 \psset{Decran=30}
 \psSolid[object=tetrahedron,r=3,linecolor=blue,action=draw]%
 \end{pspicture}
 &
 \begin{minipage}{5cm}
 \begin{verbatim}
\psSolid[
   object=tetrahedron,
   r=3,
   linecolor=blue,
   action=draw]%
 \end{verbatim}
 \end{minipage}
\\\hline
\Index{Octahedron} &
     \begin{tabular}{c}
     \texttt{[a=2]}\\
     radius of the\\
     circumscribed sphere
     \end{tabular}
     &
 \begin{pspicture}(-2,-1.85)(2,2.85)
\psset{lightsrc=10 20 30,viewpoint=50 20 30 rtp2xyz}
 \psset{Decran=30}
 \psSolid[object=octahedron,a=3,linecolor=blue,fillcolor=Turquoise]%
 \axesIIID(3,3,3)(4,4,4)
 \end{pspicture}
 &
 \begin{minipage}{5cm}
 \begin{verbatim}
 \psSolid[
    object=octahedron,
    a=3,
    linecolor=blue,
    fillcolor=Turquoise]%
 \end{verbatim}
 \end{minipage}
\\\hline
 \Index{Dodecahedron} &
     \begin{tabular}{c}
     \texttt{[a=2]}\\
     radius of the\\
     circumscribed sphere
     \end{tabular}
     &
 \begin{pspicture}(-2,-1.85)(2,1.85)
\psset{lightsrc=10 20 30,viewpoint=50 20 30 rtp2xyz}
 \psset{Decran=30}
 \psSolid[object=dodecahedron,a=2.5,RotZ=90,action=draw*,fillcolor=OliveGreen]%
 \end{pspicture}
 &
 \begin{minipage}{5cm}
 \begin{verbatim}
 \psSolid[
    object=dodecahedron,
    a=2.5,RotZ=90,
    action=draw*,
    fillcolor=OliveGreen]%
 \end{verbatim}
 \end{minipage}
\\ \hline
\Index{Icosahedron} &
     \begin{tabular}{c}
     \texttt{[a=2]}\\
     radius of the\\
     circumscribed sphere
     \end{tabular}
     &
 \begin{pspicture}(-2,-1.85)(2,2.85)
\psset{lightsrc=10 20 30,viewpoint=50 20 30 rtp2xyz}
 \psset{Decran=30}
 \psSolid[object=icosahedron,a=3,action=draw*,fillcolor=green!50]%
 \axesIIID(3,3,3)(4,4,4)
 \end{pspicture}
 &
 \begin{minipage}{5cm}
 \begin{verbatim}
\psSolid[
   object=icosahedron,
   a=3,
   action=draw*,
   fillcolor=green!50]%
 \end{verbatim}
 \end{minipage}
\\\hline
     \Index{Prism}
      &
     \begin{tabular}{c}
     \texttt{[axe=0 0 1]}\\
     direction of the axis\\
     \texttt{[base=}\\
     \texttt{-1 -1 1 -1 0 1]}\\
     coordinates of\\
     the vertices\\
     of the base\\
     \texttt{[h=6]}\\
     height
     \end{tabular}
     &
 \begin{pspicture}(-2,-2)(2,3)
\psset{lightsrc=10 20 30,viewpoint=50 20 30 rtp2xyz}
 \psset{Decran=30,unit=0.9cm}
\psSolid[object=prisme,action=draw*,linecolor=red,h=4,fillcolor=gray!50]%
\psSolid[object=grille,base=-3 3 -3 3,action=draw]%
 \axesIIID(3,3,4)(5,5,5)
 \end{pspicture}
 &
 \begin{minipage}{5cm}
 \begin{verbatim}
\psSolid[
   object=prisme,
   action=draw*,
   linecolor=red,
   h=4]%
 \end{verbatim}
 \end{minipage}
 \\\hline
\end{tabular}
\end{center}

%\newpage
%%%%%%%%%%%%%%%%%%%%%%%%%%%%%%%%%%%%%%%%%%%%%%%%%%
\begin{center}
%\psset{lightsrc=10 20 30,viewpoint=50 20 30 rtp2xyz}
%\begin{tabular}{>{\bfseries\sffamily\color{blue}}lcm{4cm}m{6cm}}
\begin{tabular}{
   >{\bfseries\sffamily\color{blue}} l
   >{\centering} m{4cm} m{4cm} m{5cm}}
  \hline
\toptableau
\\\hline
     \Index{Grid}
      &
     \begin{tabular}{c}
     \texttt{[base=-X +X -Y +Y]}
     \end{tabular}
     &
 \begin{pspicture}(-1.5,-2)(2,3)
\psset{lightsrc=10 20 30,viewpoint=50 20 30 rtp2xyz}
 \psset{Decran=30,unit=0.9cm}
\psSolid[object=grille,base=-5 5 -3 3]%
 \axesIIID(5,3,0)(6,4,4)
 \end{pspicture}
 &
 \begin{minipage}{5cm}
 \begin{verbatim}
\psSolid[
   object=grille,
   base=-5 5 -3 3]%
 \end{verbatim}
 \end{minipage}
\\\hline
%
   \Index{Cuboid}
      &
     \begin{tabular}{c}
     \texttt{[a=4,b=3,c=2]}\\
     edge lenghts\\
     with center in $O$
     \end{tabular}
     &
 \begin{pspicture}(-1.5,-2)(2,3)
 \psset{lightsrc=10 20 30,viewpoint=50 20 30 rtp2xyz}
 \psset{Decran=30}
\psSolid[object=parallelepiped,a=5,b=6,c=2,fillcolor=bleuciel,axe=1 1 1](0,0,c 2 div)
\psSolid[object=grille,base=-2.5 2.5 -3 3,action=draw](0,0,2)
\psSolid[object=grille,base=-1 1 -3 3,RotY=90,action=draw](2.5,0,1)
\psSolid[object=grille,base=-2.5 2.5 -1 1,RotX=-90,action=draw](0,3,1)
 \axesIIID(2.5,3,2)(3.5,4,4)
 \end{pspicture}
 &
 \begin{minipage}{5cm}
 \begin{verbatim}
\psSolid[
   object=parallelepiped,%
   a=5,b=6,c=2,
   fillcolor=yellow]%
   (0,0,c 2 div)
 \end{verbatim}
 \end{minipage}
\\\hline
%
   \Index{Face}
      &
     \begin{tabular}{l}
     \texttt{[base=x0 y0 x1 y1}\\
     \texttt{~     x2 y2 etc.]}\\
     the coordinates \\
     of the vertices
     \end{tabular}
     &
\begin{pspicture}(-2,-2)(3,2)
\psset{unit=0.4cm}
\psset{viewpoint=50 -20 30 rtp2xyz,Decran=50}
\psSolid[object=grille,base=-4 6 -4 4,action=draw,linecolor=gray](0,0,0)
\psSolid[object=face,fillcolor=yellow,
      incolor=blue,
      base=0 0 3 0 1.5 3
      ](0,1,0)
\psSolid[object=face,fillcolor=yellow,
      incolor=blue,
      base=0 0 3 0 1.5 3,
      RotX=180](0,-1,0)
\axesIIID(0,0,0)(6,6,3)
\end{pspicture}
 &
 \begin{minipage}{5cm}
 \begin{verbatim}
\psSolid[
   object=face,
   fillcolor=yellow,
   incolor=blue,
   base=0 0 3 0 1.5 3
   ](0,1,0)
\psSolid[
   object=face,
   fillcolor=yellow,
   incolor=blue,
   base=0 0 3 0 1.5 3,
   RotX=180](0,-1,0)
 \end{verbatim}
 \end{minipage}
\\\hline
%
   \Index{Strip}
      &
     \begin{tabular}{l}
     \texttt{[base=x0 y0 x1 y1}\\
     \texttt{~     x2 y2 etc.]}\\
     \texttt{[h=height]}\\
     \texttt{[ngrid=value]}\\
     number of gridlines\\
     \texttt{[axe=0 0 1]}\\
     direction of inclination\\
     of the strip
     \end{tabular}
     &
\begin{pspicture}(-2,-2)(5,3)
\psset{lightsrc=10 0 10,viewpoint=50 -20 30 rtp2xyz,Decran=50,unit=0.5cm}
\psSolid[object=grille,base=-4 6 -2 4,action=draw,linecolor=gray](0,0,0)
\psSolid[object=ruban,h=3,fillcolor=red!50,
      base=0 0 2 2 4 0 6 2,
      num=0 1 2 3,
      show=0 1 2 3,
      ngrid=3]%
\axesIIID(0,2,0)(6,6,6)
\end{pspicture}
 &
 \begin{minipage}{5cm}
 \begin{verbatim}
\psSolid[
   object=ruban,h=3,
   fillcolor=red!50,
   base=0 0 2 2 4 0 6 2,
   num=0 1 2 3,
   show=0 1 2 3,
   ngrid=3])
 \end{verbatim}
 \end{minipage}
\\\hline
\end{tabular}
\end{center}

%\newpage

%\begin{center}
%\psset{lightsrc=10 20 30,SphericalCoor,viewpoint=50 20 30}
%%\begin{tabular}{>{\bfseries\sffamily\color{blue}}lcm{4cm}m{6cm}}
%\begin{tabular}{
%   >{\bfseries\sffamily\color{blue}} l
%   >{\centering} m{4cm} m{4cm} m{5cm}}
%  \hline
%\toptableau
%%    chemin
%%       &
%%      \begin{tabular}{l}
%%      dessine un chemin\\
%%      d\'{e}fini en postscript\\
%%      sur un plan
%%      \end{tabular}
%%      &
%% \psset{unit=0.4cm}
%% \begin{pspicture}(-2,-5)(6,8)%
%% \psframe*[linecolor=blue!50](-6,-5)(6,7)
%% \psset{lightsrc=50 20 20,viewpoint=50 30 15,Decran=60}
%% \psProjection[object=chemin,fillstyle=solid,fillcolor=white,
%%             linewidth=.05,linecolor=red,
%%             normal=1 1 2 180,
%%             path=newpath
%%                 -4 -4 smoveto
%%                 -4 4 slineto
%%                 4 4 slineto
%%                 4 -4 slineto
%%                 closepath
%%             ](1,1,2)
%% \psProjection[object=chemin,
%%       linewidth=.02,
%%       normal=1 1 2 180,
%%       path=newpath
%%           -4 1 4
%%           {-4 exch smoveto
%%            8 0 srlineto} for
%%            -4 1 4
%%           {-4 smoveto
%%            0 8 srlineto} for
%%             ](1,1,2)
%% \psProjection[object=chemin,fillstyle=hlines,hatchcolor=yellow,
%%             linecolor=red,
%%             normal=1 1 2 180,
%%             path=newpath
%%             2 0 moveto
%%             0 2 360 {
%%                 /x exch def
%%                 x cos 2 mul
%%              x sin 2 mul
%%                 slineto
%%          } for
%%             ](1,1,2)
%% \psPoint(0,0,0){O}
%% \psPoint(1,1,2){O1}\psPoint(1.4,1.4,2.8){K}
%% \psline[linewidth=.1,linecolor=red](O1)(K)
%% \psline[linestyle=dashed](O)(O1)
%% \psProjection[object=chemin,
%%       linewidth=.1,
%%       linecolor=green,
%%       normal=1 1 2 180,
%%       path=
%%          newpath
%%             0 0 smoveto
%%             1 0 slineto](1,1,2)
%% \psProjection[object=chemin,
%%       linewidth=.1,
%%       linecolor=blue,
%%       normal=1 1 2 180,
%%       path=
%%          newpath
%%             0 0 smoveto
%%             0 1 slineto](1,1,2)
%% \axesIIID(4,4,2)(5,5,6)
%% \end{pspicture}
%%  &
%%  \begin{minipage}{6cm}
%%  \begin{verbatim}
%%     \psProjection[object=chemin,
%%     fillstyle=hlines,
%%     hatchcolor=yellow,
%%     linecolor=red,
%%     normal=1 1 2 180,
%%     path=newpath
%%     2 0 smoveto
%%     0 2 360 {
%%      /x exch def
%%      x cos 2 mul
%%      x sin 2 mul
%%      slineto
%%     } for
%%     ](1,1,2)
%%  \end{verbatim}
%%  \end{minipage}
%\end{tabular}
%\end{center}

%\newpage
%%%%%%%%%%%%%%%%%%%%%%%%%%%%%%%%%%%%%%%%%%%%%%%%%%
\begin{center}
%\begin{tabular}{>{\bfseries\sffamily\color{blue}}lcm{4cm}m{6cm}}
\begin{tabular}{
   >{\bfseries\sffamily\color{blue}} l
   >{\centering} m{4cm} m{4cm} m{5cm}}
  \hline
\toptableau
\\\hline
   \Index{Surface}
      &
     \begin{tabular}{l}
     see the related \\
     paragraph in the \\
     documentation
     \end{tabular}
     &
\begin{pspicture}(-2,-3)(3,3)
\psset{unit=0.4cm,lightsrc=30 30 25,viewpoint=50 40 30 rtp2xyz,Decran=50}
\psSurface[ngrid=.25 .25,incolor=white,axesboxed](-4,-4)(4,4){%
   x dup mul y dup mul 3 mul sub x mul 32 div}
\end{pspicture}
 &
 \begin{minipage}{5cm}
 \begin{verbatim}
\psSurface[ngrid=.25 .25,
  incolor=white,axesboxed]
  (-4,-4)(4,4){%
  x dup mul y dup mul 3 mul
  sub x mul 32 div}
 \end{verbatim}
 \end{minipage}
\\\hline
%
   \Index{New}
      &
     \begin{tabular}{l}
     solid defined\\
     by the coordinates \\
     of the vertices\\
     and the vertices\\
     of each face
     \end{tabular}
     &

\begin{pspicture}(-2,-2)(2,4)
\psset{unit=0.4cm}
\psset{viewpoint=50 -20 30 rtp2xyz,Decran=50}
\psSolid[object=new,
         action=draw,
         sommets=
         2  4  3
        -2  4  3
        -2 -4  3
         2 -4  3
         2  4  0
        -2  4  0
        -2 -4  0
         2 -4  0
         0  4  5
         0 -4  5,
    faces={
        [0 1 2 3]
        [7 6 5 4]
        [0 3 7 4]
        [3 9 2]
        [1 8 0]
        [8 9 3 0]
        [9 8 1 2]
        [6 7 3 2]
        [2 1 5 6]},
        num=all,
      show=all]%
\axesIIID(0,0,0)(5,5,7)
\end{pspicture}
 &
 \begin{minipage}{5cm}
 \begin{verbatim}
 \psSolid[object=new,
         action=draw,
         sommets=
         2  4  3
        -2  4  3
        -2 -4  3
         2 -4  3
         2  4  0
        -2  4  0
        -2 -4  0
         2 -4  0
         0  4  5
         0 -4  5,
    faces={
        [0 1 2 3]
        [7 6 5 4]
        [0 3 7 4]
        [3 9 2]
        [1 8 0]
        [8 9 3 0]
        [9 8 1 2]
        [6 7 3 2]
        [2 1 5 6]}]%
 \end{verbatim}
 \end{minipage}
\\\hline
%
   \Index{Curve}
      &
     \begin{tabular}{l}
     curve of a function\\
     $\mathbb{R} \rightarrow \mathbb{R}^3$\\
     defined by its\\
     paramterised equations\\
     \end{tabular}
     &

\begin{pspicture}(-2,-1)(1.75,2.7)
\psset{unit=0.35cm}
\psset{lightsrc=10 -20 50,viewpoint=50 -20 20 rtp2xyz,Decran=50}
%\psframe*[linecolor=blue!50](-6,-3)(6,8)
\psSolid[object=grille,base=-4 4 -4 4,linecolor=red,linewidth=0.5\pslinewidth]%
\axesIIID(0,0,0)(4,4,7)
\defFunction[algebraic]{helice}(t){3*cos(4*t)}{3*sin(4*t)}{t}
\psSolid[object=courbe,r=0,
        range=0 6,
        linecolor=blue,linewidth=0.1,
        resolution=360,
        function=helice]%
\end{pspicture}
 &
 \begin{minipage}{5cm}
% \footnotesize
 \begin{verbatim}
\defFunction[algebraic]%
   {helice}(t)
   {3*cos(4*t)}{3*sin(4*t)}{t}
\psSolid[object=courbe,r=0,
   range=0 6,
   linecolor=blue,
   linewidth=0.1,
   resolution=360,
   function=helice]%
 \end{verbatim}
 \end{minipage}
\\\hline
%%    courbeR2
%%       &
%%      \begin{tabular}{l}
%%      trac\'{e} d'une fonction\\
%%      R --> R\textsuperscript{2}\\
%%      d\'{e}finie par ses\\
%%      \'{e}quations param\'{e}triques\\
%%      \end{tabular}
%%      &
%% \psset{unit=0.4cm}
%% \begin{pspicture}(-6,-7)(6,6)
%% \psframe*[linecolor=yellow!50](-6,-6)(6,6)
%% \psset{SphericalCoor,viewpoint=50 -20 30,Decran=50}
%% {\psset{linewidth=0.5\pslinewidth,linecolor=gray}
%% \psSolid[object=grille,base=-4 4 -4 0,RotX=90,RotZ=90]%
%% \psSolid[object=grille,base=-4 4 -4 4]%
%% \psSolid[object=grille,base=-4 4 0 4,RotX=90,RotZ=90]}
%% \defFunction{parabole}(t){t}{t dup mul}{}
%% \defFunction{droite}(t){t}{t 2 add }{}
%% \axesIIID(0,0,0)(4,4,4)
%% \psProjection[object=chemin,
%%       linewidth=.1,
%%       linecolor=blue,
%%       normal=0 1 0 1 0 0,
%%       path=
%%          newpath
%%             0 0 moveto
%%             1 0 lineto]
%% \psProjection[object=chemin,
%%       linewidth=.1,
%%       linecolor=red,
%%       normal=0 1 0 1 0 0,
%%       path=
%%          newpath
%%             0 0 moveto
%%             0 1 lineto]
%% \psProjection[object=courbeR2,
%%    range=-1 2,fillstyle=vlines,hatchwidth=0.5\pslinewidth,
%%    normal=0 1 0 1 0 0,
%%    function=parabole]
%% \psProjection[object=courbeR2,
%%    range=-2 2,
%%    linecolor=green,
%%    normal=0 1 0 1 0 0,
%%    function=parabole]
%% \psProjection[object=courbeR2,
%%    range=-2 2 ,
%%    linecolor=red,
%%    normal=0 1 0 1 0 0,
%%    function=droite]
%% \psPoint(0,0,4.15){Z1}
%% \uput*[60](Z1){$z=y^2$}
%% \rput(0,-6.5){\psframebox[linecolor=yellow!50]{\texttt{$\backslash${}defFunction\{parabole\}(t)\{t\}\{t dup mul\}\{\}}}}
%% \end{pspicture}
%%  &
%%  \begin{minipage}{6cm}
%%  \footnotesize
%%  \begin{verbatim}
%% \psProjection[object=courbeR2,
%%    range=-2 2,
%%    linecolor=green,
%%    normal=0 1 0 1 0 0,
%%    function=parabole]
%%  \end{verbatim}
%%  \end{minipage}
%% \\\hline
\end{tabular}
\end{center}

Some information about rings and parallelepipeds is available in the documents:
\begin{itemize}
  \item \texttt{doc-grille-parallelepiped.tex(.pdf)};
  \item \texttt{doc-anneau.tex(.pdf).}
\end{itemize}
%%%%%%%%%%%%%%%%%%%%%%%%%%%%%%%%%%%%%%%%%%%%%%%%%%
%\newpage

\endinput
