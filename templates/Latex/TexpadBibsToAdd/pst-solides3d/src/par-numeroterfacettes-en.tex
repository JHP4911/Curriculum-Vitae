\section{Numbering of the faces}

The option \Lkeyword{numfaces} gives permission to number every face with its correspondent index number.
\begin{itemize}
  \item \texttt{\Lkeyword{numfaces}=\Lkeyval{all}} all faces are numbered;
  \item \verb+numefaces=0 1 2 3+ only the faces that have index 0, 1, 2 and 3 are numbered.
\end{itemize}
The option \Lkeyword{fontsize} allows to fix the measurement of the used character set.
Finally, the Boolean \Lkeyword{visibility} the numbering of \Index{faces} that are not visible.
By default, the Boolean is set to \texttt{\Lkeyword{visibility}=true}, so the visibility is set up (e.~g. numbers are not set to invisible faces).


\psset{viewpoint=50 20 30 rtp2xyz,Decran=50}
\begin{LTXexample}[width=8cm]
\psset{unit=1}
\begin{pspicture}(-4,-3)(3,1.5)
\psSolid[object=grille,
   base=0 4 -2 2,
   numfaces=2 6 7 10,
   linecolor=gray](0,0,0)
\axesIIID(0,0,0)(4,2,1)
\end{pspicture}
\end{LTXexample}



%% \begin{multicols}{2}
%%
%% \bgroup
%% \psset{SphericalCoor=true,viewpoint=50 20 30}
%% \begin{center}
%% \psset{unit=0.75}
%% \psset{lightsrc=30 -20 10,SphericalCoor=true,viewpoint=50 -20 10,Decran=50}
%% \begin{pspicture*}(-5,-4)(6,6)
%% \psframe(-5,-4)(6,6)
%% \axesIIID(0,0,0)(4,4,4)
%% \psSolid[object=cube,
%%    RotY=90,
%%    ngrid=4,
%%    numfaces=2 6 10,
%%    action=draw**](0,0,0)
%% \end{pspicture*}
%% \end{center}
%% \egroup
%%
%% \columnbreak
%%
%% \begin{verbatim}
%% \axesIIID(0,0,0)(4,4,4)
%% \psSolid[object=cube,
%%    RotY=90,
%%    ngrid=4,
%%    numfaces=2 6 10,
%%    action=draw**](0,0,0)
%% \end{verbatim}
%%
%% \end{multicols}




\psset{viewpoint=50 -20 10 rtp2xyz,Decran=50}
\begin{LTXexample}[width=8cm]
\begin{pspicture*}(-4,-3)(4,3)
\psSolid[object=cube,
   RotY=90,
   ngrid=4,
   fontsize=15,
   action=draw,
   numfaces=all,](0,0,0)
\end{pspicture*}
\end{LTXexample}

%%% exemple 3

The options of \Lcs{psSolid} accept PostScript commands, in particular the \verb+for+ loop.

With the instruction \verb+numfaces=0 1 5 {} for+ all faces with the index numbers between 0
and 5 are set up. The instruction \verb+numfaces=8 3 23 {} for+ sets up every third index number between 8 and 23.

\psset{viewpoint=50 -20 10 rtp2xyz,Decran=50}
\begin{LTXexample}[width=8cm]
\begin{pspicture*}(-4,-3)(4,3)
\axesIIID(0,0,0)(8,3,2)
\psSolid[object=grille,
   RotY=90,
   RotZ=180,
   ngrid=1.,
   fontsize=15,
   numfaces=
      0 1 5 {} for
      8 3 23 {} for,
   base=-2 2 -3 3,
   visibility=false,
   action=draw](0,0,0)
\end{pspicture*}
\end{LTXexample}

\endinput
