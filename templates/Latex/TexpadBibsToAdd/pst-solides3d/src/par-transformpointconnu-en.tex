\section{\Index{Transformations} to a point}

Given is an initial point $A(x,y,z)$. Now we make some
rotations around the axes $Ox$, $Oy$ and $Oz$ with the appropriate angles (in degrees):
\verb+[RotX=valueX,RotY=valueY,RotZ=valueZ]+, in this order,
then translate it with the vector $(v_x,v_y,v_z)$. The problem is to get back
the coordinates of the image (final point) $A'(x',y',z')$.

The code
\texttt{\textbackslash psTransformPoint[RotX=valueX,RotY=valueY,
 RotZ=valueZ](x y z)(vx,vy,vz)\{A'\}}\\
now allows us to save the node $A'$, the coordinates of the transformed point.

In the following example, $A(2,2,2)$ is one of the vertices of the initial
cube, where the centre is placed at the origin.

\begin{verbatim}
\psSolid[object=cube,a=4,action=draw*,linecolor=red]%
\end{verbatim}

Some transformations are applied to the cube:

\begin{verbatim}
\psSolid[object=cube,a=4,action=draw*,RotX=-30,RotY=60,RotZ=-60](7.5,11.25,10)%
\end{verbatim}

To obtain the image of $A$, we use the following command:


\begin{verbatim}
\psTransformPoint[RotX=-30,RotY=60,RotZ=-60](2 2 2)(7.5,11.25,10){A'}
\end{verbatim}

This allows us, for example, to name these points and then draw the vector $\overrightarrow{AA'}$.
\begin{center}
\begin{pspicture}(-2,-4)(6,6)
\psframe(-3,-4)(9,6)
\psset{viewpoint=50 20 30 rtp2xyz,Decran=50,unit=0.5}
\psSolid[object=cube,a=4,action=draw*,linecolor=red]%
\psPoint(2,2,2){A}\psdot(A)
\psSolid[object=cube,a=4,action=draw*,RotX=-30,RotY=60,RotZ=-60](7.5,11.25,10)%
\psTransformPoint[RotX=-30,RotY=60,RotZ=-60](2 2 2)(7.5,11.25,10){A'}
\psdot(A')\psline[linecolor=blue,arrowsize=0.3]{{o-v}}(A)(A')
\uput[u](A'){$A'$}\uput[u](A){$A$}
\psset{solidmemory,action=none}
\psSolid[object=cube,a=4,name=A1,](0,0,0)
\psSolid[object=plan,definition=solidface,args=A1 0,name=P0]
\psSolid[object=plan,definition=solidface,args=A1 1,name=P1]
\psSolid[object=plan,definition=solidface,args=A1 4,name=P4]
\psset{fontsize=100}
\psProjection[object=texte,linecolor=red,text=A,plan=P0]
\psProjection[object=texte,linecolor=red,text=B,plan=P1]
\psProjection[object=texte,linecolor=red,text=E,plan=P4]
\psSolid[object=cube,a=4,RotX=-30,RotY=60,RotZ=-60,name=A2,](7.5,11.25,10)
\psSolid[object=plan,definition=solidface,args=A2 0,name=P'0]
\psSolid[object=plan,definition=solidface,args=A2 1,name=P'1]
\psSolid[object=plan,definition=solidface,args=A2 2,name=P'2]
\psProjection[object=texte,text=A,plan=P'0]
\psProjection[object=texte,text=B,plan=P'1]
\psProjection[object=texte,text=C,plan=P'2]
\axesIIID(2,2,2)(10,10,8)
\end{pspicture}
\end{center}


\endinput
