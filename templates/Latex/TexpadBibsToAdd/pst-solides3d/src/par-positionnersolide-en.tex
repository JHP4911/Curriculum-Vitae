\section{Positioning a solid}

\subsection{\Index{Translation}}

The following command~
\texttt{\Lcs{psSolid}[object=cube,+\textit{options}](x,y,z)} shifts the
centre of the cube to the point with the coordinates $\mathtt{(x,y,z)}$.

The next example will copy the cube with edge length of 1
\begin{pspicture}(-0.5,-0.5)(.5,.5)
\psset{Decran=40,viewpoint=50 35 35 rtp2xyz,a=1,lightsrc=50 30 20}
\psset{fillcolor=yellow,mode=3}
\psSolid[object=cube](0.5,0.5,0.5)% c1
\end{pspicture}
to the points with the coordinates $\mathtt{(0.5,0.5,0.5)}$,
 $\mathtt{(4.5,0.5,0.5)}$ etc. so that the copied cubes setup the vertices
 of a new cube with the edge length 5.
\begin{center}
\begin{pspicture}(-4,-5)(5,5)
\psframe(-4,-5)(5,5)
%\psset{SphericalCoor,Decran=3,viewpoint=10 35 35,a=1,lightsrc=50 20 10}
\psset{Decran=40,viewpoint=50 35 35 rtp2xyz,a=1,lightsrc=50 30 20}
\psSolid[object=grille,base=0 6 0 6,fillcolor=gray!40]%%
\psSolid[object=grille,base=0 6 0 6,RotY=90,fillcolor=gray!30](0,0,6)%
\psSolid[object=grille,base=0 6 0 6,RotX=-90,fillcolor=gray!20](0,0,6)%
\psPoint(1,0.5,0.5){c11}
\psPoint(0.5,0.5,1){c12}
\psPoint(0.5,1,0.5){c13}
\psPoint(4.5,4.5,1){c21}
\psPoint(4,4.5,0.5){c22}
\psPoint(4.5,4,0.5){c23}
\psPoint(4,0.5,0.5){c41}
\psPoint(4.5,0.5,1){c42}
\psPoint(4.5,1,0.5){c43}
\psPoint(0.5,4,0.5){c51}
\psPoint(0.5,4.5,1){c52}
\psPoint(1,4.5,0.5){c53}
\psPoint(0.5,0.5,4){c61}
\psPoint(0.5,1,4.5){c62}
\psPoint(1,0.5,4.5){c63}
\psPoint(4,0.5,4.5){c71}
\psPoint(4.5,1,4.5){c72}
\psPoint(4.5,0.5,4){c73}
\axesIIID(1,1,1)(6,6,6)
{\psset{fillcolor=yellow,mode=3}
\psSolid[object=cube](0.5,0.5,0.5)% c1
\psline[linestyle=dashed,linecolor=red,linewidth=1.5pt](c11)(c41)
\psline[linestyle=dashed,linecolor=red,linewidth=1.5pt](c12)(c61)
\psline[linestyle=dashed,linecolor=red,linewidth=1.5pt](c13)(c51)
\psSolid[object=cube](4.5,0.5,0.5)
\psSolid[object=cube](0.5,4.5,0.5)
\psSolid[object=cube](0.5,0.5,4.5)
\psSolid[object=cube](4.5,4.5,4.5)
\psSolid[object=cube](4.5,0.5,4.5)
\psSolid[object=cube](4.5,4.5,0.5)
\psSolid[object=cube](0.5,4.5,4.5)}
\psSolid[object=grille,base=0 5 0 5,action=draw,linecolor=blue!70,linewidth=1.2pt](0,0,5)%
\psSolid[object=grille,base=0 5 0 5,action=draw,linecolor=blue!70,RotY=90,linewidth=1.2pt](5,0,5)%
\psSolid[object=grille,base=0 5 0 5,action=draw,RotX=-90,linecolor=blue!70,linewidth=1.2pt](0,5,5)%
\end{pspicture}
\end{center}
\begin{verbatim}
\psset{fillcolor=yellow,mode=3}
\psSolid[object=cube](0.5,0.5,0.5)
\psSolid[object=cube](4.5,0.5,0.5)
\psSolid[object=cube](0.5,4.5,0.5)
\psSolid[object=cube](0.5,0.5,4.5)
\psSolid[object=cube](4.5,4.5,4.5)
\psSolid[object=cube](4.5,0.5,4.5)
\psSolid[object=cube](4.5,4.5,0.5)
\psSolid[object=cube](0.5,4.5,4.5)
\end{verbatim}


\subsection{Rotation}

\subsection{Default sequence xyz}

The \Index{rotation} is effected around the three axes $Ox$, $Oy$ and $Oz$. Let's take a cuboid as an example,
\begin{pspicture}(-1,-0.2)(1,.5)
\psset{Decran=40,viewpoint=50 35 35 rtp2xyz,a=2,b=3,c=1,lightsrc=50 30 30}
\psset{fillcolor=yellow,unit=0.5,
  fcol= 0 (Apricot)
        1 (Lavender)
        2 (SkyBlue)
        3 (LimeGreen)
        4 (OliveGreen)
        5 (Yellow)
        6 (Bittersweet)}
\psSolid[object=parallelepiped](0.5,0.5,0.5)%
\end{pspicture}
which will be rotated seperately around the axes $Ox$, $Oy$ and $Oz$.

\begin{multicols}{4}
\psset{Decran=40,viewpoint=50 35 35 rtp2xyz,a=2,b=3,c=1}
\psset{unit=0.5,
  fcol= 0 (Apricot)
        1 (Lavender)
        2 (SkyBlue)
        3 (LimeGreen)
        4 (OliveGreen)
        5 (Yellow)
        6 (Bittersweet),
  object=parallelepiped}
\setlength{\columnseprule}{1pt}
\centerline{
\begin{pspicture}(-2.75,-2.5)(2.95,2.5)
\psframe(-2.75,-2.5)(2.95,2.5)
\psSolid%%
\axesIIID(1,1.5,1)(3,3,2)
\end{pspicture}}
\columnbreak
\centerline{
\begin{pspicture}(-2.75,-2.5)(2.95,2.5)
\psframe(-2.75,-2.5)(2.95,2.5)
\psSolid[RotZ=60]%%
\psSolid[action=draw,linewidth=0.5\pslinewidth]%%
\axesIIID(1,1.5,1)(2,3,2)
\end{pspicture}}

\centerline{\texttt{[RotZ=60]}}

\columnbreak
\centerline{%
\begin{pspicture}(-2.75,-2.5)(2.95,2.5))
\psframe(-2.75,-2.5)(2.95,2.5)
\psSolid[RotX=30]%%
\psSolid[action=draw,linewidth=0.5\pslinewidth]%%
\axesIIID(1,1.5,1)(2,3,2)
\end{pspicture}}

\centerline{\texttt{[RotX=30]}}

\columnbreak
\centerline{%
\begin{pspicture}(-2.75,-2.5)(2.95,2.5)
\psframe(-2.75,-2.5)(2.95,2.5)
\psSolid[RotY=45]%%
\psSolid[action=draw,linewidth=0.5\pslinewidth]%%
\axesIIID(1,1.5,1)(2,3,2)
\end{pspicture}}

\centerline{\texttt{[RotY=-45]}}
\end{multicols}


\subsection{Rotations Sequence}

\newpsstyle{sol}{fillstyle=crosshatch,hatchcolor=green,hatchwidth=0.25\pslinewidth,hatchsep=5\pslinewidth}
\makeatletter
\def\Die#1#2#3#4{
\pstVerb{/posP \pst@solides@a\space 0.3 mul def
         /rP \pst@solides@a\space 0.1 mul def
         /dP \pst@solides@a\space 2 div neg def
         /a_2 \pst@solides@a\space 2 div def}%
\psset{solidmemory}
%\psset{visibility=false}
\psSolid[action=draw**,
         object=cube,
RotX=#2,RotY=#3,RotZ=#4,RotSequence=#1,
         fontsize=15,
         trunccoeff=.1,
         trunc=all,
%         fillcolor=yellow,
         fcol=6 1 13 { (rouge) } for,
         name=A
         ](0,0,0)%
\psSolid[object=plan,action=none,
         definition=solidface,args=A 0,name=P0]
\psset{plan=P0}
\psProjection[object=cercle,fillstyle=solid,fillcolor=black,
              args=0 0 rP,
              range=0 360]
\psSolid[object=plan,action=none,
         definition=solidface,args=A 1,name=P1]
\psset{plan=P1}
\psProjection[object=cercle,fillstyle=solid,fillcolor=black,
              args=0 0 rP,
              range=0 360]
\psProjection[object=cercle,fillstyle=solid,fillcolor=black,
              args=posP posP rP,
              range=0 360]
\psProjection[object=cercle,fillstyle=solid,fillcolor=black,
              args=posP neg posP neg rP,
              range=0 360]
\psSolid[object=plan,action=none,
         definition=solidface,args=A 2,name=P2]
\psset{plan=P2}
\psProjection[object=cercle,fillstyle=solid,fillcolor=black,
              args=posP posP rP,
              range=0 360]
\psProjection[object=cercle,fillstyle=solid,fillcolor=black,
              args=posP neg posP neg rP,
              range=0 360]
\psSolid[object=plan,action=none,
         definition=solidface,args=A 3,name=P3]
\psset{plan=P3}
\psProjection[object=cercle,fillstyle=solid,fillcolor=black,
              args=posP posP rP,
              range=0 360]
\psProjection[object=cercle,fillstyle=solid,fillcolor=black,
              args=posP posP neg rP,
              range=0 360]
\psProjection[object=cercle,fillstyle=solid,fillcolor=black,
              args=posP neg posP rP,
              range=0 360]
\psProjection[object=cercle,fillstyle=solid,fillcolor=black,
              args=posP neg posP neg rP,
              range=0 360]
\psSolid[object=plan,action=none,
         definition=solidface,args=A 4,name=P4]
\psset{plan=P4}
\psProjection[object=cercle,fillstyle=solid,fillcolor=black,
              args=0 0 rP,
              range=0 360]
\psProjection[object=cercle,fillstyle=solid,fillcolor=black,
              args=posP posP rP,
              range=0 360]
\psProjection[object=cercle,fillstyle=solid,fillcolor=black,
              args=posP neg posP neg rP,
              range=0 360]
\psProjection[object=cercle,fillstyle=solid,fillcolor=black,
              args=posP posP neg rP,
              range=0 360]
\psProjection[object=cercle,fillstyle=solid,fillcolor=black,
              args=posP neg posP rP,
              range=0 360]
\psSolid[object=plan,action=none,
         definition=solidface,args=A 5,name=P5]
\psset{plan=P5}
\psProjection[object=cercle,fillstyle=solid,fillcolor=black,
              args=0 posP rP,
              range=0 360]
\psProjection[object=cercle,fillstyle=solid,fillcolor=black,
              args=0 posP neg rP,
              range=0 360]
\psProjection[object=cercle,fillstyle=solid,fillcolor=black,
              args=posP posP rP,
              range=0 360]
\psProjection[object=cercle,fillstyle=solid,fillcolor=black,
              args=posP neg posP neg rP,
              range=0 360]
\psProjection[object=cercle,fillstyle=solid,fillcolor=black,
              args=posP posP neg rP,
              range=0 360]
\psProjection[object=cercle,fillstyle=solid,fillcolor=black,
              args=posP neg posP rP,
              range=0 360]
\psSolid[object=vecteur,
         args=4 0 0,
         linecolor=green](a_2,0,0)%
\psSolid[object=vecteur,
         args=0 4 0,
         linecolor=red](0,a_2,0)
\psSolid[object=vecteur,
         args=0 0 4,
         linecolor=blue](0,0,a_2)
\rput(0,-2.5){\texttt{RotSequence=#1}}
}
\makeatother

\begin{center}
\psset{viewpoint=50 60 25 rtp2xyz,Decran=25,lightsrc=viewpoint,a=4,solidmemory}%
\begin{pspicture}(-3,-3)(3,3)
\psframe(-3,-3)(3,3)
\Die{xyz}{0}{0}{0}
\rput(0,-2){\texttt{RotX=0,RotY=0,RotZ=0}}
\end{pspicture}

\begin{pspicture}(-3,-3)(3,3)
\psframe(-3,-3)(3,3)
\Die{xyz}{90}{90}{90}
\rput(0,-2){\texttt{RotX=90,RotY=90,RotZ=90}}
\end{pspicture}
\begin{pspicture}(-3,-3)(3,3)
\psframe(-3,-3)(3,3)
\Die{xzy}{90}{90}{90}
\rput(0,-2){\texttt{RotX=90,RotY=90,RotZ=90}}
\end{pspicture}

\begin{pspicture}(-3,-3)(3,3)
\psframe(-3,-3)(3,3)
\Die{yxz}{90}{90}{90}
\rput(0,-2){\texttt{RotX=90,RotY=90,RotZ=90}}
\end{pspicture}
\begin{pspicture}(-3,-3)(3,3)
\psframe(-3,-3)(3,3)
\Die{yzx}{90}{90}{90}
\rput(0,-2){\texttt{RotX=90,RotY=90,RotZ=90}}
\end{pspicture}

\begin{pspicture}(-3,-3)(3,3)
\psframe(-3,-3)(3,3)
\Die{zxy}{90}{90}{90}
\rput(0,-2){\texttt{RotX=90,RotY=90,RotZ=90}}
\end{pspicture}
\begin{pspicture}(-3,-3)(3,3)
\psframe(-3,-3)(3,3)
\Die{zyx}{90}{90}{90}
\rput(0,-2){\texttt{RotX=90,RotY=90,RotZ=90}}
\end{pspicture}
\end{center}


\endinput
