\section{Lines}

\subsection{Direct definition}

The object \Lkeyword{line} defines a \Index{line} (or a series of line
segments). We use the option \Lkeyword{args} to specify the points:
\texttt{[object=line,args=$A_0$ $A_1$ \ldots $A_n$]}

We can also define a line that has been transformed using a
translation, a rotation, a homothety, etc., as though it were a
polygon.

\begin{LTXexample}[width=7.5cm]
\begin{pspicture}(-3,-3)(4,3.5)%
\psframe*[linecolor=blue!50](-3,-3)(4,3.5)
\psset{lightsrc=50 20 20,viewpoint=50 30 15,Decran=60}
\psset{solidmemory}
\psSolid[object=grille,
   base=-3 0 -3 3,
   linewidth=0.5\pslinewidth,linecolor=gray,]
%% definition du plan de projection
\psSolid[object=plan,
   definition=equation,
   args={[1 0 0 0] 90},
   base=-3.2 3.2 -2.2 2.2,
   name=monplan,
   planmarks]
\psset{plan=monplan}
\psSolid[object=plan,
   args=monplan,
   linecolor=gray!40,
   plangrid,
   action=none]
\psProjection[object=line,
   args=-1 0 -3 1 1 2,
   name=P]
\psProjection[object=line,
   definition=rotatepol,
   linecolor=blue,
   args=P -1 0 -45]
%% du code jps dans la definition
\psProjection[object=line,
   definition={2 -2 addv} papply,
   linestyle=dashed,
   args=P]
\composeSolid
\axesIIID(4,2,2)(5,4,3)
\end{pspicture}
\end{LTXexample}

\endinput
