\section[Curves of functions from R in R\textsuperscript{3}]%
{Curves of functions from $\mathbb{R}$ in $\mathbb{R}^3$} %$

%% \section{Fonctions R --> R\textsuperscript{3}}

The line of a defined \Index{function} calls the object \Lkeyval{courbe} and the option \Lkeyword{function}.
We can realize a helix in algebraic notation with the function:

\begin{verbatim}
\defFunction[algebraic]{helice}(t){3*cos(4*t)}{3*sin(4*t)}{t}
\end{verbatim}

\psset{lightsrc=10 -20 50,viewpoint=50 -20 20 rtp2xyz,Decran=50}
\begin{LTXexample}[width=6.5cm]
\psset{unit=0.5}
\begin{pspicture}(-6,-3)(6,8)
\psframe*[linecolor=blue!50](-6,-3)(6,8)
\psSolid[object=grille,base=-4 4 -4 4,linecolor=red,linewidth=0.5\pslinewidth]%
\axesIIID(0,0,0)(4,4,7)
\defFunction[algebraic]{helice}(t){3*cos(4*t)}{3*sin(4*t)}{t}
\psSolid[object=courbe,
        r=0,
        range=0 6,
        linecolor=blue,linewidth=0.1,
        resolution=360,
        function=helice]%
\end{pspicture}
\end{LTXexample}

\begin{LTXexample}[width=6.5cm]
\psset{unit=0.5}
\begin{pspicture}(-6,-3)(6,8)
\psframe*[linecolor=blue!50](-6,-3)(6,8)
\psset{lightsrc=10 -20 50,viewpoint=50 -20 30 rtp2xyz,Decran=50}
\psSolid[object=grille,base=-4 4 -4 4,linecolor=red,linewidth=0.5\pslinewidth]%
\axesIIID(0,0,0)(4,4,7)
\psset{range=-4 4}
\defFunction{cosRad}(t){ t 2 mul Cos 4 mul }{ t }{ 0 }
\psSolid[object=courbe,linewidth=0.1,
  r=0,linecolor=red,
  resolution=360,
  function=cosRad]
\psSolid[object=grille,base=-4 4 -4 4,linecolor=blue,linewidth=0.5\pslinewidth](0,0,3)
\psPoint(0,0,3){O1}\psPoint(0,0,7){Z1}\psline(O1)(Z1)\psline[linestyle=dashed](O1)(O)
\pstVerb{/tmin -4 def /tmax 4 def}%
\defFunction{sinRad}(t){ t }{ t Sin 3 mul }{ 3 }
\psSolid[object=courbe,linewidth=0.1,
  r=0,linecolor=blue,
  resolution=30,
  function=sinRad]
\end{pspicture}
\end{LTXexample}

\begin{LTXexample}[width=6.5cm]
\psset{unit=0.5}
\begin{pspicture}(-6.5,-3)(7,11)
\psset{lightsrc=10 -20 50,viewpoint=50 -20 20 rtp2xyz,Decran=50}
\psSolid[object=grille,base=-4 4 -4 4,
  linecolor=lightgray,linewidth=0.5\pslinewidth]%
\psSolid[object=grille,base=-4 4 0 8,
  linecolor=lightgray,RotX=90,
  linewidth=0.5\pslinewidth](0,4,0)
\psSolid[object=grille,base=-4 4 -4 4,
  linecolor=lightgray,RotY=90,
  linewidth=0.5\pslinewidth](-4,0,4)
\defFunction[algebraic]{helice}(t)%
  {1.3*(1-cos(2.5*t))*cos(6*t)}
  {1.3*(1-cos(2.5*t))*sin(6*t)}{t}
\defFunction[algebraic]{helice_xy}(t)%
  {1.3*(1-cos(2.5*t))*cos(6*t)}
  {1.3*(1-cos(2.5*t))*sin(6*t)}{0}
\defFunction[algebraic]{helice_xz}%
  (t){1.3*(1-cos(2.5*t))*cos(6*t)}{4}{t}
\defFunction[algebraic]{helice_yz}%
  (t){-4}{1.3*(1-cos(2.5*t))*sin(6*t)}{t}
\psset{range=0 8}
\psSolid[object=courbe,r=0,linecolor=blue,
  linewidth=0.05,resolution=360,
  normal=0 0 1,function=helice_xy]
\psSolid[object=courbe,r=0,
  linecolor=green,linewidth=0.05,
  resolution=360,normal=0 0 1,
  function=helice_xz]
\psSolid[object=courbe,r=0,
  linewidth=0.05,resolution=360,
  normal=0 0 1,function=helice_yz]
\psSolid[object=courbe,r=0,
  linecolor=red,linewidth=0.1,
  resolution=360,function=helice]
 \end{pspicture}
\end{LTXexample}


These last function lines are found in an animated form on the website:

\centerline{\url{http://melusine.eu.org/syracuse/pstricks/pst-solides3d/animations/}}


\endinput
