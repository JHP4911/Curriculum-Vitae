\section{The parameters of \texttt{pst-solides3d}}

\begin{longtable}{|>{\bfseries\ttfamily\color{blue}}l
 |>{\ttfamily\centering}m{2cm}|m{10cm}|}
  \hline
  \multicolumn{1}{|c|}{\textbf{Parameter}}&
  \multicolumn{1}{c|}{\textbf{Default}}&
  \multicolumn{1}{c|}{\textbf{Description}} \\ \hline\hline
\endfirsthead
\hline
  \multicolumn{1}{|c|}{\textbf{Parameter}}&
  \multicolumn{1}{c|}{\textbf{Default}}&
  \multicolumn{1}{c|}{\textbf{Description}} \\ \hline\hline
\endhead
\multicolumn{3}{|r|}{\textit{Continued on next page}}\\ \hline
\endfoot
\multicolumn{3}{|r|}{\textit{End of table}}\\ \hline
\endlastfoot

object&&predefined objects for use with
\texttt{\textbackslash{}psSolid} and
\texttt{\textbackslash{}psProjection}: \texttt{\Lkeyword{object}=myName}
where \texttt{myName} is the type of object\\
\hline

viewpoint&10 10 10&the coordinates of the point of view\\ \hline

a&2&the value of \texttt{a} has several interpretations: the edge
length of a cube, the radius of the circumscribed sphere of
regular polyhedrons, the length of one of the edges of a
parallelepiped\\ \hline

r&2&the radius of a cylinder or sphere\\ \hline

h&6&the height of a cylinder, cone, truncated cone, or prism\\
\hline

r0&1.5&the inner radius of a torus\\\hline

r1&4&the mean radius of a torus\\ \hline

phi&0&the lower latitude of a spherical zone\\ \hline

theta&90&the upper latitude of a spherical zone\\ \hline

a,b and c&4&the lengths of three incident edges of a parallelepiped\\
\hline

base&\begin{tabular}{rr}-1 & -1 \\ 1 & -1 \\ 0 &
1\end{tabular}&the coordinates of vertices in the $xy$-plane
for specified shapes\\
\hline

axe&0 0 1&the direction of the axis of inclination of a prism\\
\hline

action&draw**&uses the painting algorithm to draw the solid
without hidden edges and with coloured faces\\ \hline

lightsrc&20 30 50&the Cartesian coordinates of the light source\\
\hline

lightintensity&2&the intensity of the light source\\ \hline

ngrid&n1 n2& sets the grid for a chosen solid\\ \hline

mode&0&sets a predefined grid: values are 0 to 4.
\texttt{mode=0} is a large grid and \texttt{mode=4} is a fine
grid\\ \hline

grid& true&if \texttt{grid} is used then gridlines are suppressed\\
\hline

biface&true&draw the interior face; if you only want the exterior
shown write \texttt{biface=false}
\\ \hline

algebraic&false&\texttt{algebraic=true} (also written as
\texttt{[algebraic]}) allows you to give the equation of a surface
in algebraic form (otherwise RPN is enabled); the package
\texttt{pstricks-add} must be loaded in the preamble\\ \hline

fillcolor&white&specifies a colour for the outer faces of a
solid\\ \hline

incolor&green&specifies a colour for the inner faces of a solid\\
\hline

hue&&the colour gradient used for the outer faces of a solid\\
\hline

inhue&&the colour gradient used for internal faces\\
\hline

inouthue&&the colour gradient used for both internal and
external faces as a single continuation\\
\hline

fcol&&permits you to specify, in order of face number $0$ to $n-1$
(for $n$ faces) the colour of the appropriate face:\par
\texttt{fcol=0 (Apricot) 1 (Aquamarine) etc.}\\ \hline

rm&&removes visible faces: \texttt{rm=1 2 8} removes faces 1, 2
and 8 \\ \hline

show&&determines which vertices are shown as points:
\texttt{show=0 1 2 3} shows the vertices 0, 1, 2 and 3,
\texttt{show=all} shows all the vertices\\ \hline

num&&numbers the vertices; for example \texttt{num=0 1 2 3}
numbers the vertices 0,1,2 and 3, and \texttt{num=all} numbers
all the vertices\\ \hline

name&&the name given to a solid\\ \hline

solidname&&the name of the active solid\\ \hline

RotX&0&the angle of rotation of the solid around $Ox$ (in
degrees)\\ \hline

RotY&0&the angle of rotation of the solid around $Oy$ (in
degrees)\\ \hline

RotZ&0&the angle of rotation of the solid around $Oz$ (in
degrees)\\ \hline

hollow&false& draws the inside of hollow solids: cylinder, cone,
truncated cone and prism\\ \hline

decal&-2&reassign the index numbers of the vertices within a \texttt{base}\\
\hline

axesboxed& false& this option for surfaces allows semi-automatic
drawing of the 3D coordinate axes, since the limits of $z$ must be
set by
hand; enabled with \texttt{axesboxed}\\
\hline

Zmin&$-4$& the minimum value of $z$\\ \hline

Zmax&$4$& the maximum value of $z$\\ \hline

QZ&$0$& shifts the coordinate axes vertically by the chosen value\\
\hline

spotX&dr&the position of the tick labels on the $x$-axis\\ \hline

spotY&dl&the position of the tick labels on the $y$-axis\\ \hline

spotZ&l&the position of the tick labels on the $z$-axis\\ \hline

resolution&36&the number of points used to draw a curve\\ \hline

range&-4 4 &the limits for function input\\ \hline

function& f & the name given to a function\\ \hline

path&newpath \par 0 0 moveto& the projected path\\ \hline

%normal&0 0 1&the normal to the surface being defined\\ \hline

text&&the projected text\\ \hline

visibility&false& if \texttt{false} the text applied to a hidden
face is
not rendered\\
\hline

chanfreincoeff&0.2&the chamfering coefficient\\ \hline

trunccoeff&0.25&the truncation coefficient\\ \hline

dualregcoeff&1&the dual solid coefficient\\ \hline %%%% is this used anywhere?

affinagecoeff&0.8&the hollowing coefficient\\ \hline

affinage& & determines which faces are hollowed out:
\texttt{affinage=0 1 2 3} recesses faces 0, 1, 2 and 3,
\texttt{affinage=all} recesses all faces\\ \hline

affinagerm& &keep the central part of hollowed out faces\\ \hline

intersectiontype&-1&the type of intersection between a plane and a
solid; a positive value draws the intersection\\ \hline

plansection&&list of equations of intersecting planes, when used
only for their intersections \\
\hline

plansepare&&the equation of the separating plane for a solid\\
\hline

{\small intersectionlinewidth}&1&the thickness of an intersection
in \texttt{pt}; if there are several inter\-sections of different
thicknesses then list them like so:\par
\texttt{intersectionlinewidth=1 1.5 1.8 etc.}\\
\hline

intersectioncolor&(rouge)&the colour used for intersections; if
several inter\-sections in different colours are required, list
them as follows:\par \texttt{intersectioncolor=(rouge) (vert) etc.}\\
\hline

intersectionplan&[0 0 1 0]&the equation of the intersecting
plane\\ \hline

definition&&defines a point, a vector, a plane, a spherical arc,
etc.\\ \hline

args&&arguments associated with \texttt{definition}\\
\hline

section&\textbackslash Section&the coordinates of the vertices of
a cross-section of a solid ring\\ \hline

planmarks&false&scales the axes of the plane\\ \hline

plangrid&false&draws the coordinate axes of the plane \\ \hline

showbase&false&draws the unit vectors of the plane\\ \hline

showBase&false&draws the unit vectors of the plane and the normal
vector to the plane\\ \hline

deactivatecolor&false&disables the colour management of PSTricks\\
\hline

transform&&a formula, applied to the vertices of a solid, to
transform it\\ \hline

axisnames&\{x,y,z\}&the labels of the axes in 3D\\ \hline

axisemph&&the style of the axes labels in 3D\\ \hline

showOrigin&true&draws the axes from the origin, or not if set to
\texttt{false}\\ \hline

mathLabel&true&draws the axes labels in math mode, or not if set
to \texttt{false}\\ \hline

file&&the name of the data file having \texttt{.dat} extension
written with \texttt{action=writesolid} or read with
\texttt{object=datfile}\\
\hline

load&&the name of the object to be loaded\\ \hline

fcolor&&the colour of the refined parts of the faces of an object\\
\hline

sommets&&the list of vertices of a solid for use with \texttt{object=new}\\
\hline

faces&&the list of faces of a solid for use with \texttt{object=new}\\
\hline

stepX&1&a positive integer giving the interval between ticks on
the $x$-axis of \texttt{\textbackslash{}gridIIID}\\ \hline

stepY&1&a positive integer giving the interval between ticks on
the $y$-axis of \texttt{\textbackslash{}gridIIID}\\ \hline

stepZ&1&a positive integer giving the interval between ticks on
the $z$-axis of \texttt{\textbackslash{}gridIIID}\\ \hline

ticklength&0.2&the length of tickmarks for
\texttt{\textbackslash{}gridIIID}\\ \hline

\end{longtable}

\endinput
