%% $Id: pst-news13.tex 856 2013-12-09 10:34:40Z herbert $
\documentclass[11pt,english,BCOR10mm,DIV12,bibliography=totoc,parskip=false,smallheadings
    headexclude,footexclude,oneside]{pst-doc}
\listfiles
\let\Lfile\LFile
\usepackage[utf8]{inputenc}
\usepackage{pst-node}
\let\pstnodeFV\fileversion
\let\pstnodeFD\filedate
\usepackage{pst-plot}
\usepackage{xkvview}
\renewcommand\bgImage{\psscalebox{15}{\color{blue!20}2014}}
\def\textat{\char064}
\lstset{explpreset={pos=l,width=-99pt,overhang=0pt,hsep=\columnsep,vsep=\bigskipamount,rframe={}},
    escapechar=?}
\begin{document}

%\psset{PstDebug=1}
\title{\texttt{News -- 2014}\\ \Large new macros and bugfixes for the
basic package \nxLFile{pstricks}}
\author{Herbert Voß}
\date{\today}

\maketitle

\clearpage
\tableofcontents

\clearpage
\part{\texttt{pstricks} -- package}

%--------------------------------------------------------------------------------------
%\section{\texttt{pstricks.sty}}
%--------------------------------------------------------------------------------------


%--------------------------------------------------------------------------------------
\section{\texttt{pstricks.tex} (\pstricksFV -- \pstricksFD)}
%--------------------------------------------------------------------------------------
\subsection{Opacity}
The keyword \Lkeyword{strokeopacity} is now also valid for \Lcs{psdot}, \Lcs{psdots},
and the \Lkeyword{linestyle}/\Lkeyword{plotstyle}=\Lkeyval{dots}.


\subsection{PostScript notation for numbers}
Optional arguments which expects a real number can now have a preceeding ! character for
a PostScript notation which is directly passed to PostScript. The user has take care that
such a number isn't use before in another \TeX\ macro. In such a case it gives an error.

\begin{LTXexample}[width=5cm]
\pstVerb{ 1234321 srand }
\begin{pspicture}[showgrid](-2,-2)(2,2)
\psframe*[linecolor=blue,opacity=!Rand](2,2)
\psframe*[linecolor=red,opacity=!Rand](-1,-1)(1,1)
\psframe*[linecolor=green,opacity=!Rand](-2,-2)(0,0)
\end{pspicture}
\end{LTXexample}


\subsection{Fillstyle \texttt{eofill}}

It is an experimental fillstyle. PostScript knows only the \Lkeyval{eofill} and the other way round
needs some tricky internal commands and may not work in all cases.

\begin{LTXexample}[pos=t]
\begin{pspicture}[linewidth=2pt](12,4)    
\pscustom[linestyle=none,fillstyle=eofill,fillcolor=blue!40]{%
   \psellipse(4,2)(2,2)\psellipse(2,2)(2,2)}
\psellipse[linecolor=red](4,2)(2,2)\psellipse[linecolor=green](2,2)(2,2)
%
\pscustom[linestyle=none,fillstyle=oefill,fillcolor=blue!40]{%
   \psellipse(10,2)(2,2)\psellipse(8,2)(2,2)}
\psellipse[linecolor=red](10,2)(2,2)\psellipse[linecolor=green](8,2)(2,2)
\end{pspicture}
\end{LTXexample}


\subsection{Option \texttt{correctAngle}}
It now works also for \Lcs{psellipiticwedge}. The setting of \Lkeyword{origin} is needed
if the center of the ellipse is not the origin of the underlying coordinate system.

\begin{LTXexample}[pos=t]
\begin{pspicture}[dimen=m,showgrid=top](6,4)
\pnodes{P}(3,1)(5,1)(4,2)
\pcline[nodesep=-1](P0)(P1)
\pcline[nodesep=-1](P0)(P2)
\psellipticarc[origin={P0},correctAngle](P0)(2,1){(P2)}{(P1)}
\psellipticwedge[origin={P0},linecolor=red,correctAngle,
                 fillstyle=vlines](P0)(2,1){(P2)}{(P1)}
\end{pspicture}
\end{LTXexample}

\begin{LTXexample}[pos=t]
\begin{pspicture}[dimen=m,showgrid=top](-3,-1)(3,3)
\pnodes{P}(0,0)(2,0)(1,1)
\pcline[nodesep=-1](P0)(P1)
\pcline[nodesep=-1](P0)(P2)
\psellipticarc[correctAngle](P0)(2,1){(P2)}{(P1)}
\psellipticwedge[linecolor=red,correctAngle,
                 fillstyle=vlines](P0)(2,1){(P2)}{(P1)}
\end{pspicture}
\end{LTXexample}

\clearpage
\subsection{New macro \nxLcs{psellipseAB}}

\begin{BDef}
\LcsStar{psellipseAB}\OptArgs\Largr{\CAny}\Largb{half radius}
\end{BDef}


\begin{LTXexample}[width=7cm]
\begin{pspicture}[showgrid=true](7,7)%% showgrid=true
\pnodes{a}(0.5,0)(2.5,1.8)(5.5,2.5)(6.25,3)(7,5) 
\pnodes{b}(0,1)(2,3)(5,4)(5.5,5)(6,7)
\pscurve[arrowscale=2,linewidth=1.2pt]{->}(a0)(a1)(a2)(a3)(a4)
\pscurve[arrowscale=2,linewidth=1.2pt]{->}(b0)(b1)(b2)(b3)(b4)
\psellipseAB(a0)(b0){0.1}
\psellipseAB[fillcolor=red!40,fillstyle=solid](a1)(b1){0.15} 
\psellipseAB(a2)(b2){0.2}
\psellipseAB[fillcolor=blue!40,fillstyle=solid](a3)(b3){0.25}
\uput[135](b1){$dS_1$}\uput[135](b2){$dS_2$}
\end{pspicture}
\end{LTXexample}




\subsection{New macro \nxLcs{psRing}}

\begin{BDef}
\LcsStar{psRing}\OptArgs\Largr{\CAny}\OptArg{start,end}\Largb{Inner Radius}\Largb{Outer Radius}
\end{BDef}


\begin{LTXexample}[width=5cm]
\begin{pspicture}[showgrid](4,4)
  \psRing(2,2){0.3}{0.8}
  \psRing*[opacity=0.5](2,2){1}{2}
\psdot(2,2)
\end{pspicture}
\end{LTXexample}


\begin{LTXexample}[width=5cm]
\begin{pspicture}[showgrid](4,4)
  \psRing[linecolor=red](2,2)[30,60]{1}{2}
  \psRing[opacity=0.5,fillstyle=solid,
    fillcolor=red](2,2)[60,30]{1}{2}
\psdot(2,2)
\end{pspicture}
\end{LTXexample}



\subsection{New macro \nxLcs{pscspline} (by Christoph Bersch)}

\begin{BDef}
\LcsStar{pscspline}\OptArgs\Largr{$x_0,y_0$}\Largr{$x_1,y_1$}\ldots\Largr{\CAny}
\end{BDef}


\begin{LTXexample}[width=5cm]
\begin{pspicture}[showgrid](5,5)
  \pscspline[arrows=|->, showpoints](0,0)(1,2)
  \pscspline[arrows=->, showpoints](0,4)(2,3)(3,4)(5,0)
  \pscspline(4,4)
\end{pspicture}\par
\begin{pspicture}[showgrid](5,5)
  \pscspline[linestyle=symbol, symbol=U, 
    symbolStep=12pt](0,0)(1,2)
  \pscspline[linestyle=symbol, symbol=a, 
    symbolStep=12pt](0,4)(2,3)(3,4)(5,0)
\end{pspicture}
\end{LTXexample}

\bigskip
\begin{LTXexample}[width=5cm]
\begin{pspicture}[showgrid](5,5)
  \pscustom[fillcolor=red!20, fillstyle=solid]{%
    \pscspline(0,3)(4,2)(5,0)
    \lineto(0,0)
    \closepath}
\end{pspicture}
\end{LTXexample}

\subsection{\nxLcs{Special Coor}}

The Macro \Lcs{SpecialCoor} for scanning special coodinate expressions is now enabled by
default. You can disable it with \Lcs{NormalCoor}.

\clearpage
\nocite{*}
\bibliographystyle{plain}
\bibliography{PSTricks}

\printindex


\end{document}


