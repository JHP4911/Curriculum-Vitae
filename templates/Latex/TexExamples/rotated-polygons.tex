% Rotated polygons
% Author: Dr. rer. nat. Ismael Gutierrez Garcia
\documentclass{article}
\usepackage[usenames,dvipsnames,pdftex]{xcolor}
\usepackage{tikz}
%%%<
\usepackage{verbatim}
\usepackage[active,tightpage]{preview}
\PreviewEnvironment{tikzpicture}
\setlength\PreviewBorder{5pt}%
%%%>
\begin{comment}
:Title: Rotated polygons
:Tags: Foreach;Geometry;Decorative drawings
:Author: Dr. rer. nat. Ismael Gutierrez Garcia
:Slug: rotated-polygons

Extending the rotated triangle example
http://www.texample.net/tikz/examples/rotated-triangle/
by Alain Matthes, here are rotated hexagon, pentagon and squares.
\end{comment}
\usepackage{ifthen}
\begin{document}
\newcounter{density}
\setcounter{density}{20}
\begin{tikzpicture}
    \def\couleur{BlueGreen}
    \path[coordinate] (0,0)  coordinate(A)
                ++( 120:6cm) coordinate(B)
                ++(60:6cm) coordinate(C)
                ++(0:6cm) coordinate(D)
                ++(-60:6cm) coordinate(E)
                ++(240:6cm) coordinate(F)
                ;
    \draw[fill=\couleur!\thedensity] (A) -- (B) -- (C) --(D) -- (E) -- (F)-- cycle;
    \foreach \x in {1,...,40}{%
        \pgfmathsetcounter{density}{\thedensity+10}
        \setcounter{density}{\thedensity}
        \path[coordinate] coordinate(X) at (A){};
        \path[coordinate] (A) -- (B) coordinate[pos=.10](A)
                            -- (C) coordinate[pos=.10](B)
                            -- (D) coordinate[pos=.10](C)
                            -- (E) coordinate[pos=.10](D)
                             -- (F) coordinate[pos=.10](E)
                            -- (X) coordinate[pos=.10](F);
        \draw[fill=\couleur!\thedensity] (A)--(B)--(C)-- (D) --(E) -- (F) -- cycle;
    }
\end{tikzpicture}

\setcounter{density}{20}
\begin{tikzpicture}
    \def\couleur{LimeGreen}
    \path[coordinate] (0,0)  coordinate(A)
                ++( 144:10cm) coordinate(B)
                ++(72:10cm) coordinate(C)
                ++(0:10cm) coordinate(D)
                ++(-72:10cm) coordinate(E)
                                ;
    \draw[fill=\couleur!\thedensity] (A) -- (B) -- (C) --(D) -- (E) --  cycle;
    \foreach \x in {1,...,40}{%
        \pgfmathsetcounter{density}{\thedensity+10}
        \setcounter{density}{\thedensity}
        \path[coordinate] coordinate(X) at (A){};
        \path[coordinate] (A) -- (B) coordinate[pos=.10](A)
                            -- (C) coordinate[pos=.10](B)
                            -- (D) coordinate[pos=.10](C)
                            -- (E) coordinate[pos=.10](D)
                             -- (X) coordinate[pos=.10](E);
        \draw[fill=\couleur!\thedensity] (A)--(B)--(C)-- (D) --(E)  -- cycle;
    }
\end{tikzpicture}

\setcounter{density}{20}
\begin{tikzpicture}
    \def\couleur{OrangeRed}
    \path[coordinate] (0,0)  coordinate(A)
                ++( 90:12cm) coordinate(B)
                ++(0:12cm) coordinate(C)
                ++(-90:12cm) coordinate(D);
    \draw[fill=\couleur!\thedensity] (A) -- (B) -- (C) --(D) -- cycle;
    \foreach \x in {1,...,40}{%
        \pgfmathsetcounter{density}{\thedensity+20}
        \setcounter{density}{\thedensity}
        \path[coordinate] coordinate(X) at (A){};
        \path[coordinate] (A) -- (B) coordinate[pos=.10](A)
                            -- (C) coordinate[pos=.10](B)
                            -- (D) coordinate[pos=.10](C)
                            -- (X) coordinate[pos=.10](D);
        \draw[fill=\couleur!\thedensity] (A)--(B)--(C)-- (D) -- cycle;
    }
\end{tikzpicture}
\end{document}