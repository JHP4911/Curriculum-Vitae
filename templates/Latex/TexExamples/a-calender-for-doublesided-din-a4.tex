% DIN-A4 doublesided year calendar
% Author: Robert Krause
% License : Creative Commons attribution license
% Submitted to TeXample.net on 13 July 2012
\documentclass[landscape,a4paper, ngerman, 10pt]{scrartcl}
\usepackage[utf8]{inputenc}
\usepackage[ngerman]{babel}
\usepackage[T1]{fontenc}
\usepackage{tikz} 			% Use the calendar.sty style

\usepackage{translator}	% German Month and Day names
\usepackage{fancyhdr}		% header and footer
\usepackage{fix-cm}		% Large year in header

\usepackage[landscape, headheight = 2cm, margin=.5cm,
  top = 3.2cm, nofoot]{geometry}
\usetikzlibrary{calc}
\usetikzlibrary{calendar}
%%%<
\usepackage{verbatim}
\usepackage[tightpage]{preview}
\PreviewEnvironment{tikzpicture}
\setlength\PreviewBorder{5pt}%
%%%>
\begin{comment}
:Title: A calender for doublesided DIN-A4
Tags: Calendar library;Calendars
:Author: Robert Krause

An example how the calendar package can be used to provide
an doublesided calendar for the whole year.
\end{comment}

\renewcommand*\familydefault{\sfdefault}

% User defined
\def\year{2013}
% Names of Holidays are inserted by employing this macro
\def\termin#1#2{
  \node [anchor=north west, text width= 3.4cm] at
    ($(cal-#1.north west)+(3em, 0em)$) {\tiny{#2}};
}

%Header
\renewcommand{\headrulewidth}{0.0pt}
\setlength{\headheight}{10ex}
\chead{
  \fontsize{60}{70}\selectfont\textbf{\year}
  \Large\textbf{Ferienkalender}\hfill
}
%Footer
\cfoot{\footnotesize\texttt{http://www.texample.net/}}



\begin{document}
\pagestyle{fancy}
\begin{center}
\begin{tikzpicture}[every day/.style={anchor = north}]
\calendar[
  dates=\year-01-01 to \year-06-30,
  name=cal,
  day yshift = 3em,
  day code=
  {
    \node[name=\pgfcalendarsuggestedname,every day,shape=rectangle,
    minimum height= .53cm, text width = 4.4cm, draw = gray]{\tikzdaytext};
    \draw (-1.8cm, -.1ex) node[anchor = west]{\footnotesize%
      \pgfcalendarweekdayshortname{\pgfcalendarcurrentweekday}};
  },
  execute before day scope=
  {
    \ifdate{day of month=1}
    {
      % Shift right
      \pgftransformxshift{4.8cm}
      % Print month name 
      \draw (0,0)node [shape=rectangle, minimum height= .53cm,
        text width = 4.4cm, fill = red, text= white, draw = red, text centered]
        {\textbf{\pgfcalendarmonthname{\pgfcalendarcurrentmonth}}};
    }{}
    \ifdate{workday}
    {
      % normal days are white
      \tikzset{every day/.style={fill=white}}
      % Vacation (Germany, Baden-Wuerrtemberg) gray background
      \ifdate{between=2012-12-24 and 2013-01-05}{%
        \tikzset{every day/.style={fill=gray!30}}}{}
      \ifdate{between=2013-03-25 and 2013-04-05}{%
        \tikzset{every day/.style={fill=gray!30}}}{}
      \ifdate{between=2013-05-21 and 2013-06-01}{%
        \tikzset{every day/.style={fill=gray!30}}}{}
      \ifdate{between=2013-07-25 and 2013-09-07}{%
        \tikzset{every day/.style={fill=gray!30}}}{}
      \ifdate{between=2013-10-28 and 2013-10-30}{%
        \tikzset{every day/.style={fill=gray!30}}}{}
      \ifdate{between=2013-12-23 and 2014-01-04}{%
        \tikzset{every day/.style={fill=gray!30}}}{}
    }{}
    % Saturdays and half holidays (Christma's and New year's eve)
    \ifdate{Saturday}{\tikzset{every day/.style={fill=red!10}}}{}
    \ifdate{equals=12-24}{\tikzset{every day/.style={fill=red!10}}}{}
    \ifdate{equals=12-31}{\tikzset{every day/.style={fill=red!10}}}{}
    % Sundays and full holidays
    \ifdate{Sunday}{\tikzset{every day/.style={fill=red!20}}}{}
    \ifdate{equals=01-01}{\tikzset{every day/.style={fill=red!20}}}{}
    \ifdate{equals=01-06}{\tikzset{every day/.style={fill=red!20}}}{}
    \ifdate{equals=05-01}{\tikzset{every day/.style={fill=red!20}}}{}
    \ifdate{equals=10-03}{\tikzset{every day/.style={fill=red!20}}}{}
    \ifdate{equals=11-01}{\tikzset{every day/.style={fill=red!20}}}{}
    \ifdate{equals=12-25}{\tikzset{every day/.style={fill=red!20}}}{}
    \ifdate{equals=12-26}{\tikzset{every day/.style={fill=red!20}}}{}
    % Christian holidays
    \ifdate{equals=2013-03-29}{\tikzset{every day/.style={fill=red!20}}}{}
    \ifdate{equals=2013-04-01}{\tikzset{every day/.style={fill=red!20}}}{}
    \ifdate{equals=2013-05-09}{\tikzset{every day/.style={fill=red!20}}}{}
    \ifdate{equals=2013-05-20}{\tikzset{every day/.style={fill=red!20}}}{}
    \ifdate{equals=2013-05-30}{\tikzset{every day/.style={fill=red!20}}}{}
  },
 execute at begin day scope=
  {
    % each day is shifted down according to the day of month
    \pgftransformyshift{-.53*\pgfcalendarcurrentday cm}
  }
];

% Print name of Holidays
\termin{\year-01-01}{Neujahr}
\termin{\year-01-06}{Heilige Drei Könige}
\termin{2013-03-29}{Karfreitag}
\termin{2013-03-31}{Ostersonntag}
\termin{2013-04-01}{Ostermontag}
\termin{\year-05-01}{Tag der Arbeit}
\termin{2013-05-09}{Christi Himmelfahrt}
\termin{2013-05-19}{Pfingstsonntag}
\termin{2013-05-20}{Pfingstmontag}
\termin{2013-05-30}{Fronleichnam}
\end{tikzpicture}
% Repeat the whole thing for the second page
\pagebreak
\begin{tikzpicture}[every day/.style={anchor = north}]
\calendar[dates=\year-07-01 to \year-12-31,
  name=cal,
  day yshift = 3em,
  day code=
  {
    \node[name=\pgfcalendarsuggestedname,every day,shape=rectangle, 
      minimum height= .53cm, text width = 4.4cm, draw = gray]{\tikzdaytext};
    \draw (-1.8cm, -.1ex) node[anchor = west]
    {
      \footnotesize\pgfcalendarweekdayshortname{\pgfcalendarcurrentweekday}
    };
  },
  execute before day scope=
  {
    \ifdate{day of month=1} {
    % Shift right
    \pgftransformxshift{4.8cm}
    % Print month name 
    \draw (0,0)node [shape=rectangle, minimum height= .53cm, 
      text width = 4.4cm, fill = red, text= white, draw = red, text centered]
    {
      \textbf{\pgfcalendarmonthname{\pgfcalendarcurrentmonth}}
    };
  }{}
  \ifdate{workday}
  {
    \tikzset{every day/.style={fill=white}}
    % Vacation (Germany Baden-Wuerrtemberg)
    \ifdate{between=2012-12-24 and 2013-01-05}{%
      \tikzset{every day/.style={fill=gray!30}}}{}
    \ifdate{between=2013-03-25 and 2013-04-05}{%
      \tikzset{every day/.style={fill=gray!30}}}{}
    \ifdate{between=2013-05-21 and 2013-06-01}{%
      \tikzset{every day/.style={fill=gray!30}}}{}
    \ifdate{between=2013-07-25 and 2013-09-07}{%
      \tikzset{every day/.style={fill=gray!30}}}{}
    \ifdate{between=2013-10-28 and 2013-10-30}{%
      \tikzset{every day/.style={fill=gray!30}}}{}
    \ifdate{between=2013-12-23 and 2014-01-04}{%
      \tikzset{every day/.style={fill=gray!30}}}{}
  }{}
  % Saturdays and half holidays (Christma's and New year's eve)
  \ifdate{Saturday}{\tikzset{every day/.style={fill=red!10}}}{}
  \ifdate{equals=12-24}{\tikzset{every day/.style={fill=red!10}}}{}
  \ifdate{equals=12-31}{\tikzset{every day/.style={fill=red!10}}}{}
  % Sundays and full holidays
  \ifdate{Sunday}{\tikzset{every day/.style={fill=red!20}}}{}
  \ifdate{equals=01-01}{\tikzset{every day/.style={fill=red!20}}}{}
  \ifdate{equals=01-06}{\tikzset{every day/.style={fill=red!20}}}{}
  \ifdate{equals=05-01}{\tikzset{every day/.style={fill=red!20}}}{}
  \ifdate{equals=10-03}{\tikzset{every day/.style={fill=red!20}}}{}
  \ifdate{equals=11-01}{\tikzset{every day/.style={fill=red!20}}}{}
  \ifdate{equals=12-25}{\tikzset{every day/.style={fill=red!20}}}{}
  \ifdate{equals=12-26}{\tikzset{every day/.style={fill=red!20}}}{}
  % Christian holidays
  \ifdate{equals=2013-03-29}{\tikzset{every day/.style={fill=red!20}}}{}
  \ifdate{equals=2013-04-01}{\tikzset{every day/.style={fill=red!20}}}{}
  \ifdate{equals=2013-05-09}{\tikzset{every day/.style={fill=red!20}}}{}
  \ifdate{equals=2013-05-20}{\tikzset{every day/.style={fill=red!20}}}{}
  \ifdate{equals=2013-05-30}{\tikzset{every day/.style={fill=red!20}}}{}
  },
  execute at begin day scope=
  {
   % Each day is shifted down according to the day of month
    \pgftransformyshift{-.53*\pgfcalendarcurrentday cm}
  }
];
% Holidaynames
\termin{\year-10-03}{Tag der deutschen Einheit}
\termin{\year-11-01}{Allerheiligen}
\termin{\year-12-24}{Heilig Abend}
\termin{\year-12-25}{1.\ Weihnachtsfeiertag}
\termin{\year-12-26}{2.\ Weihnachtsfeiertag}
\termin{\year-12-31}{Silvester}
\end{tikzpicture}
\end{center}
\end{document}
