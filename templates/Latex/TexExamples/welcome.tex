\documentclass[11pt]{article}

\title{Welcome to Texpad iOS}

\begin{document}

\section{Project Switcher}

The Project switcher is screen welcoming you to Texpad. Recently opened projects appear along the top, and special panes are accessed via the buttons along the bottom.

\begin{description}
  \item[Files] This gives you access to a file manager from which you can create, move, import and generally manipulate files
  \item[Create] Press this to open a project creation view.  Specify a name, a destination (Local, Dropbox or Texpad Connect) and a type (Markdown or LaTeX), then press create.
  \item[Settings] This pane contains a list of configuration options for you to customise your copy of Texpad.
  \item[Help] The helpcentre should be your first stop for any questions about Texpad - we are continually adding information to this help centre.  The contact us option at the bottom allows you to communicate with us.
\end{description}

\section{Editor and Typesetting}
  The editor and typesetter lies at the heart of Texpad.  When you enter your document you will be presented with the editor.  Press the three bars at the top left on iPad, or back on iPhone to show the outline view.  This allows you to browse your document by the document structure, add images and files to your project and view metadata on your document.

\section{Features}
  \begin{description}
    \item[Global search] There is regex capable Global Search available from the editor toolbar on iPad and the outline view on iPhone.
    \item[Bundle manager] The bundle manager is part of the settings pane and it manages the download and installation of extra TeX bundles.  We have included a minimal set of bundles with Texpad itself, but the less commonly used bundles must be downloaded through the bundle manager.  There is rarely a need to initiate a download yourself however, simply attempt to typeset the file, if it fails due to the absence of a specific bundle, then Texpad will automatically give you the option to download the relevant bundles.
    \item[Customised Keyboard for iPad] A strip across the top of the keyboard is no substitute for an onscreen keyboard written from scratch with LaTeX in mind.  We have added a smallbackslash key to the left of the A key, reorganised the common LaTeX symbols onto the first alternate keyboard, and the new \textit{Magic} keyboard has 3 modes - edit mode, snippets mode, and a symbols mode.  If you prefer the standard keyboard you can swap to it by from the new editor settings dropdown available from the toolbar.
    \item[Autocomplete] Texpad has an advanced and mature Autocomplete system, not only will it autocomplete common commands, it will attempt to autofill \texttt{cite} and \texttt{ref} tags with symbols defined elsewhere in the document.  Aditionally it scans your files to correctly autofill \texttt{include} and \texttt{includegraphics} type commands.
    \item[Plain TeX] Texpad's local typesetter supports Plain TeX as well as LaTeX.
    \item[Background syntax checking] Texpad runs LaTeX in the background as you work, and shows any errors it finds in the line number view.
    \item[Many other editor tricks] For example, place the character at the end of this line then press return...
  \end{description}

\end{document}