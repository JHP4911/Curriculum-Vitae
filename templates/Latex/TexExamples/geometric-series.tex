% Geometric representation of the sum 1/4 + 1/16 + 1/64 + 1/256 + ...
% Author: Jimi Oke
\documentclass{article}
\usepackage{tikz}
%%%<
\usepackage{verbatim}
\usepackage[active,tightpage]{preview}
\PreviewEnvironment{tikzpicture}
\setlength\PreviewBorder{5pt}%
%%%>
\begin{comment}
:Title: Representation of a geometric series
:Tags: Foreach; Scopes
:Author: Jimi Oke
:Slug: geometric-series

The infinite series 1/4 + 1/16 + 1/64 + 1/256 + ... is one of the first computed infinite series in the history of mathematics, already used by Archimedes. Its sum is 1/3. 
\end{comment}
\begin{document}
\begin{tikzpicture}[scale=.35]\footnotesize
 \pgfmathsetmacro{\xone}{-.4}
 \pgfmathsetmacro{\xtwo}{ 16.4}
 \pgfmathsetmacro{\yone}{-.4}
 \pgfmathsetmacro{\ytwo}{16.4}

\begin{scope}<+->;
% grid
  \draw[step=1cm,gray,very thin] (\xone,\yone) grid (\xtwo,\ytwo);

% ticks
  \foreach \x/\xtext in { 8/\frac{1}{2}, 16/1}
  \draw[gray,xshift=\x cm] (0,.3) -- (0,0) node[below] {$\xtext$};
  \foreach \y/\ytext in {8/\frac{1}{2},16/1}
    \draw[gray, yshift=\y cm] (.3,0) -- (0,0)
    node[left] {$\ytext$};

% origin
 \draw[gray] (0,0) node[anchor=north east] {$O$};

% axes
  \draw[gray,thick,<->] (\xone, 0) -- (\xtwo, 0) node[right] {$x$};
  \draw[gray,thick,<->] (0, \yone) -- (0, \ytwo) node[above] {$y$};
\end{scope}

% function
\begin{scope}[thick,red]
  \foreach \x in {16, 8, 4, 2, 1,.5,.25}
    \draw (16-\x, 16-\x) rectangle (16,16);

  \foreach \x in {16, 8, 4, 2, 1,.5,.25}
  \filldraw[thin,red,opacity=.3] (16-\x, 16-\x)
    rectangle (16-.5*\x,16-.5*\x);

\foreach \x in {16, 8, 4, 2, 1,.5,.25}{
  \filldraw[thin,blue,opacity=.2] (16-\x, 16-.5*\x)
    rectangle (16-.5*\x,16);
  \filldraw[thin,blue,opacity=.2] (16-.5*\x, 16-\x)
    rectangle (16,16-.5*\x);}
\end{scope}
\end{tikzpicture}
\end{document} 

