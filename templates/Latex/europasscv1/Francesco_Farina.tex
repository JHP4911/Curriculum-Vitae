\documentclass[helvetica,italian,logo,notitle,totpages,utf8]{europecv2013}
\usepackage{graphicx}
\usepackage[a4paper,top=1.2cm,left=1.2cm,right=1.2cm,bottom=2.5cm]{geometry}
\usepackage[italian]{babel}
\usepackage[T1]{fontenc}
\usepackage{relsize}
\usepackage{lipsum}

\newcommand\CPP{C\nolinebreak[4]\hspace{-.05em}\raisebox{.4ex}{\relsize{-3}{\textbf{++}}}}
\newcommand{\tra}[1]{``#1''} %virgolette


%[Tutti i campi del CV sono facoltativi. Rimuovere i campi vuoti.]
\ecvname{Francesco Farina}

\ecvaddress{Via Nofilo, 13 - 84080 Pellezzano (SA), Italia}
\ecvtelephone{+39 380 233 1336}
\ecvemail{mail.farinafrancesco@gmail.com}
%\ecvhomepage{\href{Sostituire con url a pagina personale}{Sostituire con url a pagina personale senza http://}}
%\ecvlinkedin{\href{http://it.linkedin.com/pub/bonaventura-del-monte/85/314/3ab}{it.linkedin.com/pub/bonaventura-del-monte/85/314/3ab}}
\ecvgender{Maschile}
%\ecvskype{venturadm91}
\ecvdateofbirth{17/10/1991}
\ecvnationality{Italiana}

\ecvfootnote{Ultimo aggiornamento: \today \\ Autorizzo il trattamento dei dati personali ai sensi del \href{http://www.garanteprivacy.it/garante/doc.jsp?ID=1311248}{D. lgs. 196/03}}
\ecvbeforepicture{\raggedleft}
\ecvpicture[width=2.5cm]{cv.png}
\ecvafterpicture{\ecvspace{-20mm}}

\begin{document}
\selectlanguage{italian}

\begin{europecv}
\ecvpersonalinfo[-2pt]

%\ecvposition{Posizione per la quale si concorre
%Posizione ricoperta
%Occupazione desiderata
%Titolo di studio per la quale si concorre}{Sostituire con posizione per la quale si concorre / posizione ricoperta / occupazione desiderata / titolo per il quale si concorre (eliminare le voci non rilevanti nella colonna di sinistra)}

%\ecvsection{Esperienza professionale}
%[Inserire separatamente le esperienze professionali svolte iniziando dalla più recente.]

%\ecvworkexperience{Sostituire con date (da - a)}{Sostituire con il lavoro o posizione ricoperta}{Sostituire con il nome del datore di lavoro}{Sostituire con l'indirizzo del datore di lavoro}{Sostituire con le principali attività e responsabilità}

\ecvsection{Istruzione e formazione}
%[Inserire separatamente i corsi frequentati iniziando da quelli più recenti.]

\ecvcurreducation[3pt]{12/2013 - in corso}{Laurea Magistrale in Informatica}{Università degli Studi di Salerno - Dipartimento di Informatica}{Specializzazione in reti informatiche, programmazione parallela e concorrente, grid e cloud computing, sistemi distribuiti, analisi ed integrazione dei dati, intelligenza computazionale ed artificiale, sicurezza e crittografia, compilatori, virtualizzazione, algoritmi avanzati, struttura delle reti sociali e robotica.}{VII Livello QEQ}{29.235 / 30}{13 / 13}
\\
\ecveducation[3pt]{09/2010 – 12/2013}{Laurea Triennale in Informatica}{Università degli Studi di Salerno - Dipartimento di Informatica}{Linguaggi di programmazione, sistemi operativi, algoritmi, strutture dati, reti di calcolatori, ingegneria del software, programmazione parallela, distribuita e su reti, web development e database design.}{VI Livello QEQ}{110/110 e Lode}{\tra{Progettazione assistita di simulazioni agent-based: l’architettura di Agent Modeling Platform}, relatore: Prof. V. Scarano}
\\
\ecvstdeducation[3pt]{09/2005 – 07/2010}{Diploma di Perito Tecnico Informatico}{Istituto Tecnico Industriale Statale \tra{Basilio Focaccia} (SA)}{Materie: informatica, sistemi, elettronica, calcolo statistico, matematica, inglese.}{V Livello QEQ}{95/100}

\ecvsection{Competenze personali}

\ecvmothertongue[3pt]{Italiano}
\ecvlanguageheader
\ecvlastlanguage{Inglese}{C1}{C1}{B2}{B2}{C1}


\ecvlanguagefooter[5pt]

\ecvitem[3pt]{Competenze comunicative}
{
Possiedo buone competenze comunicative acquisite grazie alla partecipazione a diversi progetti in team composti da più persone, durante la mia carriera universitaria.
}

\ecvitem[3pt]{Competenze organizzative e gestionali}
{
Ho lavorato in team composti da 2-8 persone in occasioni dei progetti universitari a cui ho preso parte (si faccia riferimento alla sezione \tra{ulteriori informazioni}), rivestendo in buona parte dei casi incarichi di management. Ho una buona capacità di scheduling del lavoro e di risoluzione dei problemi, anche in caso di scadenze a breve termine.
}

\ecvitem[3pt]{Competenze professionali}
{
L'esperienza acquisita negli anni, mi ha conferito capacità di problem solving da un punto di vista informatico, in particolare, la capacità di calarmi nel dominio del problema in maniera tale da fornire una possibile soluzione al problema rispettando le tempistiche. 
}

\ecvitem[3pt]{Competenze tecniche}
{
Utilizzo avanzato del personal computer, conoscenza avanzata dei sistemi operativi Windows e Unix-based. Ottima capacità di sviluppo di software utilizzando le metodologie proprie dell'ingegneria del software. Ottima conoscenza  di Java, conoscenza avanzata di C, Python, JavaScript, PHP, HTML, CSS, XML, JSON e BASH Scripting. Conoscenza avanzata della piattaforma Java SE, Node.js e dell'IDE Eclipse (JDT, CDT, ADT). Buona conoscenza di SQL, Matlab, Perl, \LaTeX e Markdown. Buona conoscenza delle seguenti librerie/framework: C Standard Lib, OpenMPI, Apache Hadoop, Java RMI, Qt Framework, Java Swing, jQuery, AngularJS, Twitter Bootstrap, Express, Apache Axis2, Java FLEX/CUP, Numpy, Pandas, Scikit-learn, NetworkX, Matplotlib. Buona conoscenza di Assembly MIPS e x86-64. Conoscenza base di Prolog, Nvidia CUDA, Haskell e della Continuous Integration. Conoscenza avanzata di Git version control. Conoscenza di base delle metodologie e degli strumenti del penetration testing.
}


\ecvitem[3pt]{Altre competenze}
{
La passione per la musica mi ha portato ad esplorare il panorama musicale alla ricerca di sonorità nuove, nel voler ottenere la miglior esperienza dall'ascolto, entrando a contatto con l'alta fedeltà, ed inoltre allo studio da autodidatta di chitarra classica ed elettrica raggiungendo un livello base, così come di software per produzione musicale. Conoscenza base dei programmi di image editing come Photoshop e GIMP.
{\par\vspace{3pt}
Durante gli anni della scuola secondaria di secondo grado ho conseguito due certificazioni di conoscenza della lingua inglese: Trinity ISE 1 (livello B1 del CEFR) e Trinity Grade 7 (livello B2 del CEFR).
}
}



\ecvsection{Ulteriori informazioni}

\ecvitem[5pt]{Progetti}{
{Ho realizzato, in collaborazione con un collega ed un assegnista di ricerca in farmacia, un sistema sviluppato in Python, in grado di determinare le condizioni/performance di un guidatore, calcolandone il livello di arousal, sfruttando i dati generati da un driver hypo-vigilance.}
{Per il corso magistrale di \tra{Sicurezza}, insieme a tre colleghi, ho sviluppato una survey su Smart Grid e la relativa sicurezza.}
{\par\vspace{1pt}Per il corso magistrale di \tra{Struttura delle reti sociali}, ho implementato ed applicato misure di centralità ed un modello di diffusione ad un campione di una rete reale, per valutarne l'influenza dei nodi più importanti.}
{\par\vspace{1pt}Durante il corso magistrale di \tra{Robotica}, insieme a due colleghi, ho realizzato un veicolo guidato da gesture, utilizzando la board Intel Galileo come controller, programmandola in C++.}
{\par\vspace{1pt}Insieme ad altri tre colleghi, per il corso magistrale di \tra{Integrazione dati sul web}, ho realizzato un'applicazione web per la visualizzazione delle informazioni legate alla NBA, recuperandole da numerosi siti ed integrandole, utilizzando Node.js, Express ed AngularJS, effettuandone il deploy su Red Hat OpenShift.}
{\par\vspace{1pt}Ho sviluppato un compilatore completo durante il corso magistrale di \tra{Linguaggi di programmazione e compilatori} per il linguaggio COOL, in collaborazione ad altri colleghi, realizzando i moduli per l’analisi lessicale, sintattica e semantica.}
{\par\vspace{1pt}Ho sviluppato, insieme ad altre cinque persone per il corso magistrale di \tra{Sistemi operativi avanzati}, un'applicazione scalabile basata sul paradigma MapReduce e sul framework Apache Hadoop2 per l’allineamento di sequenze di genomi e di proteomi. Inoltre ho configurato e gestito per circa tre mesi il cluster Hadoop (composto da quaranta macchine) presso i laboratori del Dip. di Informatica dell'Università di Salerno.}
{\par\vspace{1pt}Per il corso magistrale di \tra{Reti di calcolatori II} ho contribuito, insieme ad altri due colleghi, al progetto NoTrace, già in sviluppo presso il laboratorio in cui ho conseguito il tirocinio. Trattasi di un'estensione per il browser web Firefox che protegge la privacy dell'utente. È stata realizzata la visualizzazione grafica delle informazioni perse durante la navigazione e per tale scopo mi sono servito delle librerie Sigma.js e Twitter Bootstrap.}
{\par\vspace{1pt}Durante il mio lavoro di tesi triennale ho realizzato, insieme ad un secondo tesista, l'integrazione del supporto alla libreria Mason nella Agent Modeling Platform, un sistema di progettazione visuale di simulazioni ad agenti basato su Eclipse ed i suoi plug-in, utilizzando tecnologie Java, Xpand, Xtend, EMF e PDT.}
{\par\vspace{1pt}Durante il corso di \tra{Ingegneria del software}, ho realizzato, insieme a due colleghi, un'applicazione web per la condivisione di documenti testuali. Tale sistema si basa su un backend PHP e MySQL, utilizzando HTML5, CSS3 e JavaScript nel frontend.}
{\par\vspace{1pt}Nell'ultimo anno, ho studiato la piattaforma Node.js ed i suoi framework più importanti. Ho realizzato cinque moduli, pubblicandoli su npmjs.org. Lo sviluppo è stato supportato dal versioning di git, hostando il codice su GitHub. La pubblicazione su npmjs.org ha comportato la comprensione delle diverse licenze open source e l'attuazione del meccanismo di deploy dei moduli Node.js.}
}

\ecvitem[3pt]{Dati personali}{Autorizzo il trattamento dei miei dati personali ai sensi del Decreto Legislativo 30 giugno 2003, n. 196 \tra{Codice in materia di protezione dei dati personali} \emph{(firma)}.}



%\ecvsection{Allegati}

%\ecvitem[10pt]{}{Sostituire con la lista di documenti allegati al CV. Esempio:
%\begin{itemize}
%\item copie delle lauree e qualifiche conseguite; 
%\item attestazione di servizio;
%\item attestazione del datore di lavoro.
%\end{itemize}}

\end{europecv}
\end{document} 