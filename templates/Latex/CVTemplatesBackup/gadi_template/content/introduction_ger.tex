Albert Einstein (* 14. März 1879 in Ulm; † 18. April 1955 in Princeton, USA) war ein deutscher Physiker jüdischer Herkunft, dessen Beiträge zur theoretischen Physik maßgeblich das physikalische Weltbild veränderten (Autor, 1801). 

Einsteins Hauptwerk ist die Relativitätstheorie, die das Verständnis von Raum und Zeit revolutionierte. Im Jahr 1905 erschien seine Arbeit mit dem Titel ``Zur Elektrodynamik bewegter Körper'', deren Inhalt heute als spezielle Relativitätstheorie bezeichnet wird. 1916 publizierte Einstein die allgemeine Relativitätstheorie. Auch zur Quantenphysik leistete er wesentliche Beiträge: Für seine Erklärung des photoelektrischen Effekts, die er ebenfalls 1905 publiziert hatte, wurde ihm 1921 der Nobelpreis für Physik verliehen. Seine theoretischen Arbeiten spielten – im Gegensatz zur populären Meinung – beim Bau der Atombombe und der Entwicklung der Kernenergie keine bedeutende Rolle.

Albert Einstein gilt als Inbegriff des Forschers und Genies. Er nutzte jedoch seinen erheblichen Bekanntheitsgrad auch außerhalb der naturwissenschaftlichen Fachwelt bei seinem Einsatz für Völkerverständigung und Frieden. In diesem Zusammenhang verstand er sich selbst als Pazifist, Sozialist und Zionist.