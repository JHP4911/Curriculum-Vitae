\documentclass[10pt]{article}
\usepackage{ngerman}
\usepackage[latin1]{inputenc}
\usepackage{esa}
\usepackage{url}
\usepackage{cite}
\pagestyle{plain}

\begin{document}

\title{Titel der Arbeit}

\author{{\bfseries Vorname Nachname}\\
   (Technische Universit�t Graz, �sterreich\\
   email.adresse@student.tugraz.at)\\
}
\maketitle
%%%%%%%%%%%%%%%%%%%%%%%%%%%%%%%%%%%%%%%%%%%%%%%%%%%%%%%%%%%%%%%%%%%%%%%%%%%%%%%
%%%%%%%%%%%%%%%%%%%%%%%%%%%%%%%%%%%%%%%%%%%%%%%%%%%%%%%%%%%%%%%%%%%%%%%%%%%%%%%
\begin{abstract}
Die Kurzfassung sollte eine kurze Zusammenfassung der Arbeit in ca. 200 bis 
300 W�rtern sein. 
\end{abstract}

\begin{keywords}
Schl�sselwort1, Schl�sselwort2, Schl�sselwort3
\end{keywords}

% \tableofcontents % kann optional aktiviert werden

%%%%%%%%%%%%%%%%%%%%%%%%%%%%%%%%%%%%%%%%%%%%%%%%%%%%%%%%%%%%%%%%%%%%%%%%%%%%%%%
\section{Einleitung}
%%%%%%%%%%%%%%%%%%%%%%%%%%%%%%%%%%%%%%%%%%%%%%%%%%%%%%%%%%%%%%%%%%%%%%%%%%%%%%%

In der Einleitung soll die Aufmerksamkeit des Lesers gewonnen und der Inhalt
der Arbeit n�her beschrieben werden. Hier sind auch vorw�rtsverweise auf sp�tere
Kapitel erlaubt. 

%%%%%%%%%%%%%%%%%%%%%%%%%%%%%%%%%%%%%%%%%%%%%%%%%%%%%%%%%%%%%%%%%%%%%%%%%%%%%%%
\section{Erster Abschnitt}
%%%%%%%%%%%%%%%%%%%%%%%%%%%%%%%%%%%%%%%%%%%%%%%%%%%%%%%%%%%%%%%%%%%%%%%%%%%%%%%

Hier beginnen die eigentlichen Inhalte. Zitate werden in folgender Form 
geschrieben:
"`Es war M�rz oder vielleicht April, und wenn der Schnee in der Petersgatan sich
in Matsch verwandelt haben sollte, habe ich es weder gemerkt, noch hat es mich
sonderlich interessiert. Ich verbrachte meine Zeit haupts�chlich im Bademandel,
�ber meinen unattraktiven neuen Computer gebeugt, hinter dichtgewebten schwarzen
Vorh�ngen, die mich gegen die Sonne und vor allem gegen die Au�enwelt
abschirmten. Ich kratzte die monatlichen Raten f�r meinen PC zusammen, der in
drei Jahren abgezahlt werden sollte. Damals wusste ich noch nicht, dass ich die
Raten nur noch ein Jahr lang w�rde aufbringen m�ssen. Da n�mlich w�rde ich Linux
geschrieben und sehr viel mehr Leute als Sara und Lars w�rden es gesehen haben.
Und Peter Anvin, der heute wie ich bei Transmeta arbeitet, w�rde eine Sammlung
im Internet gestartet haben, um das Geld f�r meinen Computer
aufzutreiben."'\cite{linus}

Siehe auch \cite{wp:just_for_fun}.
%%%%%%%%%%%%%%%%%%%%%%%%%%%%%%%%%%%%%%%%%%%%%%%%%%%%%%%%%%%%%%%%%%%%%%%%%%%%%%%
\section{Zweiter Abschnitt}
%%%%%%%%%%%%%%%%%%%%%%%%%%%%%%%%%%%%%%%%%%%%%%%%%%%%%%%%%%%%%%%%%%%%%%%%%%%%%%%

%%%%%%%%%%%%%%%%%%%%%%%%%%%%%%%%%%%%%%%%%%%%%%%%%%%%%%%%%%%%%%%%%%%%%%%%%%%%%%%
\section{und so weiter}
%%%%%%%%%%%%%%%%%%%%%%%%%%%%%%%%%%%%%%%%%%%%%%%%%%%%%%%%%%%%%%%%%%%%%%%%%%%%%%%

%%%%%%%%%%%%%%%%%%%%%%%%%%%%%%%%%%%%%%%%%%%%%%%%%%%%%%%%%%%%%%%%%%%%%%%%%%%%%%%
\section{Zusammenfassung und Ausblick}
%%%%%%%%%%%%%%%%%%%%%%%%%%%%%%%%%%%%%%%%%%%%%%%%%%%%%%%%%%%%%%%%%%%%%%%%%%%%%%%

%%%%%%%%%%%%%%%%%%%%%%%%%%%%%%%%%%%%%%%%%%%%%%%%%%%%%%%%%%%%%%%%%%%%%%%%%%%%%%%
%%%%%%%%%%%%%%%%%%%%%%%%%%%%%%%%%%%%%%%%%%%%%%%%%%%%%%%%%%%%%%%%%%%%%%%%%%%%%%%
\begin{thebibliography}{10}
\bibitem[Torvalds, Diamond 2001]{linus} Torvalds L., Diamond D.:
"`Just for Fun -- Wie ein Freak die Computerwelt revolutionierte"';
Carl Hanser, M�nchen / Wien (2001)
\bibitem[Wikipedia: Just For Fun 2006]{wp:just_for_fun}
\url{http://de.wikipedia.org/wiki/Just_for_Fun} (1.1.2007, Wikipedia)
\end{thebibliography}

\end{document}
