\documentclass[fontsize=12pt]{scrlttr2} 
\usepackage[latin1]{inputenc}
\usepackage[T1]{fontenc}
\usepackage{%
	ngerman,
	ae,
	times,  %% hier kann man die Schriftart einstellen
	graphicx,
	url}
 
% Hier die Optionen fuer scrlttr2 eintragen. 
% S.a. scrguide.pdf ab Seite 150
\KOMAoptions{paper=a4,fromalign=center,fromrule=aftername,
backaddress=true,parskip=half,enlargefirstpage=true} 
 
% Falls man das alte scrlettr vemisst, oder die Option 
% "enlargefirstpage=true" aus den KOMAoptions 
% benutzen will, muss man den folgenden Eintrag 
% auskommentieren (einfach das %-Zeichen löschen).
%\LoadLetterOption{KOMAold}
 
% hier Name und darunter Anschrift einsetzen:
\setkomavar{fromname}{Maxi Muster} 		
\setkomavar{fromaddress}{Musterstr. 20\\
                        12345 Musterstadt} 
 
% hier die Signatur einsetzen:
\setkomavar{signature}{Maxi Muster}		
 
% hier einsetzen, als was man sich bewirbt:
\setkomavar{subject}{Bewerbung als ...}
 
% hier kommt dein Ort hin:
\setkomavar{place}{Musterstadt} 	
 
% die Signatur ist linksbuendig
\let\raggedsignature=\raggedright		
 
% Manche finden, dass scrlttr2 so eine riesige Fußzeile hat 
% einfach die nächste Zeile auskommentieren, dann wird sie kleiner:
% \setlength{\footskip}{-6pt}	    
 
\begin{document}
 % die Anschrift des Empfaengers
 \begin{letter}{Muster GmbH \& Co. KG\\- Personalabteilung -\\
		     Landstra"se 2\\12345 Musterstadt} 		
 
\opening{Sehr geehrte Damen und Herren,}
 
hier folgt der erste Absatz, der auch gleichzeitig die 
\textbf{Einleitung} darstellt. Am besten kommt man gleich zur 
Sache: Warum interessiert mich diese Stelle, und warum halte
ich mich f"ur geeignet.
 
Im zweiten Absatz beginnt der \textbf{ Hauptteil}. Hier stellt 
man sich vor, und hier sollte man anhand von Qualifikationen 
und Erfahrungen belegen, warum man die Anforderungen
erf"ullt. Im Hauptteil sollte man auch pers"onliche Qualit"aten 
erw"ahnen: Welche Hard und Soft Skills bringe ich mit (ich bin 
teamf"ahig, flexibel, etc.).
 
Der letzte Absatz geh"ort dem \textbf{Schluss}. Hier bekundet 
man nocheinmal sein Interesse, sowie die Reaktion, die man sich 
w"unscht. ("`"Uber eine Einladung zu einem pers"onlichen 
Gespr"ach w"urde ich mich sehr freuen."')
 
\closing{Mit freundlichen Gr"u"sen}
 
% Der Anhang:
 
\encl{%
         Lebenslauf \\
         Zeugnisse}
 
 
\end{letter}
\end{document}