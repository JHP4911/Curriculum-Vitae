
\documentclass[11pt]{dinbrief}

\usepackage[T1]{fontenc}

\usepackage[applemac]{inputenc}%wenn ihr auch einen Mac verwendet

\usepackage[ngerman]{babel}

\usepackage{currvita}

\usepackage{graphicx}

% Kopf?/Fu?zeile

\setlength{\unitlength}{1cm}

\renewcommand{\labelitemi}{\labelitemii}

\pagestyle{empty}

\setlength{\parindent}{0pt}%keinen Absatzeinzug

\nowindowtics %kein Falz am Rand

\nowindowrules %keine Linien �uber und unter der Adresse

\date{}% Muss leer sein sonst gibts ein Datum in die seite

\begin{document}

% Lebenslauf

%siehe package currvita

\begin{cv}{Lebenslauf}

\begin{cvlist}{Pers�nliche Daten}

\item[Anschrift] Frau Johanna Musterfrau \\

Muster Stra�e 123\\

44568 Musterhausen

\item[Telefon](01 23 4) 5 67 89

\item[E-Mail]deine@email-adresse.xy

\item[Homepage]http://www.deine-homepage.de

\item[Geboren]01.01.1900 in Lummerland

\item[Staatsang.]deine Staatsangeh�rigkeit

\item[Familienstand]dein Familienstand

\end{cvlist}

\begin{cvlist}{Schulische Ausbildung}

\item[Monat/Jahr�Monat/Jahr]Grundschule Musterhausen

\item[Monat/Jahr�Monat/Jahr]Deine weiterf�hrende Schule Musterhausen

\item[Monat/Jahr]Dein Abschluss

\end{cvlist}

\begin{cvlist}{Ausbildung/Studium/Was auch immer}

\item[seit Monat/Jahr]Studium der Musterwissenschaften an der Muster-Universit�t Musterhausen

\item[Monat/Jahr]falls Du noch wichtige Zwischenstationen erw�hnen m�chtest,

\item[Monat/Jahr]kannst Du das hier

\item[Monat/Jahr]beliebig oft tun

\item[Monat/Jahr]Abschluss der Musteruniversit�t (Note: x.y)

\end{cvlist}

\begin{cvlist}{Sprachkenntnisse}

\item[Monat/Jahr�Monat/Jahr]Englisch

\item[Monat/Jahr�Monat/Jahr]Russisch

\item[Monat/Jahr�Monat/Jahr]Hebr�isch AG

\item[Semester Jahr] Zertifikat in Spanisch

\end{cvlist}

\cvplace{Deine Stadt}

\date{den \today}%setzt das heutige Datum ein

\end{cv}

\end{document}
Der Grundaufbau besteht aus dem Listentyp �cvlist�, dessen Sparten�berschrift (im obigen Beispiel Pers�nliche Daten, Schulische Ausbildung, Ausbildung/Studium, Sprachkenntnisse usw.) absolut frei gew�hlt werden k�nnen. Die Listenanzahl sowie die \item-Anzahl innerhalb einer Liste ist auch frei w�hlbar, wobei die M�glichkeit besteht, einen Zeitraum oder Zusatzinfo wie in den description-Listen in eckigen Klammern anzugeben:

\item[Datum oder Beschreibung]Hier das Ereignis

Lebenslauf plus Anschreiben gleich Kurzbewerbung

Wie bereits gesagt, gibt es mit currvita auch die M�glichkeit, direkt eine ganze Kurzbewerbung mit allem drum und dran zu erstellen.

Dabei �ndert sich im Prinzip nichts am Aufbau der Bewerbung:

Deckblatt mit Passfoto

F�r ein Deckblatt, welches vor dem Anschreiben erscheint, wird folgender Code direkt nach \begin{document} eingef�gt.

\address{}%Bitte LEER lassen

\begin{letter}{}%Bitte LEER lassen

\opening{}%Bitte LEER lassen

%das wird ein Deckblatt

\vspace{-9cm}

\begin{center}\textsc{\Huge Bewerbung}\end{center}

\hrule

\vspace{4cm}

\begin{center}

\fbox{\resizebox{6cm}{!}{\includegraphics{passfoto}}}%hier bitte den Namen deiner Passfotodatei angeben, als jpg und ohne Endung

\end{center}

\begin{center}

\begin{large}

Frau Johanna Musterfrau \\

Muster Stra�e 123\\

44568 Musterhausen\par

Telefon: (01 23 4) 5 67 89 \\

E-Mail: deine@email-adresse.xy

\end{large}

\end{center}

\closing{} %Bitte LEER lassen

\end{letter}

%Deckblatt Ende

Anschreiben

M�chte man nun auch noch den Bewerbungsbrief einf�gen, geht das mit folgendem Code, der � wenn vorhanden � nach dem Deckblatt und vor dem CV eingef�gt wird:

% Hier f"angt jetzt der Berwerbungsbrief an siehe dinbrief package

\place{Musterhausen}

\date{den \today}

\signature{$\overline{\hspace{1.5 cm}\textnormal{\textit{Johanna Musterfrau}}\hspace{1.5 cm}}$}

\centeraddress

\enabledraftstandard

%hier muss die Gr�o�se von tabular auf deinen Namen so angepasst werden, dass die Seitenbreite voll ist aber keine Warnung wegen einer zu vollen Box kommt. Einfach ausprobieren. Du kannst hier aber auch selbst etwas reinschreiben.

\bottomtext{

\begin{minipage}[t]{16.5cm}

\rule[1mm]{16.5cm}{0.4pt}

\fontsize{8pt}{1ex}\selectfont

%Achtung, hier beginnt eine Tabelle

\begin{tabular}{llp{6.9cm}ll}

Telefon:&(01 23 4) 5 67 89 \\

E-Mail:&deine@email-adresse.xy\\

\end{tabular}

\end{minipage}

}%bottomtext

\subject{\textbf{Bewerbung}}

\begin{letter}{Hier Der Adressat\\ und seine\\ umf�ngliche\\Adresse\\[1ex]PLZ Stadt des Adressaten}

\opening{Sehr geehrte/r Frau/Herr Adressat,}

hiermit bewerbe ich mich um eine der ausgeschriebenen Stellen als Musterarbeiter.\par

Ich bin ganz toll f�r diesen Job, weil � \par

Und falls das Anschreiben etwas l�nger wird � \par

Ist auch Platz f�r noch einen Absatz �. \par

Hier ein letzter Satz, dass Du Dich auf ein pers�nliches Gespr�ch freust (bitte keinen Konjunktiv!).

\closing{Mit freundlichen Gr��en/Hochachtungsvoll/eine noch kreativere Gru�formel}

\vfill%wenn die �Anlagen� am unteren Rand ausgerichtet sein sollen

\encl{%Anlagen

Kurzbewerbung}%oder auch: Vollst�ndige Bewerbungsunterlagen, je nach dem

\end{letter}